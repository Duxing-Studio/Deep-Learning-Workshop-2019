\section{操作符重载}
我们之前已经看到+运算符不但能加数字,还能连接字符串。这之所以可能,是以为+运算符在int类和str类中都被重载。运算符实际上对应着类中相应的方法。为运算符定义方法就是所谓的运算符重载。比如,为让自定义对象能使用+运算符,我们需要定义名叫\_\_add\_\_的方法。

让我们看个例子:

\begin{lstlisting}
import math


class Circle:

    def __init__(self, radius):
        self.__radius = radius

    def setRadius(self, radius):
        self.__radius = radius

    def getRadius(self):
        return self.__radius

    def area(self):
        return math.pi * self.__radius ** 2

    def __add__(self, another_circle):
        return Circle(self.__radius + another_circle.__radius)

c1 = Circle(4)
print(c1.getRadius())

c2 = Circle(5)
print(c2.getRadius())

c3 = c1 + c2  # 之所以能使用加法运算符,是因为我们定义了__add__方法
print(c3.getRadius())

\end{lstlisting}

在上面的例子中,我们为类添加了\_\_add\_\_方法,该方法允许使用+运算符对两个circle对象求和。在\_\_add\_\_方法中,我们创建了一个新的对象,并将其返回给调用者。运行结果如下:

\begin{lstlisting}
4
5
9
\end{lstlisting}

在Python中,除\_\_add\_\_方法对应+运算符之外,还有其他能够重载运算符的方法:如\_\_mul\_\_、\_\_sub\_\_等等\cite{thepythonguru,2015}。
