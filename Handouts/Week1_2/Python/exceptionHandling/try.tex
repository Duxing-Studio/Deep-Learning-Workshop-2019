\section{捕获异常}
异常处理可以使开发人员能以优雅的方式处理错误。

\subsection{try-except}
Python使用try-except语句处理异常。语法如下:

\begin{lstlisting}
try:
    # write some code
    # that might throw exception
except <ExceptionType>:
    # Exception handler, alert the user
\end{lstlisting}

在try语句块中,我们写入可能会产生异常的代码,当异常发生时,try语句块中的代码会被忽略,转而进入except语句块中处理异常。例如:

\begin{lstlisting}
try:
    f = open('somefile.txt', 'r')
    print(f.read())
    f.close()
except IOError:
    print('file not found')
\end{lstlisting}

上述代码的执行流程如下:
\begin{enumerate}
	\item 先执行介于try和except之间的语句。
	\item 如果没有异常,则except语句块中的代码会被跳过。
	\item 如果文件不存在,则产生异常,在try语句块中的其他代码会被跳过。
	\item 当异常发生时,如果异常类型和except关键字后的异常名称匹配,就执行except分支中的代码。上述代码中只能处理IOError异常,如要处理其他类型的异常,还需要添加更多的except分支。
\end{enumerate}

\subsection{多个except}
try声明可以有多个except分支,它还可以选择else和finally分支。语法如下:

\begin{lstlisting}
try:
    <body>
except <ExceptionType1>:
    <handler1>
except <ExceptionTypeN>:
    <handlerN>
except:
    <handlerExcept>
else:
    <process_else>
finally:
    <process_finally>
\end{lstlisting}

except分支类似于elif。当异常发生时,将检查except分支是否和异常类型匹配。如果匹配,就执行对应except分支中的代码。在最后一个except分支中,异常类型是被忽略了的。如果异常发生,但没有匹配到最后一个except之前的分支,则最后的except分支中的代码会被执行。如果没有任何异常发生,则执行else语句中的代码。finally分支中的代码,无论是否有异常发生,都会被执行。例如:

\begin{lstlisting}
try:
    num1, num2 = eval(input("Enter two numbers, separated by a comma : "))
    result = num1 / num2
    print("Result is", result)

except ZeroDivisionError:
    print("Division by zero is error !!")

except SyntaxError:
    print("Comma is missing. Enter numbers separated by comma like this 1, 2")

except:
    print("Wrong input")

else:
    print("No exceptions")

finally:
    print("This will execute no matter what")
\end{lstlisting}

\begin{myremark}{eval()}
eval()函数允许在Python程序内部运行Python代码,了解更多关于eval()的信息,请访问\url{http://stackoverflow.com/questions/9383740/what-does-pythons-eval-do}
\end{myremark}

\subsection{自定义异常}
使用关键字raise,可以在方法中自定义异常。语法为:

\begin{lstlisting}
raise ExceptionClass("Your argument")
\end{lstlisting}

例如:

\begin{lstlisting}
def enterage(age):
    if age < 0:
        raise ValueError("Only positive integers are allowed")

    if age % 2 == 0:
        print("age is even")
    else:
        print("age is odd")

try:
    num = int(input("Enter your age: "))
    enterage(num)

except ValueError:
    print("Only positive integers are allowed")
except:
    print("something is wrong")
\end{lstlisting}
当用户输入的年龄小于0时,程序显示结果为:

\begin{lstlisting}
only positive integers are allowed
\end{lstlisting}
