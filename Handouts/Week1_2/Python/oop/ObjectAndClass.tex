\section{Python对象和类}

\subsection{创建类}
Python一门面向对象的语言。在Python中所有的东西都是对象,比如之前学习的整型、字符串等等,甚至模块、函数也都是对象。

面向对象编程时使用对象创建程序,使用对象存储数据和行为。

在Python中,使用关键字class定义类。类通常包括数据区域,用以存数数据和方法的定义。Python中的所有类,都包含一个特殊的方法,叫作初始化(initializer),或者叫作构造方法。构造方法会在使用类创建新的对象时自动执行。例如:

\begin{lstlisting}
class Person:

      # 构造函数
      def __init__(self, name):
            self.name = name

      # 定义方法
     def whoami(self):
           return "You are " + self.name
\end{lstlisting}

上述代码中,我们创建了一个名叫Person的类,这个类中包含数据字段name和方法whoami()。
\begin{myremark}{What is self ??}
Python中的所有方法,包括构造方法,首个参数都是self。这个参数指向对象本身。当我们创建一个新的对象时候,self参数就会自动指向新创建的对象。
\end{myremark}

\subsection{从类中创建对象}
使用类名就可创建对象。当我们调用方法时,不需要传递self参数,Python会自动传递。例如:
\begin{lstlisting}
p1 = Person('tom')
print(p1.whoami())
print(p1.name)
\end{lstlisting}

输出结果为:

\begin{lstlisting}
You are tom
tom
\end{lstlisting}

我们还可以改变数据字段的值:

\begin{lstlisting}
p1.name = 'jerry'
print(p1.name)
\end{lstlisting}

输出结果为jerry。然而,像这样从类的外部获取数据字段,属于不太好的操作方式,下面我们看如何阻止这种操作。

\subsection{隐藏数据字段}
为了隐藏数据字段,我们需要定义私有数据字段。在Python中,使用两个前置下划线,就可定义私有数据字段和私有方法。比如:

\begin{lstlisting}
class BankAccount:

     # 构造函数
    def __init__(self, name, money):
        self.__name = name  # 定义私有数据字段
        self.__balance = money  # 定义私有数据字段

    def deposit(self, money):
        self.__balance += money

    def withdraw(self, money):
        if self.__balance > money:
            self.__balance -= money
            return money
        else:
            return "Insufficient funds"

    def checkbalance(self):
        return self.__balance

b1 = BankAccount('tim', 400)
print(b1.withdraw(500))
b1.deposit(500)
print(b1.checkbalance())
print(b1.withdraw(800))
print(b1.checkbalance())
\end{lstlisting}

在上述代码中,我们定义了BankAccout类,这个类有两个数据字段,但都是私有字段。代码运行结果为:

\begin{lstlisting}
Insufficient funds
900
800
100
\end{lstlisting}

现在,让我们尝试访问私有数据字段:

\begin{lstlisting}
print(b1.__balance)
\end{lstlisting}

结果显示:

\begin{lstlisting}
AttributeError: 'BankAccount' object has no attribute '__balance'
\end{lstlisting}

这就表明,设置为私有的数据字段,无法在类的外部访问。
