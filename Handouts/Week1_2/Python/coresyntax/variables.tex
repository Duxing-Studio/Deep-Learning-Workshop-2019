\section{Python中的变量}

\subsection{注释}

在Python中,使用“\#”标记注释。注释不会被Python解释器执行。注释是开发人员用来提醒自己或他人程序如何工作的重要手段,注释还会用在文档的写作中。

\begin{lstlisting}
#display hello world
print("hello world")
\end{lstlisting}

上述代码将会打印出hello world字符串。


\begin{myremark}{物理行与逻辑行}

所谓物理行(Physical Line)是你在编写程序时 你所看到 的内容。所谓逻辑行(Logical Line)是 Python 所看到 的单个语句。Python 会假定每一 物理行 会对应一个 逻辑行。
有关逻辑行的一个例子是诸如 print('hello world') 这样一句语句——如果其本身是一行(正如你在编辑器里所看到的那样),那么它也对应着一行物理行。

Python 之中暗含这样一种期望:Python 鼓励每一行使用一句独立语句从而使得代码更加可读。
\end{myremark}
\subsection{变量命名}

变量(Vaiable)实质上是对内存中地址的命名,在内存中存储着诸多对象,为了方便使用这些对象,便有了变量。把变量和函数的名称我们叫作标识符(Identifier)。在Python中,标识符必须遵守以下规则:

\begin{enumerate}
	\item 所有标识符必须以字母或者下划线( \_ )开头,不能以数字开头。如 my\_var 就是一个有效的标识符,而 1digit 就不是。
	\item 标识符可以包含字母、数字和下划线。标识符不限长度。
	\item 标识符不能是关键字(所谓关键字,就是Python中已经使用并有特定含义的单词)。Python的关键字参见附录\ref{Python关键字}。
\end{enumerate}

\subsection{变量赋值}

值(Value)是程序运行过程中的基本元素之一,例如1,3.14,"hello"等等都是值。在编程属于中,它们又被叫作字面量(literals)。字面量拥有不同的类型,如1是整型(int),3.14是浮点型(float),"hello"是字符串(string)。在之后的章节中,我们将详细学习数据类型。

在Python中,无需声明变量类型,解释器会根据变量的值自动判断变量类型。使用等于号为变量赋值,等于号也被认为赋值操作符(operator)。以下是变量声明的一些例子:


\begin{lstlisting}
x = 100                       # x 是整型
pi = 3.14                     # pi 是浮点类型
empname = "python is great"   # empname 是字符串
a = b = c = 100               # 将100赋值给a、b、c
\end{lstlisting}

注意,变量x中并不储存100自身,它存储的是100(它是一个整型对象)的引用(reference)地址。

\subsection{同步赋值}

Python可以使用以下语法对多个变量同步赋值:

\begin{lstlisting}
var1, var2, ..., varn = exp1, exp2, ..., expn
\end{lstlisting}

上述声明告诉Python,将表达式右边的值依次赋值给表达式左侧的变量。同步赋值在要交换两个变量的值时非常有用。例如:

\begin{lstlisting}
>>> x = 1
>>> y = 2

>>> y, x = x, y # 交换x、y的值

>>> print(x)
2
>>> print(y)
1
\end{lstlisting}
\begin{myremark}{}
$>>>$是Python交互模式中的提示符。
\end{myremark}
\subsection{数据类型}
Python拥有6种标准数据类型。
\begin{enumerate}
	\item Numbers,数字
	\item String,字符串
	\item List,列表
	\item Tuple,元组
	\item Dictionary,字典
	\item Boolean,布尔值
\end{enumerate}

\subsection{从控制台中接受输入值}
input()函数用来接受从控制台输入的值。它的用法如下:

\begin{lstlisting}
input([prompt]) -> string
\end{lstlisting}

\begin{myremark}{使用input()函数和用户进行交互}
input()函数接受一个可选的字符串变量,用以提示输入内容,该函数返回值为字符串。
\end{myremark}

例如:

\begin{lstlisting}
>>> name = input("Enter your name: ")
>>> Enter your name: tim
>>> name
'tim'
\end{lstlisting}

\subsection{引入模块}
Python使用模块(module)组织代码。Python内置了许多常用的模块,比如math模块用来处理数学运算,re模块用来处理正则表达式等等。但是在使用这些模块之前,你需要使用以下语法引入这些模块:

\begin{lstlisting}
import module_name
\end{lstlisting}

你还可以使用以下语法导入多个模块:

\begin{lstlisting}
import module_name_1, module_name_2
\end{lstlisting}

例如:

\begin{lstlisting}
>>> import math
>>> math.pi
3.141592653589793
\end{lstlisting}

上述代码中的第一行引入了math模块,这样我们就可以使用math模块中的所有函数、类、变量和常量。为了使用math模块中的这些内容,我们需要在模块名称后使用( . ),然后就可以使用模块中定义的类、函数、常量或者变量了。在上面的例子中,math.pi表示math模块中的pi常量。
