\section{函数}

函数是可重用的代码块,使用函数可以帮助我们组织代码的结构。我们创建函数的目的,是能在程序运行中多次使用一系列代码,而不用重复书写代码。

\subsection{创建函数}
Python使用def关键词创建函数,语法如下:

\begin{lstlisting}
def function_name(arg1, arg2, arg3, .... argN):
     #statement inside function
\end{lstlisting}

\begin{myremark}{缩进}
空白区在 Python 中十分重要。实际上,空白区在各行的开头非常重要。这被称作 缩进(Indentation)。在逻辑行的开头留下空白区(使用空格或制表符)用以确定各逻辑行的缩进级别,而后者又可用于确定语句的分组。

这意味着放置在一起的语句必须拥有相同的缩进。每一组这样的语句被称为 块(block)。有一件事你需要记住:错误的缩进可能会导致错误。

所有在函数内部的声明,都必须使用相等的缩进。函数可以没有参数,也可以有多个参数。多个参数之间用逗号隔开。还可以使用pass关键字忽略掉函数主题的声明。
\end{myremark}

我们看一个函数的例子,下面的函数将计算指定范围的整数之和:

\begin{lstlisting}
def sum(start, end):
    result = 0
    for i in range(start, end + 1):
        result += i
    print(result)

sum(1, 10)
\end{lstlisting}

在上面的代码中,我们定义了一个叫作sum()的函数,该函数有两个参数(start和end),该函数将从start开始,累加到end,最后打印出累积之和。代码运行的结果为55。

\subsection{函数返回值}
上文定义的函数只是简单地在控制台打印出结果,如果我们想要将计算结果赋值给变量,以便做更深入的处理时应该怎么办?当我们遇到这种情况时,可使用return语句,将返回函数计算结果并且退出函数。例如:

\begin{lstlisting}
def sumReturn(start, end):
    result = 0
    for i in range(start, end + 1):
        result += i
    return result

a = sumReturn(1, 5)
print(a)
\end{lstlisting}

在上面这段代码中,我们定义了有返回值的函数sumReturn(),并将其结果赋值给变量a。上面代码的运行结果为15。

当然,return语句也可以不返回值,而是用来退出函数(实际上会返回None,为NoneType对象)。每一个函数都在其末尾隐含了一句 return None,除非你写了你自己的 return 语句。

\begin{lstlisting}
def sum2(start, end):
    if(start > end):
        print("start should be less than end")
        return
    result = 0
    for i in range(start, end + 1):
        result += i
    return result

s = sum2(110, 50)
print(s, type(s))
\end{lstlisting}

上述代码的运行结果如下:

\begin{lstlisting}
start should be less than end
None <class 'NoneType'>
\end{lstlisting}

在Python中,如果你不指定return的返回值,则会返回None值。

\subsection{全局变量和局域变量}

全局变量指的是不属于任何函数,但又可以在函数内外访问的变量。而局域变量指的是在函数内部声明的变量,局域变量只能在函数内部使用,无法在函数外访问(函数执行完后,会销毁内部定义的局部变量)。

下面我们通过例子来演示这两者的区别:

\begin{lstlisting}
global_var = 12         # 定义全局变量

def func():
    local_var = 100     # 定义局部变量
    print(global_var)   # 可以在函数内部访问全局变量

func()                  # 调用函数func()

print(local_var)       	# 无法访问变量local_var
\end{lstlisting}

上述代码将会出现错误:

\begin{lstlisting}
NameError: name 'local_var' is not defined
\end{lstlisting}

我们再看一个例子:

\begin{lstlisting}
xy = 100								# 定义全局变量xy

def func():
    xy = 200						# 定义局部变量xy
    print(xy)						# 此时访问的是局部变量xy

func()                  # 调用函数func()
\end{lstlisting}

该代码显示的结果是200,不是100。

使用global关键字,可以将局部变量同全局变量绑定在一起。例如:

\begin{lstlisting}
t = 1

def increment():
    global t   	# 现在的变量t在函数内外都是一致的
    t = t + 1
    print(t) 		# 输出 2

increment()
print(t) 				# 输出 2
\end{lstlisting}

\begin{myremark}{}
使用global关键字声明全局变量时,无法直接赋值,比如“global t = 1”的写法存在语法错误。
\end{myremark}

\subsection{参数的默认值}
为参数指定默认值,只需在定义函数时使用赋值语句即可。例如:

\begin{lstlisting}
def func(i, j = 100):
    print(i, j)
\end{lstlisting}

上述定义的函数func()有两个参数i和j。j的默认值为100,这意味着我们在调用这个函数的时候可以忽略掉j的值,比如func(2),运行结果为2 100。

\subsection{关键字参数}
为函数传递参数值的方法有两种:位置参数和关键字参数。我们之前调用函数的时候都使用的是位置参数。下面我们看如何使用关键字参数:

\begin{lstlisting}
def named_args(name, greeting):
    print(greeting + " " + name)

named_args(name='jim', greeting='Hello')
named_args(greeting='Hello', name='jim')
named_args('jim', greeting='hello')
\end{lstlisting}
上述代码运行结果都是“hello jim"。

关键字参数使用“name=value”的名称、值对传递数据,正如上面代码演示的那样,使用关键字参数的时候,参数的顺序是可以调换的,而且位置参数和关键字参数可以混合使用(只能先使用位置参数,后使用关键字参数)。

\subsection{返回多个值}
我们可以通过在return语句中使用逗号,将多个值返回,这种返回值的类型是元组。例如:

\begin{lstlisting}
def bigger(a, b):
    if a > b:
        return a, b
    else:
        return b, a

s = bigger(12, 100)
print(s)
print(type(s))
\end{lstlisting}

运行结果为:

\begin{lstlisting}
(100, 12)
<class 'tuple'>
\end{lstlisting}
\subsection{函数文档字符串} % (fold)
\label{sub:函数文档字符串}
Python 有一个甚是优美的功能称作文档字符串(Documentation Strings),在称呼它时通常会使用另一个短一些的名字docstrings。DocStrings 是一款你应当使用的重要工具,它能够帮助你更好地记录程序并让其更加易于理解。令人惊叹的是,当程序实际运行时,我们甚至可以通过一个函数来获取文档!

\begin{lstlisting}
def print_max(x, y):
    '''Prints the maximum of two numbers.

    The two values must be integers.'''
    # 如果可能,将其转换至整数类型
    x = int(x)
    y = int(y)

    if x > y:
        print(x, 'is maximum')
    else:
        print(y, 'is maximum')

print_max(3, 5)
print(print_max.__doc__)
输出:

$ python function_docstring.py
5 is maximum
Prints the maximum of two numbers.

    The two values must be integers.
\end{lstlisting}

该文档字符串所约定的是一串多行字符串,其中第一行以某一大写字母开始,以句号结束。第二行为空行,后跟的第三行开始是任何详细的解释说明。强烈建议你的文档字符串中都遵循这一约定。

我们可以通过使用函数的 \_\_doc\_\_(注意其中的双下划线)属性(属于函数的名称)来获取函数 print\_max 的文档字符串属性。
% subsection 函数文档字符串 (end)
