\section{创建模块}
创建一个名为mymodule.py的新文件,并写入下面的代码:

\begin{lstlisting}
foo = 100

def hello():
    print("i am from mymodule.py")
\end{lstlisting}
在这个文件中,我们定义了一个全部变量foo和一个名为hello()的方法。现在我们可以使用import关键词来引入这个模块,并使用mymodule.py中的变量和函数:

\begin{lstlisting}
import mymodule

print(mymodule.foo)
mymodule.hello()
\end{lstlisting}

上述代码的运行结果如下:

\begin{lstlisting}
100
i am from mymodule.py
\end{lstlisting}

如之前代码所示,调用模块的变量和函数时,需要指定模块的名称。

\section{使用模块中的指定内容}
当我们使用import声明导入模块时,模块中的所有内容都被导入到当前文件中。如果我们只需要模块中的个别内容时该如何操作呢?使用from关键词,就可以达到这样的目的,比如:

\begin{lstlisting}
from mymodule import foo
print(foo)
\end{lstlisting}
上述代码的运行结果为100。
\begin{myremark}{}
当使用from improt语句导入特定内容后,访问这些内容就不需要再使用模块名了。
\end{myremark}

\section{dir函数} % (fold)
\label{sec:dir函数}
内置的 dir() 函数能够返回由对象所定义的名称列表。 如果这一对象是一个模块,则该列表会包括函数内所定义的函数、类与变量。
该函数接受参数。 如果参数是模块名称,函数将返回这一指定模块的名称列表。 如果没有提供参数,函数将返回当前模块的名称列表。

\begin{lstlisting}
>>> import sys

# 给出 sys 模块中的属性名称
>>> dir(sys)
['__displayhook__', '__doc__',
'argv', 'builtin_module_names',
'version', 'version_info']
# only few entries shown here

# 给出当前模块的属性名称
>>> dir()
['__builtins__', '__doc__',
'__name__', '__package__']

# 创建一个新的变量 'a'
>>> a = 5

>>> dir()
['__builtins__', '__doc__', '__name__', '__package__', 'a']
\end{lstlisting}

\section{包} % (fold)
\label{sec:包}
包是指一个包含模块与一个特殊的 \_\_init\_\_.py 文件的文件夹,后者向 Python 表明这一文件夹是特别的,因为其包含了 Python 模块。

假设你想创建一个名为“world”的包,其中还包含着 ”asia“、”africa“等其它子包,同时这些子包都包含了诸如”india“、”madagascar“等模块。下面是你会构建出的文件夹的结构:

\begin{lstlisting}
- <some folder present in the sys.path>/
    - world/
        - __init__.py
        - asia/
            - __init__.py
            - india/
                - __init__.py
                - foo.py
        - africa/
            - __init__.py
            - madagascar/
                - __init__.py
                - bar.py
\end{lstlisting}

包是一种能够方便地分层组织模块的方式。
% section 包 (end)
