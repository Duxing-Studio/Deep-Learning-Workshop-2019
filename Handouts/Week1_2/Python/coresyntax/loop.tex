\section{循环}

Python只有两种循环:for循环和while循环。

\subsection{for循环}
for循环语法:
\begin{lstlisting}
for i in iterable_object:
   # do something
\end{lstlisting}
\begin{myremark}{}
所有在for循环或者while循环中的声明,必须使用相同的缩进值。否则会出现语法错误。
\end{myremark}
我们看下面这段代码:

\begin{lstlisting}
mylist = [1, 2, 3, 4]

for i in mylist:
    print(i)
\end{lstlisting}
在第一次循环时,值1被传递给i,第二次循环时,值2被传递给i。循环一直到列表变量mylist没有更多元素时停止。运行结果为:

\begin{lstlisting}
1
2
3
4
\end{lstlisting}

\subsection{范围循环}
range()函数能够指定循环的起始值和结束值,从而让循环体在指定的范围内循环。例如:

\begin{lstlisting}
for i in range(10):
    print(i) 					# 0-9
for i in range(1,10):
    print(i) 					# 1-9
for i in range(1,10,2):
    print(i) 					# 1,3,5,7
\end{lstlisting}
range()函数只有1个参数时,表示从0开始循环;两个参数时,第一个参数是起始值,第二个参数是结束值;三个参数时,第三个参数表示循环步长。

\subsection{while循环}
语法:

\begin{lstlisting}
while condition:
    # do something
\end{lstlisting}

While循环会一直执行循环体内部的声明,直到条件变成false。每次循环都会检查判断条件,如果为真,就继续循环。例如:

\begin{lstlisting}
count = 0

while count < 10:
    print(count)
    count += 1
\end{lstlisting}

这段代码将会打印出0-9,直到count等于10。

\subsection{中断循环}
使用break语句,可以中断循环,例如:

\begin{lstlisting}
count = 0

while count < 10:
    count += 1
    if count == 5:
        break
    print("inside loop", count)

print("out of while loop")
\end{lstlisting}

运行结果为:

\begin{lstlisting}
inside loop 1
inside loop 2
inside loop 3
inside loop 4
out of while loop
\end{lstlisting}

\subsection{继续循环}
当循环体内部出现continue声明时,会结束本次循环,跳转到循环体开始位置,开始下一次循环。例如:

\begin{lstlisting}
count = 0

while count < 10:
    count += 1
    if count % 2 == 0:
        continue
    print(count)
\end{lstlisting}
运行结果将打印出1,3,5,7,9。
