\section{控制声明}
在程序中,常常要根据一些条件执行相应的命令。

\subsection{分支判断}
Python使用if-else进行控制声明。语法如下:

\begin{lstlisting}
if boolean-expression:
   #statements
else:
   #statements
\end{lstlisting}

\begin{myremark}{}
在每一个if程序块中,必须使用相同数量的缩进,否则会产生语法错误。这是Python和其他语言非常不同的一点。
\end{myremark}

现在我们看一个例子:

\begin{lstlisting}
i = 11
if i % 2 == 0:
    print("偶数")
else:
    print("奇数")
\end{lstlisting}

运行结果将根据i的值发生变化。

如果需要判断多个条件,我们就可以使用if-elif-else控制声明,例如:

\begin{lstlisting}
today = "monday"

if today == "monday":
   print("this is monday")
elif today == "tuesday":
   print("this is tuesday")
elif today == "wednesday":
   print("this is wednesday")
elif today == "thursday":
   print("this is thursday")
elif today == "friday":
   print("this is friday")
elif today == "saturday":
   print("this is saturday")
elif today == "sunday":
   print("this is sunday")
else:
   print("something else")
\end{lstlisting}

我们可以根据实际需求,添加对应的多个elif条件。

\subsection{分支嵌套}
我们可以在if声明语句块中嵌套使用if声明。例如:

\begin{lstlisting}
today = "holiday"
bank_balance = 25000
if today == "holiday":
    if bank_balance > 20000:
        print("Go for shopping")
    else:
        print("Watch TV")
else:
    print("normal working day")
\end{lstlisting}
