\bibliographystyle{gbt7714-2005}

\chapter{Python简介}

\section{Python发展历史}

Python的创始人为Guido van Rossum。1989年圣诞节期间,在阿姆斯特丹,Guido为了打发圣诞节的无趣,决心开发一个新的脚本解释程序,做为ABC 语言的一种继承。之所以选中Python(大蟒蛇的意思)作为程序的名字,是因为他是一个叫Monty Python的喜剧团体的爱好者。ABC是由Guido参加设计的一种教学语言。就Guido本人看来,ABC 这种语言非常优美和强大,是专门为非专业程序员设计的。但是ABC语言并没有成功,究其原因,Guido 认为是非开
放造成的。Guido 决心在 Python 中避免这一错误。同时,他还想实现在ABC 中闪现过但未曾实现的东西。

截至目前,Python的版本为3.5.1,2015年9月13日发布。\cite{PSF,2015a}

\section{Python特点}

\begin{description}
	\item [简单] Python是一种代表简单主义思想的语言。阅读一个良好的Python程序就感觉像是在读英语一样。它使你能够专注于解决问题而不是去搞明白语言本身。
	\item [易学] Python很容易上手,一方面是由于Python有完善的说明文档,另一方面网络中有大量的教程,学习资源可谓丰富。本书的写作就参考了诸多网络教程。\cite{pythonguru,2015}
	\item [开源] 开源意味着人们可以自由地发布这个软件的拷贝、阅读它的源代码、对它做改动、把它的一部分用于新的自由软件中。Python有非常活跃的开源社区,来自世界各地的程序员不断完善着Python,如今Python拥有功能强大且门类齐全的扩展库。它可以帮助处理各种工作,包括正则表达式、文档生成、单元测试、线程、数据库、网页浏览器、CGI、FTP、电子邮件、XML、XML-RPC、HTML、WAV文件、密码系统、GUI(图形用户界面)、Tk和其他与系统有关的操作。Python语言及其众多的扩展库所构成的开发环境十分适合工程技术、科研人员处理实验数据、制作图表,甚至开发科学计算应用程序。
	\item [解释性] 在计算机内部,Python解释器把源代码转换成称为字节码的中间形式,然后再把它翻译成计算机使用的机器语言并运行。这使得使用Python更加简单。也使得Python程序更加易于移植。
	\item [可移植] Python已经被移植在许多平台上(经过改动使它能够工作在不同平台上)。这些平台包括Linux、Windows、FreeBSD、Macintosh、Solaris、OS/2、Amiga、AROS、AS/400、BeOS、OS/390、z/OS、Palm OS、QNX、VMS、Psion、Acom RISC OS、VxWorks、PlayStation、Sharp Zaurus、Windows CE、PocketPC、Symbian以及Google基于linux开发的android平台。
	\item [面向对象] Python既支持面向过程的编程也支持面向对象的编程。在“面向过程”的语言中,程序是由过程或仅仅是可重用代码的函数构建起来的。在“面向对象”的语言中,程序是由数据和功能组合而成的对象构建起来的。
	\item [可扩展] 如果需要一段关键代码运行得更快或者希望某些算法不公开,可以部分程序用C或C++编写,然后在Python程序中使用它们。
	\item [可嵌入] 可以把Python嵌入C/C++程序,从而向程序用户提供脚本功能。
\end{description}

\section{使用Python的知名项目}

以下是使用Python作为主力开发语言的知名项目,其中有一些是用python进行开发,有一些在部分业务或功能上使用到了python,还有的是支持python作为扩展脚本语言。

\begin{description}
	\item [Reddit] 社交分享网站,最早用Lisp开发,在2005年转为python。
	\item [Dropbox] 文件分享服务。
	\item [豆瓣网] 图书、唱片、电影等文化产品的资料数据库网站。
	\item [Django] 鼓励快速开发的Web应用框架。
	\item [EVE] 网络游戏EVE大量使用Python进行开发。
	\item [Fabric] 用于管理成百上千台Linux主机的程序库。
	\item [Blender] 以C与Python开发的开源3D绘图软件。
	\item [BitTorrent] bt下载软件客户端。
	\item [Ubuntu Software Center] Ubuntu 9.10版本后自带的图形化包管理器。
	\item [YUM] 用于RPM兼容的Linux系统上的包管理器。
	\item [Civilization IV] 游戏《文明4》。
	\item [Battlefield 2] 游戏《战地2》。
	\item [Google] 谷歌在很多项目中用python作为网络应用的后端,如Google Groups、Gmail、Google Maps。
	\item [NASA] 美国宇航局,从1994年起把python作为主要开发语言。
	\item [Industrial Light \& Magic] 工业光魔,乔治·卢卡斯创立的电影特效公司。
	\item [Yahoo Groups] 雅虎推出的群组交流平台。
	\item [YouTube] 视频分享网站,在某些功能上使用到python。
	\item [Cinema 4D] 一套整合3D模型、动画与绘图的高级三维绘图软件,以其高速的运算和强大的渲染插件著称。
	\item [Autodesk Maya] 3D建模软件,支持python作为脚本语言。
	\item [gedit] Linux平台的文本编辑器。
	\item [GIMP] Linux平台的图像处理软件。
	\item [Minecraft: Pi Edition] 游戏《Minecraft》的树莓派版本。
	\item [MySQL Workbench] 可视化数据库管理工具。
	\item [Digg] 社交新闻分享网站。
	\item [Mozilla] 为支持和领导开源的Mozilla项目而设立的一个非营利组织。
	\item [Quora] 社交问答网站。
	\item [Path] 私密社交应用。
	\item [Pinterest] 图片社交分享网站。
	\item [SlideShare] 幻灯片存储、展示、分享的网站。
	\item [Yelp] 美国商户点评网站。
	\item [Slide] 社交游戏/应用开发公司,被谷歌收购。
\end{description}

\section{搭建Python开发环境}

Python支持多个平台,其中在Mac、类UNIX平台中已默认安装,Windows平台中的安装也非常简单,从官方网站下载安装包安装即可,注意安装时将Python所在目录添加到系统路径中即可。

虽然Python自带编辑器,但其不够方便,推荐使用轻量级的编辑器Sublime Text。使用编辑器将文件保存成.py后缀,然后通过命令行调用即可执行,也可以在Sublime Text编辑器中使用编译命令(ctrl+b)查看运行结果。

如在编辑器中键入如下内容:

\begin{lstlisting}
print ('Hello world!')
\end{lstlisting}

保存为hello.py,注意设置文件编码方式为UTF-8,然后使用ctrl+b即可在编辑器内部查看运行结果。

或者通过命令行进入到脚本所在路径,键入脚本名称(后缀名可省略)也可运行脚本。

\begin{lstlisting}
cd c:/wamp/www/python/example
hello
\end{lstlisting}



\bibliography{../bib/yangjh}
