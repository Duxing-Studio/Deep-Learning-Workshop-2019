\section{列表}

Python的列表(list)对象是最常用的序列(Sequence)。与字符串是不可变序列不同,列表是可变的。可通过对偏移量进行修改和读取。

\subsection{列表赋值}

列表可通过索引对其对应的元素进行赋值,从而改变列表的内容,如:

\begin{lstlisting}
>>> a = [2, 2, 2]
>>> a[1] = 1
>>> print(a)
[2, 1, 2]
\end{lstlisting}

通过上述代码的运行,我们可以看到列表确实是可以改变的。

\subsection{删除元素}

使用del语句可以删除列表中的元素,如:

\begin{lstlisting}
>>> a = [2, 2, 2]
>>> del a[1]
>>> print(a)
[2, 2]
\end{lstlisting}

\subsection{分片赋值}

分片赋值可以一次为多个元素赋值,并且不用考虑原列表的长度是否和新的列表长度一直,如:

\begin{lstlisting}
>>> name = list('Python')
>>> print(name)
['P', 'y', 't', 'h', 'o', 'n']
>>> name[2:] = list('data')
>>> print(name)
['P', 'y', 'd', 'a', 't', 'a']
\end{lstlisting}

上述代码中的list函数是Python内置函数,其作用是将字符串转换为列表。
运行结果显示,通过分片赋值,将原有列表['P', 'y', 't', 'h', 'o', 'n'],修改为['P', 'y', 'd', 'a', 't', 'a']。

分片赋值还可以用来插入元素,如:

\begin{lstlisting}
>>> name = list('Python')
>>> name[1:1] = list('--')
>>> print(name)
['P', '-', '-', 'y', 't', 'h', 'o', 'n']
\end{lstlisting}
结果显示将原有列表['P', 'y', 't', 'h', 'o', 'n'],修改为['P', '-', '-', 'y', 'd', 'a', 't', 'a']。

\subsection{列表对象常用内置方法}

\subsubsection{追加列表元素}
。列表提供了在列表尾部追加新对象的方法append。

\begin{lstlisting}
>>> code = [1, 2, 3]
>>> code.append(4)
>>> print(code)
[1, 2, 3, 4]
\end{lstlisting}

\subsubsection{计数}

count方法统计指定元素在列表中出现的次数,如:

\begin{lstlisting}
>>> code = ['to', 'be', 'or', 'not', 'to', 'be']
>>> print(code.count('to'))
2
\end{lstlisting}

以上代码将统计出列表中‘to’元素出现的次数,结果为2。

\subsubsection{合并列表}

extend方法在列表的末尾一次性追加另一个序列中的多个值,如:

\begin{lstlisting}
a = [1, 2, 3]
b = [4, 5, 6]
a.extend(b)
print(a)
\end{lstlisting}

以上代码将把b列表追加到a列表中,打印出的a列表的值为[1, 2, 3, 4, 5, 6]。和序列加运算不同,extend方法将改变原有列表的内容,而加运算却不会。例如:

\begin{lstlisting}
b = [4, 5, 6]
b + [7, 8, 9]
print(b)
\end{lstlisting}

上述代码结果显示为[4, 5, 6],b列表的内容并没有改变。

\subsubsection{元素索引}

index方法用于从列表中找出指定值第一次匹配的索引值。例如:

\begin{lstlisting}
a = [1, 2, 3, 3, 2, 1]
print(a.index(1))
\end{lstlisting}

以上代码运行结果为0,即第一个1出现的索引为0。

\subsubsection{插入元素}

insert方法用于将对象插入到列表中,例如:

\begin{lstlisting}
a = [1, 2, 3]
a.insert(2, 2.5)
print(a)
\end{lstlisting}

运行结果为[1, 2, 2.5, 3],insert方法的两个参数值很好理解,第一个参数为在哪个元素后插入,表示位置,第二个参数为插入的内容。

\subsubsection{pop}

pop方法会移除列表中的一个元素,默认为最后一个,和append方法刚好相反,并且返回该元素的值。例如:

\begin{lstlisting}
a = [1, 2, 3]
print(a.pop())
print(a)
\end{lstlisting}

运行结果为3和[1, 2],当然,pop方法也可以指定移除某个索引的元素。

\subsubsection{remove}

remove方法用于移除列表中某个值的第一个匹配项:

\begin{lstlisting}
code = ['to', 'be', 'or', 'not', 'to', 'be']
print(code.remove('or'))
print(code)
\end{lstlisting}

运行结果为None和['to', 'be', 'not', 'to', 'be']。这说明remove方法并不返回匹配到的内容。

\subsubsection{reverse}

reverse方法将倒序排列列表元素:

\begin{lstlisting}
a = [1, 2, 3]
a.reverse()
print(a)
\end{lstlisting}

运行结果为[3, 2, 1]。

\subsubsection{sort}
sort方法用于对列表排序,如:

\begin{lstlisting}
a = [1, 3, 4, 8, 6, 2]
a.sort()
print(a)
\end{lstlisting}

运行结果为:[1, 2, 3, 4, 6, 8]。需要注意的是,sort方法没有返回值,并且改变列表的内容,如果你不但要排序,而且还要保持原有数据的内容,解决的方法之一是将原有内容赋值到另外一个变量中保存。
