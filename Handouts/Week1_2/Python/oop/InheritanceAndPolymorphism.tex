\section{继承和多态}

继承(inheritance)允许开发人员先创建一个通用的类,然后扩展为特定类。使用继承机制,我们可以获得类的数据字段和方法,还可以增加自定义的字段和方法,因此,继承提供了一种组织代码、重用代码的方式。

在面向对象的术语中,当类X继承自类Y时,Y被叫做超类(super class)或基类(base class),而X被成为子类(subclass)或者衍生类(derived class)。
\begin{myremark}{}
私有数据字段和私有方法只在类的内部使用。子类只能继承父类的非私有数据字段和非私有方法。
\end{myremark}

继承的语法如下:

\begin{lstlisting}
class SubClass(SuperClass):
  # data fields
  # instance methods
\end{lstlisting}

让我们看个例子:

\begin{lstlisting}
class Vehicle:

    def __init__(self, name, color):
        self.__name = name      # __name是私有数据字段
        self.__color = color

    def getColor(self):
        return self.__color

    def setColor(self, color):
        self.__color = color

    def getName(self):
        return self.__name


class Car(Vehicle):

    def __init__(self, name, color, model):
        # 调用父类的构造方法
        super().__init__(name, color)
        self.__model = model

    def getDescription(self):
        return self.getName() + self.__model + " in " + self.getColor() + " color"

c = Car("Ford Mustang", "red", "GT350")
print(c.getDescription())
print(c.getName())
\end{lstlisting}

上述代码中,我们创建了基类Vehicle和子类Car。在子类Car中,我们没有定义getName()方法,但我们仍然可以访问getName(),这是因为类Car继承自Vehicle类。在这段代码中,super()方法用来调用基类的方法。上述代码的运行结果如下:

\begin{lstlisting}
Ford MustangGT350 in red color
Ford Mustang
\end{lstlisting}

\subsection{多重继承}
不像Java和C\#语言,Python允许多重继承。即一次继承多个基类,比如:

\begin{lstlisting}
class Subclass(SuperClass1, SuperClass2, ...):
   # initializer
   # methods
\end{lstlisting}

看如下实例:

\begin{lstlisting}
class MySuperClass1():

    def method_super1(self):
        print("method_super1 method called")


class MySuperClass2():

    def method_super2(self):
        print("method_super2 method called")


class ChildClass(MySuperClass1, MySuperClass2):

    def child_method(self):
        print("child method")

c = ChildClass()
c.method_super1()
c.method_super2()

\end{lstlisting}

输出结果为:

\begin{lstlisting}
method_super1 method called
method_super2 method called
\end{lstlisting}

因为子类ChildClass继承自MySuperClass1 , MySuperClass2,因此,ChildClass对象c可以访问method\_super1()方法和 method\_super2()方法。

\subsection{重写方法}
为重写基类的某个方法,子类需要定义一个同名的方法(即拥有相同名称和相同数量的参数)。例如:

\begin{lstlisting}
class A():

    def __init__(self):
        self.__x = 1

    def m1(self):
        print("m1 from A")


class B(A):

    def __init__(self):
        self.__y = 1

    def m1(self):
        print("m1 from B")

c = B()
c.m1()
\end{lstlisting}
在这段代码中,我们重写了基类的m1()方法。输出结果为:

\begin{lstlisting}
m1 from B
\end{lstlisting}

\subsection{判断对象是否属于某类}

isinstance() 方法用来检测指定对象是否是某个类的实例。例如:

\begin{lstlisting}
>>> isinstance(1, int)
True

>>> isinstance(1.2, int)
False

>>> isinstance([1,2,3,4], list)
True
\end{lstlisting}
