%% LyX 2.3.2 created this file.  For more info, see http://www.lyx.org/.
%% Do not edit unless you really know what you are doing.
\documentclass[english]{article}

\pdfoutput=1
\usepackage[T1]{fontenc}
\usepackage[latin9]{inputenc}
\usepackage{geometry}
\geometry{verbose,tmargin=1in,bmargin=1in,lmargin=1in,rmargin=1in}
%\usepackage{babel}

%\RequirePackage{natbib}
\usepackage{verbatim}
\usepackage{listings}
\usepackage{float}
\usepackage{bm}
\usepackage{bbm}
\usepackage{amsmath}
\usepackage{subcaption}
\usepackage{amssymb}
\usepackage{graphicx}
\usepackage{hyperref}
\usepackage{hhline}
\hypersetup{
 colorlinks,linkcolor=red,anchorcolor=blue,citecolor=blue}
\usepackage{breakurl}
%\usepackage{cite}
\usepackage{amsthm}
\usepackage{dsfont}
\usepackage{array}
%\usepackage{mathrsfs}
\usepackage{color}




\definecolor{codegreen}{rgb}{0,0.6,0}
\definecolor{codegray}{rgb}{0.5,0.5,0.5}
\definecolor{codepurple}{rgb}{0.58,0,0.82}
\definecolor{backcolour}{rgb}{0.95,0.95,0.92}

\makeatletter

%%%%%%%%%%%%%%%%%%%%%%%%%%%%%% LyX specific LaTeX commands.
%% Because html converters don't know tabularnewline
\providecommand{\tabularnewline}{\\}
\floatstyle{ruled}
\newfloat{algorithm}{tbp}{loa}
\providecommand{\algorithmname}{Algorithm}
\floatname{algorithm}{\protect\algorithmname}

\newcommand{\mypara}[1]{\paragraph{#1.}}

%%%% for colors

\definecolor{yxc}{RGB}{255,0,0}
\definecolor{yjc}{RGB}{125,0,0}
\definecolor{cm}{RGB}{0,0,200}
\definecolor{kzw}{RGB}{0,150,0}
\newcommand{\cm}[1]{\textcolor{cm}{[CM: #1]}}
\newcommand{\yxc}[1]{\textcolor{yxc}{[YXC: #1]}}
\newcommand{\yjc}[1]{\textcolor{yjc}{[YJC: #1]}}


\allowdisplaybreaks

\usepackage{enumitem}
\setlist[itemize]{leftmargin=1em}
\setlist[enumerate]{leftmargin=1em}

\usepackage{algorithm}% http://ctan.org/pkg/algorithms
\usepackage{algorithmic}% http://ctan.org/pkg/algorithmicx
\usepackage{arydshln}

\newcommand{\DLFigs}{figure}
\newcommand{\citep}{\cite}
\newcommand{\citealp}{\cite}
\makeatother


%\startlocaldefs
%\numberwithin{equation}{section}
%\theoremstyle{plain}
%\newtheorem{thm}{Theorem}[section]
%\newtheorem{definition}{Definition}[section]
%\newtheorem{cor}{Corollary}[section]
%\newtheorem{rem}{Remark}[section]
%\newtheorem{lem}{Lemma}[section]
%\newtheorem{ass}{Assumption}[section]
%\endlocaldefs


\makeatletter
\newcommand*{\rom}[1]{\expandafter\@slowromancap\romannumeral #1@}
\makeatother

\newcommand{\eqi}[1]{\mathrel{\overset{\makebox[0pt]{\normalfont\scriptsize\sffamily (#1)}}{=}}}
\newcommand{\lei}[1]{\mathrel{\overset{\makebox[0pt]{\normalfont\scriptsize\sffamily (#1)}}{\le}}}
\newcommand{\gei}[1]{\mathrel{\overset{\makebox[0pt]{\normalfont\scriptsize\sffamily (#1)}}{\ge}}}
\newcommand{\lessi}[1]{\mathrel{\overset{\makebox[0pt]{\normalfont\scriptsize\sffamily (#1)}}{\lesssim}}}
\newcommand{\gtri}[1]{\mathrel{\overset{\makebox[0pt]{\normalfont\scriptsize\sffamily (#1)}}{\gtrsim}}}


%%%% special operators

\newcommand{\diag}{\mathrm{diag}}
\newcommand{\diagg}{\mathrm{diagg}}
\newcommand{\rank}{\mathrm{rank}}
\newcommand{\spann}{\mathrm{span}}
\newcommand{\supp}{\mathrm{supp}}
\newcommand{\sgnn}{\mathrm{sign}}
\newcommand{\sgn}{\mathrm{sgn}}
\newcommand{\sign}{\mathrm{sign}}
\newcommand{\conv}{\mathrm{conv}}
\newcommand{\cov}{\mathrm{Cov}}
\newcommand{\var}{\mathrm{Var}}
\newcommand{\trace}{\mathrm{trace}}
\newcommand{\Tr}{\mathrm{Tr}}
\newcommand{\vect}{\mathrm{Vec}}
\newcommand{\RE}{\mathrm{RE}}
\newcommand{\GMM}{\mathrm{GMM}}
\newcommand{\minimize}{\mbox{minimize}}
\newcommand{\st}{\mbox{subject to}}
\newcommand{\argmin}{\mbox{argmin}}
\newcommand{\argmax}{\mbox{argmax}}
\newcommand{\svd}{\mathrm{svd}}
\newcommand{\eigen}{\mathrm{eigen}}
\newcommand{\sam}{\mathrm{sam}}
\newcommand{\KL}{\normalfont \mathrm{\texttt{KL}}}
\newcommand{\Bern}{\mathrm{Bern}}
\newcommand{\row}{\mathrm{row}}
\newcommand{\col}{\mathrm{col}}
\newcommand{\where}{\text{where}}
\newcommand{\mle}{\mathrm{MLE}}
\newcommand{\iid}{\mathrm{i.i.d.}}
\newcommand{\btanh}{\mathrm{\textbf{tanh}}}


\newcommand{\R}{\mathbb{R}}
\newcommand{\bbS}{\mathbb{S}}
\newcommand{\E}{\mathbb{E}}
\newcommand{\Z}{\mathbb{Z}}
\newcommand{\bfR}{\mathbf{R}}
\newcommand{\calA}{\mathcal{A}}
\newcommand{\veps}{\varepsilon}
\renewcommand{\O}{\mathcal{O}}
\renewcommand{\P}{\mathbb{P}}
\DeclareMathOperator{\Var}{{\rm Var}}
\DeclareMathOperator{\Cor}{\rm Corr}
\DeclareMathOperator{\Cov}{\rm Cov}
\DeclareMathOperator{\ind}{\mathds{1}}  % Indicator
\newcommand{\smallfrac}[2]{{\textstyle \frac{#1}{#2}}}

\newcommand*{\zero}{{\bm 0}}
\newcommand*{\one}{{\bm 1}}
\newcommand{\bbone}{\mathbbm{1}}
\newcommand{\bbzero}{\mathbbm{0}}




%%%% Norms & dot product, & brackets

\newcommand{\norm}[1]{\left\|{#1}\right\|}
\newcommand{\bignorm}[1]{\bigg|\bigg|#1\bigg|\bigg|}
\newcommand{\opnorm}[2]{| \! | \! | #1 | \! | \! |_{{#2}}}
\newcommand{\dotp}[2]{\langle{#1},{#2}\rangle}
\newcommand{\inner}[2]{\left\langle #1,#2 \right\rangle}
\newcommand{\rbr}[1]{\left(#1\right)}
\newcommand{\sbr}[1]{\left[#1\right]}
\newcommand{\cbr}[1]{\left\{#1\right\}}
\newcommand{\nbr}[1]{\left\|#1\right\|}
\newcommand{\abr}[1]{\left|#1\right|}



%%%% boldface and caligraphicals

\newcommand{\bff}{\mathrm{\bf f}}
\newcommand{\ba}{\mathrm{\bf a}}
\newcommand{\be}{\mathrm{\bf e}}
\newcommand{\bh}{\mathrm{\bf h}}
\newcommand{\br}{\mathrm{\bf r}}
\newcommand{\bs}{\mathrm{\bf s}}
\newcommand{\bt}{\mathrm{\bf t}}
\newcommand{\bu}{\mathrm{\bf u}}
\newcommand{\bv}{\mathrm{\bf v}}
\newcommand{\bw}{\mathrm{\bf w}}
\newcommand{\bx}{\mathrm{\bf x}}
\newcommand{\by}{\mathrm{\bf y}}
\newcommand{\bz}{\mathrm{\bf z}}
\newcommand{\bA}{\mathrm{\bf A}}
\newcommand{\bB}{\mathrm{\bf B}}
\newcommand{\bZ}{\mathrm{\bf Z}}
\newcommand{\bC}{\mathrm{\bf C}}
\newcommand{\bD}{\mathrm{\bf D}}
\newcommand{\bE}{\mathrm{\bf E}}
\newcommand{\bF}{\mathrm{\bf F}}
\newcommand{\bK}{\mathrm{\bf K}}
\newcommand{\bT}{\mathrm{\bf T}}
\newcommand{\bW}{\mathrm{\bf W}}
\newcommand{\bG}{\mathrm{\bf G}}
\newcommand{\bM}{\mathrm{\bf M}}
\newcommand{\bH}{\mathrm{\bf H}}
\newcommand{\bI}{\mathrm{\bf I}}
\newcommand{\bg}{\mathrm{\bf g}}
\newcommand{\bP}{\mathrm{\bf P}}
\newcommand{\bV}{\mathrm{\bf V}}
\newcommand{\bQ}{\mathrm{\bf Q}}
\newcommand{\bR}{\mathrm{\bf R}}
\newcommand{\bS}{\mathrm{\bf S}}
\newcommand{\bU}{\mathrm{\bf U}}
\newcommand{\bX}{\mathrm{\bf X}}
\newcommand{\bY}{\mathrm{\bf Y}}
\newcommand{\bL}{\mathrm{\bf L}}



\newcommand{\xx}{\text{\boldmath $x$}}
\newcommand{\XX}{\text{\boldmath $X$}}
\newcommand{\yy}{\text{\boldmath $y$}}
\newcommand{\zz}{\text{\boldmath $z$}}
\newcommand{\YY}{\text{\boldmath $Y$}}
\newcommand{\ZZ}{\text{\boldmath $Z$}}
\newcommand{\WW}{\text{\boldmath $W$}}
\newcommand{\VV}{\text{\boldmath $V$}}
\newcommand{\UU}{\text{\boldmath $U$}}
\newcommand{\vv}{\text{\boldmath $v$}}
\newcommand{\ww}{\text{\boldmath $w$}}
\newcommand{\kk}{\text{\boldmath $k$}}
\newcommand{\RR}{\text{\boldmath $R$}}
\newcommand{\uu}{\text{\boldmath $u$}}
\newcommand{\uI}{\text{\boldmath $u_I$}}
\newcommand{\II}{\text{\boldmath $I$}}
\newcommand{\hh}{\text{\boldmath $h$}}
\renewcommand{\aa}{\text{\boldmath $a$}}
\newcommand{\bb}{\text{\boldmath $b$}}
\newcommand{\cc}{\text{\boldmath $c$}}
\newcommand{\pp}{\text{\boldmath $p$}}
\newcommand{\bgg}{\text{\boldmath $g$}}
\newcommand{\oo}{\text{\boldmath $o$}}
\newcommand{\ii}{\text{\boldmath $i$}}
\newcommand{\ff}{\text{\boldmath $f$}}
\newcommand{\FF}{\text{\boldmath $F$}}





\newcommand{\balpha}{\text{\boldmath $\alpha$}}
\newcommand{\bpi}{\text{\boldmath $\pi$}}
\newcommand{\bvarphi}{\text{\boldmath $\varphi$}}
\newcommand{\bbeta}{\text{\boldmath $\beta$}}
\newcommand{\bdelta}{\text{\boldmath $\delta$}}
\newcommand{\btheta}{\text{\boldmath $\theta$}}
\newcommand{\bomega}{\text{\boldmath $\omega$}}
\newcommand{\bsigma}{\bm{\sigma}}
\newcommand{\bDelta}{\text{\boldmath $\Delta$}}
\newcommand{\bone}{\mathrm{\bf 1}}
\newcommand{\bzero}{\mathrm{\bf 0}}
\newcommand{\bet}{\text{\boldmath $\eta$}}
\newcommand{\bxi}{\text{\boldmath $\xi$}}
\newcommand{\bveps}{\text{\boldmath $\varepsilon$}}
\newcommand{\bmu}{\text{\boldmath $\mu$}}
\newcommand{\bLambda}{\text{\boldmath $\Lambda$}}
\newcommand{\blambda}{\text{\boldmath $\lambda$}}
\newcommand{\bgamma}{\text{\boldmath $\gamma$}}
\newcommand{\bGamma}{\text{\boldmath $\Gamma$}}
\newcommand{\bTheta}{\text{\boldmath $\Theta$}}
\newcommand{\bSigma}{\text{\boldmath $\Sigma$}}
\newcommand{\bOmega}{\text{\boldmath $\Omega$}}
\newcommand{\bPhi}{\text{\boldmath $\Phi$}}
\newcommand{\bXi}{\text{\boldmath $\Xi$}}
\newcommand{\bPsi}{\text{\boldmath $\Psi$}}
\newcommand{\bPi}{\text{\boldmath $\Pi$}}
\newcommand{\bepsilon}{\text{\boldmath $\epsilon$}}

\newcommand{\cI}{\mathcal{I}}
\newcommand{\cB}{\mathcal{B}}
\newcommand{\cN}{\mathcal{N}}
\newcommand{\cS}{\mathcal{S}}
\newcommand{\cL}{\mathcal{L}}
\newcommand{\cV}{\mathcal{V}}
\newcommand{\cU}{\mathcal{U}}
\newcommand{\cO}{\mathcal{O}}
\newcommand{\cA}{\mathcal{A}}
\newcommand{\cP}{\mathcal{P}}
\newcommand{\cX}{\mathcal{X}}
\newcommand{\cM}{\mathcal{M}}
\newcommand{\cE}{\mathcal{E}}
\newcommand{\cT}{\mathcal{T}}
\newcommand{\cF}{\mathcal{F}}
\newcommand{\cW}{\mathcal{W}}
\newcommand{\cH}{\mathcal{H}}
\newcommand{\wtilde}{\widetilde}


%%%% hatted and tilded letters

\newcommand{\hE}{\widehat \bE}
\newcommand{\hF}{\widehat \bF}
\newcommand{\hf}{\widehat \bff}
\newcommand{\hR}{\widehat \bR}
\newcommand{\hG}{\widehat \bG}
\newcommand{\hu}{\widehat \bu}
\newcommand{\hvar}{\widehat \var}
\newcommand{\hcov}{\widehat \cov}
\newcommand{\hbveps}{\widehat\bveps}
\newcommand{\heps}{\widehat\beps}
\newcommand{\hSig}{\widehat\Sig}
\newcommand{\tSig}{\widetilde\Sig}
\newcommand{\hsig}{\widehat\sigma}
\newcommand{\hlam}{\widehat\lambda}
\newcommand{\hbeta}{\widehat\beta}
\newcommand{\hLam}{\widehat \bLambda}
\newcommand{\hmu}{\widehat\bmu}
\newcommand{\hxi}{\widehat\bxi}
\newcommand{\Sig}{\mathbf{\Sigma}}
\newcommand{\hw}{\widehat \bw}
\newcommand{\wt}{\widetilde}



%%%%  for editting and other commands


\newcommand{\RMK}[1]{{\color{blue}{[#1]}}}
\newcommand{\TODO}[1]{{\color{red}{[#1]}}}
\newcommand{\mcomment}[1]{\marginpar{\tiny{#1}}}
\newcommand{\fcomment}[1]{\footnote{\tiny{#1}}}
\newcommand{\overbar}[1]{\mkern 1.5mu\overline{\mkern-1.5mu#1\mkern-1.5mu}\mkern 1.5mu}
\newcommand{\ud}{{\,\mathrm{d}}}








%%%%%
% Default fixed font does not support bold face
\DeclareFixedFont{\ttb}{T1}{txtt}{bx}{n}{10} % for bold
\DeclareFixedFont{\ttm}{T1}{txtt}{m}{n}{10}  % for normal

% Custom colors
\usepackage{color}
\definecolor{deepblue}{rgb}{0,0,0.5}
\definecolor{deepred}{rgb}{0.6,0,0}
\definecolor{deepgreen}{rgb}{0,0.5,0}




\begin{document}
\theoremstyle{plain} 
\newtheorem{lem}{\textbf{Lemma}} 
\newtheorem{prop}{\textbf{Proposition}}
\newtheorem{thm}{\textbf{Theorem}}\setcounter{thm}{0}
\newtheorem{corollary}{\textbf{Corollary}} 
\newtheorem{example}{\textbf{Example}}
\newtheorem{definition}{\textbf{Definition}} 
\newtheorem{fact}{\textbf{Fact}}
\newtheorem{claim}{\textbf{Claim}}

\theoremstyle{definition}

\theoremstyle{remark}\newtheorem{remark}{\textbf{Remark}}\newtheorem{conjecture}{Conjecture}\newtheorem{condition}{\textbf{Condition}}\newtheorem{assumption}{\textbf{Assumption}}

\title{A Selective Overview of Deep Learning\footnotetext{Author
names are sorted alphabetically.}}

\author{Jianqing Fan\thanks{Department of Operations Research and Financial Engineering, Princeton
University, Princeton, NJ 08544, USA; Email: \texttt{\{jqfan, congm,
yiqiaoz\}@princeton.edu}.} \and Cong Ma\footnotemark[3] \and Yiqiao Zhong\footnotemark[1] }


\maketitle
\input{abstract.tex}
\medskip
\noindent\textbf{Keywords:} neural networks, over-parametrization, stochastic gradient descent, approximation theory, generalization error.

\setcounter{tocdepth}{2}
\tableofcontents{}


%\begin{keyword}
%\kwd{neural networks}
%\kwd{over-parametrization}
%\kwd{stochastic gradient descent}
%\kwd{approximation theory}
%\kwd{generalization error}
%\end{keyword}

%\setcounter{tocdepth}{2}
%\tableofcontents
%\newpage
\section{Introduction}\label{sec:intro}
%% I added some background. If you consider it too technical, we can certainly remove it.
Modern machine learning and statistics deal with the problem of \emph{learning from data}: given a training dataset $\{(y_i,\xx_i)\}_{1\leq i \leq n}$ where $\xx_i \in \mathbb{R}^{d}$ is the input and $y_i \in \mathbb{R}$ is the output\footnote{When the label $y$ is given, this problem is often known as \emph{supervised learning}. We mainly focus on this paradigm throughout this paper and remark sparingly on its counterpart, \emph{unsupervised learning}, where $y$ is not given.}, one seeks a function $f: \mathbb{R}^{d} \mapsto \mathbb{R}$ from a certain function class $\mathcal{F}$ that has good prediction performance on test data. This problem is of fundamental significance and finds applications in numerous scenarios. For instance, in image recognition, the input $\xx$ (reps.~the output $y$) corresponds to the raw image (reps.~its category) and the goal is to find a mapping $f(\cdot)$ that can classify future images accurately. Decades of research efforts in statistical machine learning have been devoted to developing methods to find $f(\cdot)$ efficiently with provable guarantees. Prominent examples include linear classifiers (e.g., linear$\,$/$\,$logistic regression, linear discriminant analysis), kernel methods (e.g., support vector machines), tree-based methods (e.g., decision trees, random forests), nonparametric regression (e.g., nearest neighbors, local kernel smoothing), etc. Roughly speaking, each aforementioned method corresponds to a different function class $\mathcal{F}$ from which the final classifier $f(\cdot)$ is chosen.

Deep learning~\citep{lecun2015deep}, in its simplest form, proposes the following \emph{compositional} function class:
\begin{equation}\label{model:1}
\left\{f(\xx; \btheta) = \bW_{L} \bsigma_L(\bW_{L-1}\cdots \bsigma_{2}(\bW_2 \bsigma_1(\bW_1 \xx))) \;\big| \;\btheta = \{\bW_1,\ldots, \bW_{L}\}\right\}.
\end{equation}
Here, for each $1\leq l \leq L$, $\bsigma_\ell(\cdot)$ is some nonlinear function, and $\btheta = \{\bW_1,\ldots, \bW_{L}\}$ consists of matrices with appropriate sizes. Though simple, deep learning has made significant progress towards addressing the problem of learning from data over the past decade. Specifically, it has performed close to or better than humans in various important tasks in artificial intelligence, including image recognition~\citep{he2016deep}, game playing~\citep{silver2017mastering}, and machine translation~\citep{wu2016google}. Owing to its great promise, the impact of deep learning is also growing rapidly in areas beyond artificial intelligence; examples include statistics~\citep{bauer2017deep, schmidt2017nonparametric, liang2017well, romano2018deep,gao2018robust}, applied mathematics~\citep{weinan2017deep, chen2018neural}, clinical research~\citep{de2018clinically}, etc.

%The widespread use of deep leaning has lead to a myriad of frameworks (e.g.~Tensroflow~\citep{tensorflow2015-whitepaper}, PyTorch~\citep{paszke2017automatic}) for researchers and practitioner to build and deploy the models efficiently.

%Within the realm of artificial intelligence, deep learning has performed close to or better than humans in image recognition. In games such as Go and Chess, methods trained with deep learning are much better than humans, even without any input of human knowledge. In machine translation, deep learning has made great improvement over existing methods. Apart from the field of artificial intelligence, deep learning has shown huge promises.

% from a certain function class $\mathcal{F}$


%Deep learning has been recognized as one of the most important methods in artificial intelligence, and its impact is growing in statistics~\citep{bauer2017deep, schmidt2017nonparametric, liang2017well, romano2018deep}, applied mathematics~\citep{weinan2017deep, chen2018neural}, statistical physics, clinical research~\citep{de2018clinically}, etc. Within the realm of artificial intelligence, deep learning has drawn huge research interest and achieved significant progress in various important tasks since 2012,. In image recognition, deep learning has performed close to or better than humans. In games such as Go and Chess, methods trained with deep learning are much better than humans, even without any human knowledge input. In machine translation, deep learning has made great improvement over existing methods. Apart from the field of artificial intelligence, deep learning has shown huge promises.
\begin{table}[htb]
\caption{Winning models for ILSVRC image classification challenge.}
\label{tab:intro}
\begin{center}
\begin{tabular}{|c|c|c|c|c|}
\hline
Model & Year & \# Layers & \# Params & Top-5 error \\
%\hhline{|=|=|=|=|=|}
\hline
Shallow & $<2012$ & --- & --- & $>25\%$ \\
\hline
AlexNet & $2012$ & $8$ & $61$M & $16.4\%$ \\
\hline
VGG19 & $2014$ & $19$ & $144$M & $7.3\%$ \\
\hline
GoogleNet & $2014$ & $22$ & $7$M & $6.7\%$ \\
\hline
ResNet-$152$ & $2015$ & $152$ & $60$M & $3.6\%$ \\
\hline
\end{tabular}
\end{center}
\end{table}
To get a better idea of the success of deep learning, let us take the ImageNet Challenge~\citep{ILSVRC15} (also known as ILSVRC) as an example. In the classification task, one is given a training dataset consisting of 1.2 million color images with $1000$ categories, and the goal is to classify images based on the input pixels. The performance of a classifier is then evaluated on a test dataset of 100 thousand images, and in the end the top-5 error\footnote{The algorithm makes an error if the true label is not contained in the $5$ predictions made by the algorithm.} is reported. Table~\ref{tab:intro} highlights a few popular models and their corresponding performance. As can be seen, deep learning models (the second to the last rows) have a clear edge over shallow models (the first row) that fit linear models$\,$/$\,$tree-based models on handcrafted features. This significant improvement raises a foundational question:

\begin{itemize}
\centering
\item[] \mbox{\emph{Why is deep learning better than classical methods on tasks like image recognition?}}
\end{itemize}



%\begin{center}
%\emph{why is deep learning better than classical methods on tasks like image recognition?}
%\end{center}

%For comparison, a Stanford Ph.D. student achieved a $5.1\%$ error on the same task.

%Clearly, generally over-parametrized and deep models have superior performance compared with shallow models\footnote{Shallow models are typically feature extraction followed by linear/nonlinear SVMs, which are used in ILSVRC classification challenge before 2012.}.


%One well-known example of the success of deep learning is the ImageNet Challenge~\citep{ILSVRC15}, where the goal is to classify images based on input pixels.  In the classification task, the training dataset consists of 1.2 million color images (typically, $256 \times 256$) with $1000$ categories. The performance of a classifier is evaluated on a test dataset of 100 thousands images, and the top-5 error\footnote{The top-5 error refers to the ratio of mismatch between the true label $y$ and $5$ labels predicted by an algorithm (a mismatch is counted if $y$ is not any of the $5$ predicted labels). It is a more tolerant criterion than the usual classification/top-1 error.} is reported. In Table~\ref{tab:intro}, we highlight several famous models and their performance, among which the last two models will be discussed in Section~\ref{sec:skip}. Clearly, generally over-parametrized and deep models have superior performance compared with shallow models\footnote{Shallow models are typically feature extraction followed by linear/nonlinear SVMs, which are used in ILSVRC classification challenge before 2012.}.





%\subsection{Classical methods revisited}
%
%Before delving into deep learning, let us imagine what a data analyst would do to attack a classification problem in the traditional way. The data analyst may start with simple visualization and logistic regression (or support vector machine). After realizing the nonlinear nature of the data, he$\,$/$\,$she carefully reexamines the task and falls into thoughts, before coming up with key structural assumptions. For example, the density function is supposed to be quite smooth, or there should be a sparse representation of the data after proper transformation, and so on. With these assumptions in mind, the data analyst then diligently spends hours working on feature construction, before finally running standard packages from \texttt{R}, \texttt{MATLAB}, or \texttt{Python} to fit a model.
%
%%One of the most popular methods for classification (resp.~regression) is logistic regression (resp.~linear regression). Due to its linear nature, it can be very restrictive in modeling nonlinear decision boundaries in complex problems such as image classification. Decades of research efforts in nonparametric statistics, semiparametric statistics, machine learning, etc., have been devoted to discovering routes in the world of nonlinearity. Prominent examples include basis expansion (e.g., splines, wavelets, reproducing kernel Hilbert spaces)~\citep{wahba1990spline, daubechies1992ten}, local kernel smoothing (e.g., nearest neighbors, local regression, polynomial smoothing)~\citep{fan2018local, loader2006local}, tree-based methods (e.g., decision trees, random forests)~\citep{breiman1984classification, breiman2001random}, etc.
%
%%{\scriptsize
%%\begin{equation*}
%%\begin{array}{lc}
%%\text{Known structure} & \left\{ \begin{array}{lcl} \text{smoothness} & \longrightarrow & \text{kernel smoothing, fourier/wavelets transformation} \\ \text{sparsity} & \longrightarrow & \text{thresholding}, \ell_1 \text{-, SCAD-regularization, basis pursuit} \\ \text{low rank} & \longrightarrow & \text{PCA, SDP, nuclear-norm minimization} \\ \cdots & \longrightarrow & \cdots \end{array} \right. \\
%%\text{Unclear structure?} & \begin{array}{lcl} & \hspace{-43mm}  \longrightarrow & \text{deep learning}  \end{array}
%%\end{array}
%%\end{equation*}
%%}
%
%Indeed, if a problem has a certain known structure, such as smoothness, sparsity, and low-rankness, then one can usually devise near-optimal statistical methods in the minimax sense~\citep{stone1982optimal, donoho1994ideal, candes2009power,chen2019noisy}. However, for datasets such as images and natural languages, it is not \emph{a priori} obvious what structure best characterizes these datasets. In addition, classical methods do not seem to be flexible enough to model the nonlinear dependency between $y$ and $\xx$ for these datasets.
%
%The success of deep learning is an indication that deep learning models represent a very different class of functions that are suitable for highly complex datasets. In particular, while the \textit{curse of dimensionality} is often encountered with the class of smooth functions, deep learning excels at handling many high-dimensional data (images, videos, etc.). In general, the curse of dimensionality refers to the phenomenon that the sample and computational complexities have to grow exponentially with the dimension $d$ to achieve a given accuracy. For example, in nonparametric regression, the optimal convergence rate for a Lipschitz regression function is $O(n^{-2/(2+d)})$. Therefore to obtain a small error $\varepsilon$, the number of samples has an exponential dependence on $d$~\citep{stone1982optimal}. Smoother functions, e.g., functions with all derivatives up to the $m$-th order, can have a better convergence rate $O(n^{-2m/(2m+d)})$; however, it is often unrealistic to expect high smoothness in high-dimensional real data. Apart from the sample complexity, many optimization problems require computational cost that is exponential in terms of the input dimension in the worst case (a.k.a.~NP-hardness)~\citep{arora2009computational}. Nevertheless, generic algorithms (e.g., stochastic gradient descent, \citealp{robbins1951stochastic}) are powerful enough to find a good classifier with the deep learning models despite its large size. Due to the huge difference, deep learning can be regarded as a new framework for statistical modeling and methods.
%
%%These pessimistic results on exponential dependency are related to a very simple intuition: generally it requires $O((1/\varepsilon)^d)$ evaluations on a grid of a cube to determine a function within $\varepsilon$ tolerance \citep{donoho2000high}.
%
%%Deep learning is not only good at representing complex functions, but also efficient in finding them. This is perhaps more surprising, since the complexities of deep learning models are much higher than classical methods.
%
%%In particular, the \textit{curse of dimensionality} is one difficulty often encountered under the classical structural assumptions. In general, it refers to the phenomenon that sample and computational complexities have to grow exponentially with the dimension $d$ to achieve a given accuracy. For example, in nonparametric regression, the optimal convergence rate for a Lipschitz regression function is $O(n^{-2/(2+d)})$. Therefore to obtain a small error $\varepsilon$, the number of samples has an exponential dependence on $d$~\citep{stone1982optimal}. Smoother functions, e.g., functions with all derivatives up to the $m$-th order, can have better convergence rate $O(n^{-2m/(2m+d)})$, however, it is often unrealistic to expect high smoothness in high-dimensional real data.  Apart from the sample complexity, in the worst case, many optimization problems require computational cost that is exponential in terms of the input dimension (a.k.a.~NP-hardness)~\citep{arora2009computational}. These pessimistic results on exponential dependency are related to a very simple intuition: generally it requires $O((1/\varepsilon)^d)$ evaluations on a grid of a cube to determine a function within $\varepsilon$ tolerance \citep{donoho2000high}.
%
%%The success of deep learning is an indication that deep learning models represent a very different class of functions that are suitable for highly complex datasets. Moreover, on the computational side, generic algorithms (e.g., stochastic gradient descent~\citep{robbins1951stochastic}) are powerful enough to find a good classifier, despite the large size of the model. Due to the huge difference, deep learning can be regarded as a new framework for statistical modeling and methods.
%%
%%%In a way, deep learning sheds new lights to the \textit{curse of dimensionality} by learning certain complicated nonlinearity efficiently.
%%
%%%Deep learning models do not seem to suffer from this curse of dimensionality, likely because they represent a different class of functions that are suitable for many real datasets.
\begin{figure}
\centering
\includegraphics[width = 0.75\textwidth]{ImagnetFilterVisualization2by6}
\caption{Visualization of trained filters in the first layer of AlexNet. The model is pre-trained on ImageNet and is downloadable via PyTorch package \texttt{torchvision.models}. Each filter contains $11 \times 11 \times 3$ parameters and is shown as an RGB color map of size $11 \times 11$.}\label{fig:vis}
\end{figure}
\subsection{Intriguing new characteristics of deep learning}

It is widely acknowledged that two indispensable factors contribute to the success of deep learning, namely (1) huge datasets that often contain millions of samples and (2) immense computing power resulting from clusters of graphics processing units (GPUs). Admittedly, these resources are only recently available: the latter allows to train larger neural networks which reduces biases and the former enables variance reduction. However, these two alone are not sufficient to explain the mystery of deep learning due to some of its ``dreadful'' characteristics: (1) \emph{over-parametrization}: the number of parameters in state-of-the-art deep learning models is often much larger than the sample size (see Table~\ref{tab:intro}), which gives them the potential to overfit the training data, and (2) \emph{nonconvexity}: even with the help of GPUs, training deep learning models is still NP-hard~\citep{arora2009computational} in the worst case due to the highly {nonconvex} loss function to minimize. In reality, these characteristics are far from nightmares. This sharp difference motivates us to take a closer look at the salient features of deep learning, which we single out a few below. %\cm{We should shorten the introduction and the bullet points.}


%\begin{figure}
%\centering
%\begin{tabular}{cc}
%\includegraphics[width = 0.3\textwidth]{Figure/MNIST_whole.pdf} & \includegraphics[width = 0.3\textwidth]{Figure/my_visualization}  \tabularnewline
%(a) One-to-many & (b) Many-to-one
%\end{tabular}
%\caption{Vanilla RNNs with different inputs/outputs settings. (a) has one input but multiple outputs; (b) has multiple inputs but one output; (c) has multiple inputs and outputs. Note that the parameters are shared across time steps.}\label{fig:mnist}
%\end{figure}

%The departure from conventional wisdoms and classical methods generates many intriguing new phenomena.
%All of this motivates us to take a closer look at what is new in deep learning and what are the connections with conventional methods and wisdoms. Below we single out three salient features of deep learning.

\subsubsection{Depth}
Deep learning expresses complicated nonlinearity through composing many nonlinear functions; see~(\ref{model:1}). The rationale for this multilayer structure is that, in many real-world datasets such as images, there are different levels of features and lower-level features are building blocks of higher-level ones. See~\cite{yosinski2015understanding} for a visualization of trained features of convolutional neural nets; here in Figure~\ref{fig:vis}, we sample and visualize weights from a pre-trained AlexNet model. This intuition is also supported by empirical results from physiology and neuroscience~\citep{hubel1962receptive, abbasi2018deeptune}. The use of function composition marks a sharp difference from traditional statistical methods such as projection pursuit models \citep{friedman1981projection} and multi-index models \citep{li1991sliced, cook2007fisher}. It is often observed that depth helps efficiently extract features that are representative of a dataset. In comparison, increasing width (e.g., number of basis functions) in a shallow model leads to less improvement. This suggests that deep learning models excel at representing a very different function space that is suitable for complex datasets.


%Although each nonlinear component a simple element-wise nonlinear function, stacking many such functions allows to model complicated nonlinearity well.

%traditional methods often require hand-crafted features that are not flexible enough for a task.
%This suggests that deep learning models have great representation power
%where much focus is on structured smoothness (e.g., Sobolev space) and structure sparsity (e.g., sparse signals, low rank).
%This may explain why the curse of dimensionality, usually found in statistics, is not observed in deep learning.
%Deep models bring not only representational power, but new computational challenges as well.
%New challenges arise for deep models, and they are particularly acute in the computational aspect. For example, as the number of composed functions increases, training can be very slow; also the gradients are usually vanishing or exploding, which leads to numerical instability~\citep{hochreiter1991untersuchungen, hochreiter2001gradient}. Many techniques are proposed to address these issues: for example, weight sharing and downsampling speed up the computation, and batch normalization and skip connections stabilize gradient flows (Section~\ref{sec:pop} and~\ref{sec:opt}).
%two perspectives, namely its representation power and generalization power. The former refers to the capacity of deep neural nets to approximate functions (i.e., what is the function space) and the latter refers to the effectiveness of controlling out-of-sample errors (i.e., why they achieve small test errors).

%Moreover, deep learning algorithms can find good representations and thus good prediction results with reasonable computational costs. We identify three important characteristics of deep learning in the following. The first two will be highlighted in this paper.

%\subsubsection{Over-parametrization.}

\subsubsection{Algorithmic regularization}
%\subsubsection{Entanglement with training algorithms}
The statistical performance of neural networks (e.g., test accuracy) depends heavily on the particular optimization algorithms used for training~\citep{NIPS2017_7003}. This is very different from many classical statistical problems, where the related optimization problems are less complicated. For instance, when the associated optimization problem has a relatively simple structure (e.g., convex objective functions, linear constraints), the solution to the optimization problem can often be unambiguously computed and analyzed. However, in deep neural networks, due to over-parametrization, there are usually many local minima with different statistical performance \citep{li2018visualizing}. Nevertheless, common practice runs stochastic gradient descent with random initialization and finds model parameters with very good prediction accuracy. %For example, the choice of batch size has significant influence on test accuracy (see Section~\ref{sec:opt}). This phenomenon calls for understanding optimization algorithms from the statistical lens.\\


%Recent research suggests that certain optimization algorithms, such as gradient descent and stochastic gradient descent, works well with over-parametrized models---with stochastic gradient descent, a huge model does not lead to inferior generalization power (test error). Also, the algorithms exhibit \textit{implicit regularization}, which means that even without explicit regularization such at $\ell_2$ regularizers or $\ell_1$ regularizers, optimization algorithms can produce regularized functions/classifiers $f_{\btheta}$.

\subsubsection{Implicit prior learning}
It is well observed that deep neural networks trained with only the raw inputs (e.g., pixels of images) can provide a useful representation of the data. This means that after training, the units of deep neural networks can represent features such as edges, corners, wheels, eyes, etc.; see~\cite{yosinski2015understanding}. Importantly, the training process is automatic in the sense that no human knowledge is involved (other than hyper-parameter tuning). This is very different from traditional methods, where algorithms are designed after structural assumptions are posited. %Because of this reason, tasks involving complex large-scale datasets can become more efficient in terms of labor investment.
It is likely that training an over-parametrized model efficiently learns and incorporates the prior distribution $p(\xx)$ of the input, even though deep learning models are themselves discriminative models. With automatic representation of the prior distribution, deep learning typically performs well on similar datasets (but not very different ones) via transfer learning.

%\subsubsection{Over parametrization and implicit prior.}

%Over parametrization can drive easily the training errors to zero. This is traditionally regarded as over-fitting and can have an adverse effect on the generalization error (prediction erro).  However, in highly complex high-dimensional model, the biases are the key factors in prediction error and are hard to control. Overparameterized models are typically have lower biases than, for example, the nearest neighborhood learning. Since the algorithms are trained based on a large amount of samples, this incorporates implicitly the prior of the data, so long as the training examples are not biasedly sampled, namely the testing samples (features) are similar to the those from the training samples.  When this over-parametrization is coupled with the aforementioned implicit regularization, the variance does not increase as much.  Indeed, the generalization error controlled by two factors: training errors and Rademacher complexity of the model {\bf Ref?}.  For overparametrized models, the first term is nearly zero. Yet, overparametrized models do not need many steps of iterations of gradient steps to drive the training error to zero and hence its Rademacher complexity is also controlled.  This provides a heuristic explanation the complex story behind deep learning's successes.


%new models and algorithms that enable scalable training of deep neural nets. While the first two factors involve resources that are only recently available, the third requires clever algorithmic insights---in this aspect, research communities have recently contributed many novel ideas, including residual nets, dropout training, generative models, batch normalization, etc. %Nevertheless, the architectures of deep learning, such as the deep neural networks and recurrent neural networks, are not completely new ideas.
%These innovations are built upon ideas and methods developed over several decades. Deep learning architectures and training can be traced back to as early as the perceptron algorithm~\citep{rosenblatt1958perceptron}, and they gradually evolved into the modern appearance~\citep{fukushima1979neural, lecun1989backpropagation, krizhevsky2012imagenet}. For a comprehensive monograph and the history, we refer the readers to~\citep{lecun2015deep, schmidhuber2015deep, deeplearningbook}.
%%The architectures of deep learning, such as the deep neural networks and recurrent neural networks, are not completely new ideas. They can be traced back to as early as the perceptron algorithm~\citep{rosenblatt1958perceptron}, and gradually evolved into its modern appearance~\citep{fukushima1979neural, lecun1989backpropagation, krizhevsky2012imagenet}. However, there are important resources that are only recently available and are critical to the success of deep learning: (1) huge datasets that often contain millions of samples, (2) immense computing power, especially from graphics processing units (GPUs). Besides, research communities have contributed many novel ideas, including, notably, dropout training, residual nets, generative adversarial nets, batch normalization, etc. For a comprehensive monograph or the history, we refer the readers to~\citep{lecun2015deep, schmidhuber2015deep}.
%
%Fundamentally, deep learning is a method for the classical statistical problem: given a training data $\{(\xx_i,y_i)\}_{1\leq i \leq n}$ where $\xx_i \in \mathbb{R}^{d}$ is the (raw) input and $y_i \in \mathbb{R}$ (categorical or real-valued) is the output, we want to find a function $f_{\btheta}: \mathbb{R}^{d} \mapsto \mathbb{R}$ with good prediction performance on unseen data. As with standard approaches in statistics and machine learning, the parameters $\btheta$ are determined by minimizing certain loss functions. Often, the input dimension $d$ is very large (e.g., the number of pixels in a color image), and the dependence between $y_i$ and $\xx_i$ is highly nonlinear (e.g., predicting image categories based on raw pixels); these together pose great challenges for traditional approaches. Deep learning uses a very large model with many layers, and the total number of parameters as large as hundreds of millions. A vanilla feed-forward neural network has the following structure:
%\begin{equation}\label{model:1}
%f_{\btheta}(\xx) = \bA_\ell \bsigma_*(\cdots \bsigma_*(\bA_2 \bsigma_*(\bA_1 \xx))),
%\end{equation}
%where $\bA_1,\cdots, \bA_\ell$ are matrices with appropriate sizes and $\bsigma_*(\cdot)$ is a nonlinear function applied in an element-wise fashion.
%
%With reasonably good computing resources, deep learning usually achieves a small training error (in-sample error) on the training data, and perhaps surprisingly, a small test error (out-of-sample error) on separate test data. This clearly stands in contrast with the traditional wisdom, which typically works with smaller models to avoid overfitting. Before suggesting perspectives to explain this apparent discrepancy, we first have a look at classical statistical methods.
%
%\subsubsection{Classical methods for nonlinearity.}
%
%One of the most popular methods for classification (resp.~regression) is logistic regression (resp.~linear regression). Due to its linear nature, it can be very restrictive in modeling nonlinear decision boundaries in complex problems such as image classification. Decades of research efforts in nonparametric statistics, semiparametric statistics, machine learning, etc., have been devoted to discovering routes in the world of nonlinearity. Prominent examples include basis expansion (e.g., splines, wavelets, reproducing kernel Hilbert spaces)~\citep{wahba1990spline, daubechies1992ten}, local kernel smoothing (e.g., nearest neighbors, local regression, polynomial smoothing)~\citep{fan2018local, loader2006local}, tree-based methods (e.g., decision trees, random forests)~\citep{breiman1984classification, breiman2001random}, etc.
%
%\begin{equation*}
%\begin{array}{lc}
%\text{Known structure} & \left\{ \begin{array}{lcl} \text{smoothness} & \longrightarrow & \text{kernel smoothing, fourier/wavelets transform} \\ \text{sparsity} & \longrightarrow & \text{thresholding}, \ell_1 \text{-regularization, basis pursuit} \\ \text{low rank} & \longrightarrow & \text{PCA, SDP, nuclear-norm minimization} \\ \cdots & \longrightarrow & \cdots \end{array} \right. \\
%\text{Unclear structure?} & \begin{array}{lcl} & \hspace{-31mm}  \longrightarrow & \text{deep learning}  \end{array}
%\end{array}
%\end{equation*}
%
%If a problem has a certain known structure, such as smoothness, sparsity, and low rank, then one can usually devise (near-)optimal statistical methods in the minimax sense~\citep{stone1982optimal, donoho1994ideal, candes2009power}. However, for datasets such as images and natural languages, it is not \emph{a priori} obvious what structure best characterizes these datasets. Even with hand-crafted features, the resulting model can be unsatisfactory. For these datasets, classical methods do not seem to be flexible enough to handle nonlinearity.
%
%In particular, the \textit{curse of dimensionality} is one difficulty often encountered under the classical structural assumptions. In general, it refers to the phenomenon that sample and computational complexities have to grow exponentially with the dimension $d$ to achieve a given accuracy. For example, in nonparametric regression, the optimal convergence rate for a Lipschitz regression function is $O(n^{-2/(2+d)})$. Therefore to obtain a small error $\varepsilon$, the number of samples has an exponential dependence on $d$~\citep{stone1982optimal}. Smoother functions, e.g., functions with all derivatives up to $m$th order, can have better convergence rate $O(n^{-2m/(2m+d)})$, however, it is often unrealistic to expect high smoothness in high-dimensional real data.  Apart from the sample complexity, in the worst case, many optimization problems require computational cost that is exponential in terms of the input dimension (a.k.a.\ NP-hardness)~\citep{arora2009computational}. These pessimistic results on exponential dependency are related to a very simple intuition: generally it requires $O((1/\varepsilon)^d)$ evaluations on a grid of a cube to determine a function within $\varepsilon$ tolerance.
%
%The success of deep learning is an indication that deep learning models represent a very different class of functions that are suitable for highly complex datasets. Moreover, on the computational side, gradient-based methods are generally powerful enough to find a good classifier, despite the large size of the model. Due to the huge difference, deep learning can be regarded as a new framework for statistical modeling and methods.
%
%%In a way, deep learning sheds new lights to the \textit{curse of dimensionality} by learning certain complicated nonlinearity efficiently.
%
%%Deep learning models do not seem to suffer from this curse of dimensionality, likely because they represent a different class of functions that are suitable for many real datasets.
%
%\subsubsection{Characteristics of deep learning.}
%
%%Despite its empirical success, theoretical justifications for this is still at infancy. Below we summarizes the efforts which have been made towards explaining the mystery of deep learning, namely its representation power and generalization power. The former refers to the capacity of deep neural nets to approximate functions, and the latter refers to the effectiveness of controlling out-of-sample errors.
%
%
%%Another interesting characteristic of deep neural nets that will not be covered in this paper is automatic feature extraction. While classical statistical methods require pre-specified features (basis, dictionaries,
%

\begin{figure}
\centering
\begin{tabular}{cc}
\includegraphics[width = 0.45\textwidth]{MNIST.pdf} & \includegraphics[width = 0.45\textwidth]{train_test_accuracy.pdf}    \tabularnewline
(a) MNIST images & (b) training and test accuracies
\end{tabular}
\caption{(a) shows the images in the public dataset MNIST; and (b) depicts the training and test accuracies along the training dynamics. Note that the training accuracy is approaching $100\%$ and the test accuracy is still high (no overfitting). }\label{fig:mnist}
\end{figure}



\subsection{Towards theory of deep learning}

Despite the empirical success, theoretical support for deep learning is still in its infancy. Setting the stage, for any classifier $f$, denote by $\mathbb{E}(f)$ the expected risk on fresh sample (a.k.a.~test error, prediction error or generalization error), and by $\mathbb{E}_n(f)$ the empirical risk$\,$/$\,$training error averaged over a training dataset. Arguably, the key theoretical question in deep learning is %\cm{The introduction here is not mathematically and a bit unclear to those who do not have previous experience with learning theory.}
\begin{center}
\emph{why is $\mathbb{E}(\hat{f}_{n})$ small, where $\hat{f}_{n}$ is the classifier returned by the training algorithm?}
\end{center}

We follow the conventional approximation-estimation decomposition (sometimes, also bias-variance tradeoff) to decompose the term $\mathbb{E}(\hat{f}_{n})$ into two parts.
Let $\cF$ be the function space expressible by a family of neural nets.
Define $f^* = \argmin_f \mathbb{E}(f)$ to be the best possible classifier and $f^*_{\cF} = \argmin_{f \in \cF} \mathbb{E}(f)$ to be the best classifier in $\cF$. Then, we can decompose the excess error $\cE \triangleq \mathbb{E}(\hat f_n) - \mathbb{E}(f^*)$ into two parts:
\begin{equation}
\cE = \underbrace{\mathbb{E}(f^*_{\cF}) - \mathbb{E}(f^*)}_{\text{approximation error}} ~ + ~ \underbrace{\mathbb{E}(\hat f_n) - \mathbb{E}(f^*_{\cF})}_{\text{estimation error}}.\label{eq:error_decomposition}
\end{equation}
Both errors can be small for deep learning (cf. Figure~\ref{fig:mnist}), which we explain below.
%where the approximation error is governed by its representation power and the estimation error is related to its generalization ability.
\begin{itemize}
\item{The \emph{approximation error} is determined by the function class $\cF$. Intuitively, the larger the class, the smaller the approximation error. Deep learning models use many layers of nonlinear functions (Figure~\ref{fig:FFNN})that can drive this error small. Indeed, in Section~\ref{sec:approx}, we provide recent theoretical progress of its representation power. For example, deep models allow efficient representation of interactions among variable while shallow models cannot.
}
\item{The \emph{estimation error} reflects the generalization power, which is influenced by both the complexity of the function class $\mathcal{F}$ and the properties of the training algorithms. Interestingly, for \emph{over-parametrized} deep neural nets, stochastic gradient descent typically results in a near-zero  training error (i.e., $\mathbb{E}_{n}(\hat{f}_{n})\approx 0$; see e.g. left panel of Figure~\ref{fig:mnist}). Moreover, its generalization error $\mathbb{E}(\hat{f}_{n})$ remains small or moderate. This ``counterintuitive'' behavior suggests that for over-parametrized models, gradient-based algorithms enjoy benign statistical properties; we shall see in Section~\ref{sec:generalization} that gradient descent enjoys \textit{implicit regularization} in the over-parametrized regime even without explicit regularization (e.g., $\ell_2$ regularization).
}
\end{itemize}

The above two points lead to the following heuristic explanation of the success of deep learning models. The large depth of deep neural nets and heavy over-parametrization lead to small or zero training errors, even when running simple algorithms with moderate number of iterations. In addition, these simple algorithms with moderate number of steps do not explore the entire function space and thus have limited complexities, which results in small generalization error with a large sample size. Thus, by combining the two aspects, it explains heuristically that the test error is also small.

\subsection{Roadmap of the paper}

We first introduce basic deep learning models in Sections~\ref{sec:super}--\ref{sec:unsup}, and then examine their representation power via the lens of approximation theory in Section~\ref{sec:approx}. Section~\ref{sec:opt} is devoted to training algorithms and their ability of driving the training error small. Then we sample recent theoretical progress towards demystifying the generalization power of deep learning in Section~\ref{sec:generalization}. Along the way, we provide our own perspectives, and at the end we identify a few interesting questions for future research in Section~\ref{sec:discuss}. The goal of this paper is to present suggestive methods and results, rather than giving conclusive arguments (which is currently unlikely) or a comprehensive survey. We hope that our discussion serves as a stimulus for new statistics research.


%\subsection{Paper organization}
%Sections~\ref{sec:super}-\ref{sec:unsup} describes the basic model classes in deep learning. Specifically, Section~\ref{sec:super} introduces the basic feed-forward neural nets. In Section~\ref{sec:pop}, we introduce two important deep learning models, namely CNNs and RNNs, as well as other modeling techniques. In Section~\ref{sec:unsup}, we turn to unsupervised learning, which include GANs that are used for learning data distributions. In Section~\ref{sec:approx}, we examine the representation power of deep learning via the lens of approximation theory. Section~\ref{sec:opt} is devoted to optimization techniques for deep learning. Section~\ref{sec:generalization} samples recent theoretical progress towards demystifying the generalization power of deep learning. Section~\ref{sec:discuss} identifies a few questions for statistical research.

%\subsection{Notations}
%We use bold fonts to denote vectors and matrices. Similarly, functions with vector values are in the bold font; in particular, $\bsigma, \bsigma_*$ and $\btanh$ denote a vector of unbolded functions applied element-wise to a vector of the same length.

%
%\subsection{Connections with statistical models}
%
%The one-hidden layer neural network model
%\begin{equation*}
%   Y = \bsigma*(\bW_1 \xx) + \varepsilon = \mbox{$\sum_{k=1}^{K_1} $} \sigma_*(\bw_{1,k}^T \xx) + \varepsilon,
%\end{equation*}
%where $\varepsilon$ is the random error, is innately related to the
%projection pursuit model $Y = \sum_{k=1}^{K} f_k(\bbeta_k \xx) + \varepsilon$
%in statistics \citep{friedman1981projection}.  The key difference is that $f_k(\cdot)$ is nonparametric to facilitate the flexibility of the model, whereas $\sigma_*$ is a known function in neural network model to facilitate the computation.  Deeper neural network models can be written as a multi-index model $Y = f(\bW_1\xx, \cdots, \bW_L \xx, \varepsilon)$ \citep{li1991sliced,li1992principal,cook2007fisher,cook2009regression}, where the function $f(\cdot)$ is known in deep neural network models to facilitate computation and high-dimensional indices, whereas in statistics, $f(\cdot)$ is usually nonparametric to facilitate the modeling biases and therefore can not handle too many indices.  Statistical estimation methods such as sliced inverse regression and principal Hessian directions can not handle too many indices.
%
%The neural network model is also related to the principal component regression \citep{fan2017sufficient,stock2002forecasting}.  Suppose that there are $K$ latent factors $\bff$ that drive both $\bx$ and $Y$.  The common technique is to extract latent factors $\bff$ from $\bx$ via principal component analysis and estimate $\bff$ as $\hat{\bff} = \hat{\bXi}^T\bx$ where $\hat \bXi$ consists of principal component directions.  Now, create the prediction indices of $\bTheta \bff$ to predict the value $Y$ \citep{fan2017sufficient}.  This approach is closely related to the two-hidden layer neural network.  The indices $\bTheta$ are trained by the methods such as slice inverse regression via principal component directions.  Again, the main difference is in scalability and activation functions.
% 
%\noindent\RMK{something to mention:} \\
%The rise of deep learning is accompanied with the revolutionary of depth (use a figure or a table to illustrate this).

%empirical success: alphaGo, image recognition, machine translation (BERT???)
\section{Feed-forward neural networks}\label{sec:super}

Before introducing the vanilla feed-forward neural nets, let us set up necessary notations for the rest of this section. We focus primarily on classification problems, as regression problems can be addressed similarly. Given the training dataset $\{ (y_i, \xx_i )\}_{1\leq i\leq n}$ where $y_i \in [K] \triangleq \{1,2,\ldots,K\}$ and $\xx_{i} \in \R^d$ are independent across $i \in [n]$, supervised learning aims at finding a (possibly random) function $\hat f(\xx)$ that predicts the outcome $y$ for a new input $\xx$, assuming $(y, \xx)$ follows the same distribution as $(y_i, \xx_i )$. In the terminology of machine learning, the input $\xx_i$ is often called the \textit{feature}, the output $y_i$ called the \textit{label}, and the pair $(y_i,\bm{x}_i)$ is an \textit{example}. The function $\hat f$ is called the \textit{classifier}, and estimation of $\hat f$ is \textit{training} or \textit{learning}. The performance of $\hat f$ is evaluated through the prediction error $\P(y \neq \hat f(\xx))$, which can be often estimated from a separate test dataset.

As with classical statistical estimation, for each $k \in [K]$, a classifier approximates the conditional probability $\P(y = k | \xx)$ using a function $f_k(\xx; \btheta_k)$ parametrized by $\btheta_k$. Then the category with the highest probability is predicted. Thus, learning is essentially estimating the parameters $\btheta_k$. In statistics, one of the most popular methods is (multinomial) logistic regression, which stipulates a specific form for the functions $f_k(\xx; \btheta_k)$: let $z_k = \xx^\top \bbeta_k + \alpha_k$ and $f_k(\xx; \btheta_k) = Z^{-1} \exp(z_k)$ where $Z = \sum_{k=1}^K \exp(z_k)$ is a normalization factor to make $\{f_k(\xx; \btheta_k)\}_{1\leq k \leq K}$ a valid probability distribution. It is clear that logistic regression induces linear decision boundaries in $\mathbb{R}^{d}$, and hence it is restrictive in modeling nonlinear dependency between $y$ and $\xx$. The deep neural networks we introduce below provide a flexible framework for modeling nonlinearity in a fairly general way.

\subsection{Model setup}

From the high level, deep neural networks (DNNs) use composition of a series of simple nonlinear functions to model nonlinearity
\begin{equation*}
\hh^{(L)} = \bg^{(L)} \circ  \bg^{(L-1)} \circ \ldots \circ \bg^{(1)} (\xx),
\end{equation*}
where $\circ$ denotes composition of two functions and $L$ is the number of hidden layers, and is usually called \emph{depth} of a NN model. Letting $\bm{h}^{(0)}\triangleq\bm{x}$, one can recursively define
$\hh^{(l)} =  \bg^{(l)} \big(\bm{h}^{(l-1)}\big)$ for all $\ell = 1,2,\ldots, L$. The \textit{feed-forward neural networks}, also called the \textit{multilayer perceptrons} (MLPs), are neural nets with a specific choice of $\bg^{(l)}$: for $\ell = 1,\ldots,L$, define
\begin{equation}\label{eq:fc}
\hh^{(\ell)} = \bg^{(l)} \big(\bm{h}^{(l-1)}\big) \triangleq \bsigma \big(\bW^{(\ell)} \hh^{(\ell-1)} + \bb^{(\ell)}  \big),
\end{equation}
where $\bW^{(l)}$ and  $\bb^{(l)}$ are the weight matrix and the bias$\,$/$\,$intercept, respectively, associated with the $l$-th layer, and $\bsigma(\cdot)$ is usually a simple given (known) nonlinear function called the \textit{activation function}. In words, in each layer $\ell$, the input vector $\hh^{(\ell-1)}$ goes through an affine transformation first and then passes through a fixed nonlinear function $\bsigma(\cdot)$. See Figure~\ref{fig:FFNN} for an illustration of a simple MLP with two hidden layers. The activation function $\bsigma(\cdot)$ is usually applied element-wise, and a popular choice is the ReLU (Rectified Linear Unit) function:
\begin{equation}
[\bsigma(\zz)]_j = \max\{ z_j, 0 \}.
\end{equation}
Other choices of activation functions include leaky ReLU, $\tanh$ function \citep{maas2013rectifier} and the classical sigmoid function $(1+e^{-z})^{-1}$, which is less used now.

%It is worthwhile noting that the ReLU activation function is a crucial element in most deep learning models. Its popularity is largely due to the fact that its derivative is either $0$ or $1$, which makes training more efficient. See Section~\ref{sec:opt} for discussion on numerical stability. %\TODO{add a figure?}

\begin{figure}
\centering
\includegraphics[scale=0.4]{MLP}\caption{A feed-forward neural network with an input layer, two hidden layers and an output layer. The input layer represents raw features $\{\bm{x}_{i}\}_{1\leq i\leq n}$. Both hidden layers compute an affine transform (a.k.s. indices) of the input and then apply an element-wise activation function $\bsigma(\cdot)$. Finally, the output returns a linear transform followed by the softmax activation (resp.~simply a linear transform) of the hidden layers for the classification (resp.~regression) problem. \label{fig:FFNN}}
\end{figure}

Given an output $\hh^{(L)}$ from the final hidden layer and a label $y$, we can define a loss function to minimize. A common loss function for classification problems is the multinomial logistic loss. Using the terminology of deep learning, we say that $\hh^{(L)}$ goes through an affine transformation and then the \textit{soft-max} function:
\begin{equation*}
f_k(\xx; \btheta) \triangleq  \frac{\exp(z_k)}{\sum_k \exp(z_k)}, \quad \forall\, k\in[K], \qquad \where~ \zz = \bW^{(L+1)} \hh^{(L)} + \bb^{(L+1)} \in \mathbb{R}^{K}.
\end{equation*}
Then the loss is defined to be the cross-entropy between the label $y$ (in the form of an indicator vector) and the score vector $ (f_1(\xx; \btheta),\ldots,f_K(\xx; \btheta))^\top$, which is exactly the negative log-likelihood of the multinomial logistic regression model:
\begin{equation}\label{eq:crossentropy}
\mathcal{L}(\ff(\xx; \btheta), y)=-\sum_{k=1}^K \bbone\{y = k\} \log p_k,
\end{equation}
where $\btheta \triangleq \{ \bW^{(\ell)}, \bb^{(\ell)}: 1\leq \ell \leq L+1\}$.
As a final remark, the number of parameters scales with both the depth $L$ and the width (i.e., the dimensionality of $\bW^{(\ell)}$), and hence it can be quite large for deep neural nets. %Here we abuse the notation $\mathcal{L}$ to denote either the loss on a single example $(y_i, \bm{x}_i)$ or the loss over the whole data $\{(y_i, \bm{x}_i)\}_{1\leq i \leq n}$. \TODO{We may use $\ell(\btheta)$ to represent the sample average.} \cm{Yes, I have adopted this in training section.}

\subsection{Back-propagation in computational graphs}
%Training neural nets is essentially the process of finding parameters to minimize the empirical loss over the whole dataset, i.e.,
%\begin{equation}\label{eq:loss-nn}
%\ell_n(\btheta) \triangleq \frac{1}{n}\sum_{i=1}^{n}\mathcal{L}(\ff(\xx_i; \btheta), y_i).
%\end{equation}

Training neural networks follows the \emph{empirical risk minimization} paradigm that minimizes the loss (e.g., (\ref{eq:crossentropy})) over all the training data. This minimization is usually done via \textit{stochastic gradient descent} (SGD). In a way similar to gradient descent, SGD starts from a certain initial value $\btheta^{0}$ and then iteratively updates the parameters $\btheta^{t}$ by moving it in the direction of the negative gradient.
%In a way similar to the gradient descent method, starting from some initial values, the parameters $\btheta^{t}$ are iteratively updated by moving a step in the direction of the negative gradient $-\nabla \ell_n(\btheta^{t})$ for certain empirical loss function $\ell_n$.
The difference is that, in each update, a small subsample $\mathcal{B} \subset [n]$ called a \textit{mini-batch}---which is typically of size 32--512---is randomly drawn and the gradient calculation is only on $\mathcal{B}$ instead of the full batch $[n]$.  This saves considerably the computational cost in calculation of gradient.  By the law of large numbers, this stochastic gradient should be close to the full sample one, albeit with some random fluctuations.  A pass of the whole training set is called an \textit{epoch}. Usually, after several or tens of epochs, the error on a validation set levels off and training is complete. See Section~\ref{sec:opt} for more details and variants on training algorithms.

The key to the above training procedure, namely SGD, is the calculation of the gradient $\nabla \ell_{\cB}(\btheta)$, where
\begin{equation}\label{eq:loss-nn}
\ell_{\cB}(\btheta) \triangleq |\cB|^{-1} \sum_{i \in \cB} \mathcal{L}(\ff(\xx_i ; \btheta), y_i).
\end{equation}
Gradient computation, however, is in general nontrivial for complex models, and it is susceptible to numerical instability for a model with large depth. Here, we introduce an efficient approach, namely \textit{back-propagation}, for computing gradients in neural networks.

%It essentially follows the principle of the gradient descent method, which is the algorithmic routine of many statistics and machine learning problems.
%This makes gradient calculation very efficient.
%For neural networks, gradient calculation is based on \textit{back-propagation} \citep{rumelhart1985learning}, which
Back-propagation \citep{rumelhart1985learning} is a direct application of the chain rule in networks. As the name suggests, the calculation is performed in a backward fashion: one first computes $\partial \ell_{\cB}/\partial \hh^{(L)}$, then $\partial \ell_{\cB}/\partial \hh^{(L-1)}$, $\ldots$, and finally $\partial \ell_{\cB}/\partial \hh^{(1)}$. For example, in the case of the ReLU activation function\footnote{The issue of non-differentiability at the origin is often ignored in implementation.}, we have the following recursive$\,$/$\,$backward relation
\begin{equation}\label{eq:grad}
\frac{\partial \ell_{\cB}}{\partial \hh^{(\ell-1)}} =  \frac{\partial \hh^{(\ell)}}{\partial \hh^{(\ell-1)}} \cdot \frac{\partial \ell_{\cB}}{\partial \hh^{(\ell)}} = (\bW^{(\ell)})^\top \mathsf{diag}\left( \bbone\{\bW^{(\ell)} \hh^{(\ell-1)} + \bb^{(\ell)}  \ge \bm{0}\}  \right) \frac{\partial \ell_{\cB}}{\partial \hh^{(\ell)}}
\end{equation}
where $\mathsf{diag}(\cdot)$ denotes a diagonal matrix with elements given by the argument. Note that the calculation of $\partial \ell_{\cB} / \partial \hh^{(\ell-1)}$ depends on $\partial \ell_{\cB} / \partial \hh^{(\ell)}$, which is the partial derivatives from the next layer. In this way, the derivatives are ``back-propagated'' from the last layer to the first layer. These derivatives $\{\partial \ell_{\cB} / \partial \hh^{(\ell)}\}$ are then used to update the parameters. For instance, the gradient update for $\bW^{(\ell)}$ is given by
\begin{equation}\label{eq:Wupdate}
\bW^{(\ell)} \leftarrow \bW^{(\ell)} - \eta \frac{\partial \ell_{\cB}}{\partial \bW^{(\ell)}}, \quad \where\quad \frac{\partial \ell_{\cB}}{\partial W_{jm}^{(\ell)}} = \frac{\partial \ell_{\cB}}{\partial h_j^{(\ell)}} \cdot \sigma' \cdot h_m^{(\ell-1)},
\end{equation}
where $\sigma' = 1$ if the $j$-th element of $\bW^{(\ell)} \hh^{(\ell-1)} + \bb^{(\ell)}$ is nonnegative, and $\sigma' = 0$ otherwise. The step size $\eta >0$, also called the \textit{learning rate}, controls how much parameters are changed in a single update.

\begin{figure}[t]
\centering
\includegraphics[width=0.75\textwidth]{compuGraph2}\caption{The computational graph illustrates the loss \eqref{eq:regloss}. For simplicity, we omit the bias terms. Symbols inside nodes represent functions, and symbols outside nodes represent function outputs (vectors/scalars). {\normalfont \texttt{matmul}} is matrix multiplication, {\normalfont \texttt{relu}} is the ReLU activation, {\normalfont \texttt{cross entropy}} is the cross entropy loss, and {\normalfont \texttt{SoS}} is the sum of squares.} \label{fig:comgraph}
\end{figure}

A more general way to think about neural network models and training is to consider \textit{computational graphs}. Computational graphs are directed acyclic graphs that represent functional relations between variables. They are very convenient and flexible to represent function composition, and moreover, they also allow an efficient way of computing gradients. Consider an MLP with a single hidden layer and an $\ell_2$ regularization:
\begin{equation}\label{eq:regloss}
\ell_{\cB}^\lambda (\btheta) = \ell_{\cB}(\btheta) + r_\lambda(\btheta) = \ell_{\cB}(\btheta) + \lambda \Big( \sum_{j,j'} \big(W_{j,j'}^{(1)}\big)^2 + \sum_{j,j'} \big(W_{j,j'}^{(2)}\big)^2 \Big),
\end{equation}
where $\ell_{\cB}(\btheta)$ is the same as \eqref{eq:loss-nn}, and $\lambda \ge 0$ is a tuning parameter. A similar example is considered in~\cite{deeplearningbook}. The corresponding computational graph is shown in Figure~\ref{fig:comgraph}. Each node represents a function (inside a circle), which is associated with an output of that function (outside a circle). For example, we view the term $\ell_{\cB}(\btheta)$ as a result of $4$ compositions: first the input data $\xx$ multiplies the weight matrix $\bW^{(1)}$ resulting in $\uu^{(1)}$, then it goes through the ReLU activation function \texttt{relu} resulting in $\hh^{(1)}$, then it multiplies another weight matrix $\bW^{(2)}$ leading to $\pp$, and finally it produces the cross-entropy with label $y$ as in \eqref{eq:crossentropy}. The regularization term is incorporated in the graph similarly.

A forward pass is complete when all nodes are evaluated starting from the input $\xx$. A backward pass then calculates the gradients of $\ell_{\cB}^\lambda$ with respect to all other nodes in the reverse direction. Due to the chain rule, the gradient calculation for a variable (say, $\partial \ell_{\cB} / \partial \uu^{(1)}$) is simple: it only depends on the gradient value of the variables ($\partial \ell_{\cB} / \partial \hh$) the current node points to, and the function derivative evaluated at the current variable value ($\bsigma'(\uu^{(1)})$). Thus, in each iteration, a computation graph only needs to (1) calculate and store the function evaluations at each node in the forward pass, and then (2) calculate all derivatives in the backward pass.

Back-propagation in computational graphs forms the foundations of popular deep learning programming softwares, including TensorFlow~\citep{tensorflow2015-whitepaper} and PyTorch~\citep{paszke2017automatic}, which allows more efficient building and training of complex neural net models.  %There are many techniques for building modules in computational graphs, which we shall see in Section~\ref{sec:pop}.

\subsection{Some general remarks}

We give a few remarks about general characteristics of deep neural network models and training. Specific deep neural network models and training are detailed in later sections.
Before we move on to introducing various popular models in deep learning, we pause here to single out several distinctive characteristics of deep neural nets that are widely believed to contribute towards the success of deep learning. Further theoretical justifications are left to Section~\ref{sec:theory}. \cm{The following three paragraphs need to be revised.}

\cm{I think overparametrization might be a major reason, and it seems relevant to Statistics.}

First, during training, stochastic gradient descent (SGD) and its variants are widely used. The subsamples of SGD are typically of size 32--512, and are drawn without replacement from a training set of size from thousands to millions. A pass of the whole training set is called an \textit{epoch}. Usually, after several or tens of epochs, the validation error levels off and training is complete. \TODO{add a figure showing the typical training/test curve} There are at least two reasons why SGD is widely used: (1) it usually speeds up computation in large machine learning tasks \citep{bottou2010large}, which has been well known; (2) it can achieve optimal statistical accuracy, and even outperforms the full-batch gradient descent (that is, $\cB = [n]$) \citep{keskar2016large}. The latter is particularly intriguing from the perspective of statistics, which stirs widespread interest in recent years. See Section~\ref{sec:stochastic-opt} for details of SGD. \TODO{discuss in Section 6.}

Second, prediction performance generally improves as neural nets become deeper (e.g., 5-30 layers). Deep neural networks are usually over-parametrized, which means that the number of parameters is much larger the sample size. The benefits of depth are at least two-fold: (1) deep neural nets form a larger function space, and thus training error is typically smaller due to better fitting; (2) with explicit or even implicit regularization, the test error is also usually smaller. With regard to the first point, an interesting phenomenon is that the effect of increasing depth is much better than of increasing width. In particular, for many problems, DNNs do not seem to suffer from the curse of dimensionality. Thus, in this respect, DNNs brings new and powerful tools for efficient data fitting. The second point is intriguing from the statistical perspective, because over-parametrization would normally run the risk of severe overfitting in other models. Section~\ref{sec:approx} and~\ref{sec:generalization} provide recent theories for the the effects of depth and over-parametrization.

It is easy to understand that a larger model has more capacity for data fitting, so (1) should be well expected.  For deep nets, the benefit in data fitting, namely (1), is much more significant compared with wide shallow nets \TODO{def!}. Many traditional statistical methods such as basis expansion is analogous to shallow nets \TODO{need to discuss this before}. Thus, in this respect, deep neural nets brings very new and powerful tools for efficient data fitting. The benefit in generalization, namely (2), is exciting from the statistics standpoint. While explicit regularization through $\ell_2$ or $\ell_1$ has been well studied, implicit regularization such as SGD is recognized only recently \TODO{check; refs}.

\cm{My friend Chenxi's comment: I wouldn't agree with the second point of "several distinctive characteristics", that the deeper the better. In fact, the current best ImageNet models (e.g. mine https://arxiv.org/abs/1712.00559; better than ResNet ~5\% top-1) only has maximum depth of less than 30. "The deeper the better" may only be true in networks that rely heavily on residual connections.}

Third, the ReLU activation function is a crucial element in most deep learning architectures. \TODO{add a figure to show different activation functions} Its popularity is largely due to the fact that its derivative is either $0$ or $1$, which makes training more efficient. Note that in back-propagation, gradient calculation is multiplicative due to the recursive relation \eqref{eq:grad}. Thus, it is important to keep gradients from being ``killed'', which means that gradients become approximately zero if in the multiplication a factor $\sigma'$ is very small due to an input with large magnitude. Historically, the sigmoid function (a.k.a.\ the logistic function) was the common choice for the activation function, but for deep neural nets, it tends to kill gradients and has inferior training performance \citep{krizhevsky2012imagenet, maas2013rectifier}. In contrast, ReLU or its variants (e.g., leaky ReLU) is much better for training as well as parameter initialization.

\subsection{Numerical experiments}

\begin{lstlisting}[language=Python]
from keras.datasets import mnist
from keras.models import Sequential
from keras.layers import Dense, Flatten

# the data, split between train and test sets
(x_train, y_train),(x_test, y_test) = mnist.load_data()
x_train, x_test = x_train / 255.0, x_test / 255.0

MLP = Sequential([
 Flatten(),
 Dense(512, activation=tf.nn.relu),
 Dense(10, activation=tf.nn.softmax)
])
MLP.compile(optimizer='sgd',
             loss='sparse_categorical_crossentropy',
             metrics=['accuracy'])

MLP.fit(x_train, y_train, epochs=5, batch_size=32,
         verbose=1,
         validation_split=0.1)
score = MLP.evaluate(x_test, y_test)
\end{lstlisting}

%%%%%%%%% Below are some good links
% How PyTorch implements autograd: https://towardsdatascience.com/getting-started-with-pytorch-part-1-understanding-how-automatic-differentiation-works-5008282073ec
% On activation functions: https://medium.com/@kanchansarkar/relu-not-a-differentiable-function-why-used-in-gradient-based-optimization-7fef3a4cecec


\section{Popular models}\label{sec:pop}


Moving beyond vanilla feed-forward neural networks, we introduce two other popular deep learning models, namely, the convolutional neural networks (CNNs) and the recurrent neural networks
(RNNs). %Both of them have demonstrated their power and usefulness in practice; see Section~\ref{sec:intro} for examples of their applications in image recognition, machine translation, etc.
One important characteristic shared by the two models is \textit{weight sharing}, that is some model parameters are identical across locations in CNNs or across time in RNNs. This is related to the notion of translational invariance in CNNs and stationarity in RNNs. At the end of this section, we introduce a modular thinking for constructing more flexible neural nets.
%Mathematically, CNNs and RNNs are both nonlinear compositional mappings from an input $\bm{x}$ to an output $y$, with differences lying in their building blocks.
%In what follows, we shall discuss these two models in detail.

%We will also introduce advanced concepts of creating variations of neural nets, including residual neural networks and modules. These ideas provides additional modeling flexibility, as well as computational and statistical gains.

\subsection{Convolutional neural networks}\label{sec:CNN}

The convolutional neural network (CNN) \citep{lecun1998gradient, fukushima1982neocognitron} is a special type of feed-forward neural networks that is tailored for image processing. More generally, it is suitable for analyzing data with salient spatial structures. In this subsection, we focus on image classification using CNNs, where the raw input (image pixels) and features of each hidden layer are represented by a 3D tensor $\bm{X}\in\mathbb{R}^{d_{1}\times d_{2}\times d_{3}}$. Here, the first two dimensions $d_1, d_2$ of $\bm{X}$ indicate spatial coordinates of an image while the third $d_3$ indicates the number of channels. For instance, $d_3$ is $3$ for the raw inputs due to the red, green and blue channels, and $d_3$ can be much larger (say, 256) for hidden layers. Each channel is also called a \textit{feature map}, because each feature map is specialized to detect the same feature at different locations of the input, which we will soon explain. %\TODO{better to have a figure}
We now introduce two building blocks of CNNs, namely the convolutional layer and the pooling layer.
\begin{enumerate}


\item \emph{Convolutional layer (CONV)}. A convolutional layer has the same functionality as described in~(\ref{eq:fc}), where the input feature $\bm{X}\in\mathbb{R}^{d_1 \times d_2 \times d_3}$ goes through an affine transformation first and then an element-wise nonlinear activation. The difference lies in the specific form of the affine transformation. A convolutional layer uses a number of \emph{filters} to extract local features from the previous input. More precisely, each filter is represented by a 3D tensor $\bm{F}_{k}\in\mathbb{R}^{w\times w\times d_{3}}$ ($1\leq k\leq \tilde d_3$), where $w$ is the size of the filter (typically 3 or 5) and $\tilde d_3$ denotes the total number of filters. Note that the third dimension $d_3$ of $\bm{F}_{k}$ is equal to that of the input feature $\bm{X}$. For this reason, one usually says that the filter has size $w \times w$, while suppressing the third dimension $d_3$. Each filter $\bm{F}_{k}$ then convolves with the input feature $\bm{X}$ to obtain one single feature map $\bm{O}^{k} \in \mathbb{R}^{(d_1 - w +1) \times (d_1 - w +1)} $, where\footnote{To simplify notation, we omit the bias/intercept term associated with each filter.}
\begin{equation}\label{eq:conv}
O^{k}_{ij}= \big\langle \left[\bm{X}\right]_{ij}, \bm{F}_{k} \big\rangle = \sum_{i'=1}^w \sum_{j'=1}^w \sum_{l=1}^{d_3} [\bm{X}]_{i+i'-1, j+j'-1, l} [\bm{F}_{k}]_{i',j',l}.
\end{equation}
Here $[\bm{X}]_{ij}\in\mathbb{R}^{w\times w\times d_{3}}$ is a small ``patch'' of $\bm{X}$ starting at location $(i,j)$. See Figure \ref{fig:Convolution-operation}
for an illustration of the convolution operation. If we view the 3D tensors $[\bm{X}]_{ij}$ and $\bm{F}_{k}$ as vectors, then each filter essentially computes their inner product with a part of $\bm{X}$ indexed by $i,j$ (which can be also viewed as convolution, as its name suggests). One then pack the resulted feature maps $\{\bm{O}^{k}\}$ into a 3D tensor $\bm{O}$ with size $(d_1 - w +1) \times (d_1 - w +1) \times \tilde d_3$, where
\begin{equation}
[\bm{O}]_{ijk} = [\bm{O}^{k}]_{ij}.
\end{equation}
\begin{figure}
\centering

\includegraphics[height=0.4\textwidth]{convolution_3D}

\caption{$\bm{X}\in \mathbb{R}^{28\times 28 \times 3}$ represents the input feature consisting of $28 \times 28$ spatial coordinates in a total number of 3  channels / feature maps. $\bm{F}_{k}\in\mathbb{R}^{5\times 5 \times 3}$ denotes the $k$-th filter with size $5\times 5$. The third dimension $3$ of the filter automatically matches the number $3$  of channels in the previous input. Every 3D patch of $\bm{X}$ gets convolved with the filter $\bm{F}_{k}$ and this as a whole results in a single output feature map $\tilde{X}_{:,:,k}$ with size $24\times 24\times 1$. Stacking the outputs of all the filters $\{\bm{F}_{k}\}_{1\leq k\leq K}$ will lead to the output feature with size $24\times 24\times K$. \label{fig:Convolution-operation}}

\end{figure}
The outputs of convolutional layers are then followed by nonlinear activation functions. In the ReLU case, we have
\begin{equation}\label{eq:relu}
\tilde{X}_{ijk} = \sigma(O_{ijk}), \qquad \forall\, i \in [d_1-w+1], j \in [d_2-w+1], k \in [\tilde d_3].
\end{equation}
The convolution operation \eqref{eq:conv} and the ReLU activation \eqref{eq:relu} work together to extract features $\tilde{\bm{X}}$ from the input $\bm{X}$. %, which functions in a way similar to the feedforward neural net \eqref{eq:fc}.
Different from feed-forward neural nets, the filters $\bm{F}_k$ are shared across all locations $(i,j)$. A patch $[\bm{X}]_{ij}$ of an input responds strongly (that is, producing a large value) to a filter $\bm{F}_{k}$ if they are positively correlated. Therefore intuitively, each filter $\bm{F}_{k}$ serves to extract features similar to $\bm{F}_{k}$.

As a side note, after the convolution~(\ref{eq:conv}), the spatial size $d_1 \times d_2$ of the input $\bm{X}$ shrinks to ${(d_1-w+1)\times (d_2-w+1)}$ of $\tilde{\bm{X}}$. However one may want the spatial size unchanged. This can be achieved via \emph{padding}, where one appends zeros to the margins of the input $\bm{X}$ to enlarge the spatial size to $(d_1+w-1) \times (d_2+w-1)$. %The case when we apply padding to leave the spatial size unchanged is often called \emph{same convolution}, while the case with no padding is dubbed as \emph{valid convolution}. 
In addition, a \emph{stride} in the convolutional layer determines the gap $i' - i$ and $j'-j$ between two patches $\bm{X}_{ij}$ and $\bm{X}_{i'j'}$: in \eqref{eq:conv} the stride is $1$, and a larger stride would lead to feature maps with smaller sizes.

\item \emph{Pooling layer (POOL)}. A pooling layer aggregates the information of nearby features into a single one. This downsampling operation reduces the size of the features for subsequent layers and saves computation. One common form of the pooling layer is composed of the $2 \times 2$ max-pooling filter. It computes $\max \{X_{i,j,k}, X_{i+1,j,k}, X_{i,j+1,k}, X_{i+1,j+1,k} \}$, that is, the maximum of the $2 \times 2$ neighborhood in the spatial coordinates; see Figure \ref{fig:pooling} for an illustration. Note that the pooling operation is done separately for each feature map $k$. As a consequence, a $2 \times 2$ max-pooling filter acting on $\bm{X}\in \mathbb{R}^{d_1 \times d_2 \times d_3}$ will result in an output of size $d_1/2 \times d_2/2 \times d_3$. In addition, the pooling layer does not involve any parameters to optimize. Pooling layers serve to reduce redundancy since a small neighborhood around a location $(i,j)$ in a feature map is likely to contain the same information.
\end{enumerate}

\begin{figure}
\centering

\includegraphics[width=0.95 \linewidth]{pooling}\caption{A $2\times 2$ max pooling layer extracts the maximum of 2 by 2 neighboring pixels$\,$/$\,$features across the spatial dimension. }\label{fig:pooling}
\end{figure}

In addition, we also use fully-connected layers as building blocks, which we have already seen in Section~\ref{sec:super}. Each fully-connected layer treats input tensor $\bm{X}$ as a vector $\vect(\bm{X})$, and computes $\tilde{\bm{X}} = \bsigma(\bW \vect(\bm{X}))$. A fully-connected layer does not use weight sharing and is often used in the last few layers of a CNN. As an example, Figure~\ref{fig:CNN} depicts the well-known LeNet 5 \citep{lecun1998gradient}, which is composed of two sets of CONV-POOL layers and three fully-connected layers.

%Now we are ready to build a convolutional neural network, which is simply a stack of convolutional layers, pooling layers and fully-connected layers. See 

%\begin{python}
%from keras.datasets import mnist
%from keras.models import Sequential
%from keras.layers import Dense, Flatten
%from keras.layers import Conv2D, MaxPooling2D
%
%# input image dimensions
%img_rows, img_cols = 28, 28
%
%# the data, split between train and test sets
%(x_train, y_train), (x_test, y_test) = mnist.load_data()
%x_train = x_train.reshape(x_train.shape[0], img_rows, img_cols, 1)
%x_test = x_test.reshape(x_test.shape[0], img_rows, img_cols, 1)
%input_shape = (img_rows, img_cols, 1)
%
%x_train = x_train.astype('float32')
%x_test = x_test.astype('float32')
%x_train, x_test = x_train / 255.0, x_test / 255.0
%
%LeNet = Sequential([
%    Conv2D(6, kernel_size=(5, 5), activation='relu',
%    	 padding='same', input_shape=input_shape),
%    MaxPooling2D(pool_size=(2, 2)),
%    Conv2D(16, (5, 5), activation='relu'),
%    MaxPooling2D(pool_size=(2, 2)),
%    Flatten(),
%    Dense(120, activation='relu'),
%    Dense(84, activation='relu'),
%    Dense(10, activation='softmax')]
%)
%
%LeNet.compile(loss='sparse_categorical_crossentropy',
%	optimizer='sgd', metrics=['accuracy'])
%
%history = LeNet.fit(x_train, y_train, batch_size=32, epochs=5, validation_split=0.1)
%
%score = model.evaluate(x_test, y_test, verbose=0)
%
%print('Test loss:', score[0])
%print('Test accuracy:', score[1])
%\end{python}

\begin{figure}
\centering
\includegraphics[width=0.9 \textwidth]{LeNet}\caption{LeNet is composed of an input layer, two convolutional layers, two pooling layers and three fully-connected layers. Both convolutions are valid and use filters with size $5 \times 5$. In addition, the two pooling layers use $2 \times 2$ average pooling. \label{fig:CNN}}
\end{figure}


\subsection{Recurrent neural networks}\label{sec:RNN}
%\cm{Mention that RNN is cyclic graph.}
Recurrent neural nets (RNNs) are another family of powerful models, which are designed to process time series data and other sequence data. RNNs have successful applications in speech recognition \citep{sak2014long}, machine translation \citep{wu2016google}, genome sequencing \citep{cao2018deep}, etc. The structure of an RNN naturally forms a computational graph, and can be easily combined with other structures such as CNNs to build large computational graph models for complex tasks. Here we introduce vanilla RNNs and improved variants such as long short-term memory (LSTM).

%\begin{figure}
%\centering
%\includegraphics[width = 0.65\textwidth]{Figure/RNN1}
%\caption{A vanilla RNN with one hidden layer. Top row is outputs, middle row is hidden states, and bottom row is inputs. Parameters are shared across time.} \label{fig:RNN1}
%\end{figure}
%\begin{figure}\label{fig:RNN1}
%\centering
%\begin{subfigure}[b]{0.3\textwidth}
%\centering
%        \includegraphics[width = 0.95\textwidth]{Figure/RNN1-1}
%\end{subfigure}
%\begin{subfigure}[b]{0.3\textwidth}
%\centering
%        \includegraphics[width = 0.95\textwidth]{Figure/RNN1-2}
%\end{subfigure}
%\begin{subfigure}[b]{0.3\textwidth}
%\centering
%        \includegraphics[width = 0.95\textwidth]{Figure/RNN1-3}
%\end{subfigure}
%\caption{Vanilla RNNs with different inputs/outputs settings. \textbf{Left} has one input but multiple outputs. \textbf{Middle} has multiple inputs but one output. \textbf{Right} has multiple inputs and outputs. Parameters are shared across time.}
%\end{figure}

\begin{figure}
\centering
\begin{tabular}{ccc}
\includegraphics[width = 0.3\textwidth]{RNN1-1} & \includegraphics[width = 0.3\textwidth]{RNN1-2} & \includegraphics[width = 0.3\textwidth]{RNN1-3} \tabularnewline
(a) One-to-many & (b) Many-to-one & (c) Many-to-many
\end{tabular}
\caption{Vanilla RNNs with different inputs/outputs settings. (a) has one input but multiple outputs; (b) has multiple inputs but one output; (c) has multiple inputs and outputs. Note that the parameters are shared across time steps.}\label{fig:RNN1}
\end{figure}

\subsubsection{Vanilla RNNs}
Suppose we have general time series inputs $\xx_1,\xx_2,\ldots,\xx_T$. A vanilla RNN models the ``hidden state'' at time $t$ by a vector $\hh_t$, which is subject to the recursive formula
\begin{equation}\label{eq:recur}
\hh_t  = \ff_{\btheta}(\hh_{t-1}, \xx_t).
\end{equation}
Here, $f_{\btheta}$ is generally a nonlinear function parametrized by $\btheta$. Concretely, a vanilla RNN with one hidden layer has the following form\footnote{Similar to the activation function $\bsigma(\cdot)$, the function $\btanh(\cdot)$ means element-wise operations.}
\begin{align*}
\hh_t  &= \btanh\left( \bW_{hh} \hh_{t-1}+ \bW_{xh} \xx_t + \bb_\hh \right), \qquad \text{where}~ \tanh(a) = \tfrac{e^{2a} - 1}{e^{2a} + 1}, \\
\zz_t &= \bsigma \left(\bW_{hy} \hh_t + \bb_\zz \right),
\end{align*}
where $\bW_{hh}, \bW_{xh}, \bW_{hy}$ are trainable weight matrices, $\bb_\hh, \bb_\zz$ are trainable bias vectors, and $\zz_t$ is the output at time $t$. Like many classical time series models, those parameters are shared across time. Note that in different applications, we may have different input/output settings (cf.~Figure~\ref{fig:RNN1}). Examples include
\begin{itemize}
\item{ \textbf{One-to-many:} a single input with multiple outputs; see Figure~\ref{fig:RNN1}(a). A typical application is image captioning, where the input is an image and outputs are a series of words.
}
\item{ \textbf{Many-to-one:} multiple inputs with a single output; see Figure~\ref{fig:RNN1}(b). One application is text sentiment classification, where the input is a series of words in a sentence and the output is a label (e.g., positive vs.~negative).
}
\item{ \textbf{Many-to-many:} multiple inputs and outputs; see Figure~\ref{fig:RNN1}(c). This is adopted in machine translation, where inputs are words of a source language (say Chinese) and outputs are words of a target language (say English).
}
\end{itemize}

As the case with feed-forward neural nets, we minimize a loss function using back-propagation, where the loss is typically
\begin{equation*}
\ell_{\cT}(\btheta) = \sum_{t \in \cT} \cL(y_t, \zz_t) = - \sum_{t \in \cT} \sum_{k=1}^K \bbone\{y_t = k\} \log \left( \frac{\exp([\zz_t]_k)}{\sum_k \exp([\zz_t]_k)} \right),
\end{equation*}
where $K$ is the number of categories for classification (e.g., size of the vocabulary in machine translation), and $\cT \subset [T]$ is the length of the output sequence. During the training, the gradients $\partial \ell_{\cT} / \partial \hh_t$ are computed in the reverse time order (from $T$ to $t$). For this reason, the training process is often called \textit{back-propagation through time}. 

One notable drawback of vanilla RNNs is that, they have difficulty in capturing long-range dependencies in sequence data when the length of the sequence is large. This is sometimes due to the phenomenon of \emph{exploding$\,$/$\,$vanishing gradients}. Take Figure~\ref{fig:RNN1}(c) as an example. Computing $\partial \ell_{\cT} / \partial \hh_1$ involves the product $\prod_{t=1}^3 (\partial \hh_{t+1} / \partial \hh_{t})$ by the chain rule. However, if the sequence is long, the product will be the multiplication of many Jacobian matrices, which usually results in exponentially large or small singular values. To alleviate this issue, in practice, the forward pass and backward pass are implemented in a shorter sliding window $\{t_1, t_1+1, \ldots,t_2\}$, instead of the full sequence $\{1,2,\ldots, T\}$. Though effective in some cases, this technique alone does not fully address the issue of long-term dependency. 

\subsubsection{GRUs and LSTM} There are two improved variants that alleviate the above issue: gated recurrent units (GRUs) \citep{cho2014learning} and long short-term memory (LSTM) \citep{hochreiter1997long}.
\begin{itemize}
\item{ A \textbf{GRU} refines the recursive formula \eqref{eq:recur} by introducing \textit{gates}, which are vectors of the same length as $\hh_t$. The gates, which take values in $[0,1]$ elementwise, multiply with $\hh_{t-1}$ elementwise and determine how much they keep the old hidden states.
}
\item{ An \textbf{LSTM} similarly uses gates in the recursive formula. In addition to $\hh_t$, an LSTM maintains a \textit{cell state}, which takes values in $\mathbb{R}$ elementwise and are analogous to counters.
}
\end{itemize}
Here we only discuss LSTM in detail. Denote by $\odot$ the element-wise multiplication. We have a recursive formula in replace of \eqref{eq:recur}:
\begin{align*}
\left( \begin{array}{c} \ii_t \\ \ff_t \\ \oo_t \\ \bgg_t \end{array} \right) &= \left( \begin{array}{c} \bsigma \\ \bsigma \\ \bsigma \\ \btanh \end{array} \right) \bW \left( \begin{array}{c} \hh_{t-1} \\ \xx_t \\ 1 \end{array} \right), \\
\cc_t &= \ff_t \odot \cc_{t-1} + \ii_t \odot \bgg_t, \\
\hh_t &= \oo_t \odot \btanh(\cc_t),
\end{align*}
where $\bW$ is a big weight matrix with appropriate dimensions. The cell state vector $\cc_t$ carries information of the sequence (e.g., singular/plural form in a sentence). The forget gate $\ff_t$ determines how much the values of $\cc_{t-1}$ are kept for time $t$, the input gate $\ii_t$ controls the amount of update to the cell state, and the output gate $\oo_t$ gives how much $\cc_t$ reveals to $\hh_t$.  Ideally, the elements of these gates have nearly binary values. For example, an element of $\ff_t$ being close to $1$ may suggest the presence of a feature in the sequence data. Similar to the skip connections in residual nets, the cell state $\cc_t$ has an additive recursive formula, which helps back-propagation and thus captures long-range dependencies.

\begin{figure}
\centering
\includegraphics[width = 0.4\textwidth]{RNN2}
\caption{A vanilla RNN with two hidden layers. Higher-level hidden states $\hh_t^{\ell}$ are determined by the old states $\hh_{t-1}^\ell$ and lower-level hidden states $\hh_t^{\ell-1}$. Multilayer RNNs generalize both feed-forward neural nets and one-hidden-layer RNNs.}\label{fig:RNN2}
\end{figure}

\subsubsection{Multilayer RNNs} Multilayer RNNs are generalization of the one-hidden-layer RNN discussed above. Figure~\ref{fig:RNN2} shows a vanilla RNN with two hidden layers. In place of \eqref{eq:recur}, the recursive formula for an RNN with $L$ hidden layers now reads
\begin{equation*}
\hh_t^{\ell} =  \btanh \left[\bW^\ell \left( \begin{array}{c} \hh_t^{\ell-1} \\ \hh_{t-1}^\ell \\ 1 \end{array} \right) \right], \quad \text{for all}\, \ell \in [L], \qquad \hh_t^{0} \triangleq \xx_t.
\end{equation*}
Note that a multilayer RNN has two dimensions: the sequence length $T$ and depth $L$. Two special cases are the feed-forward neural nets (where $T=1$) introduced in Section~\ref{sec:super}, and RNNs with one hidden layer (where $L=1$). Multilayer RNNs usually do not have very large depth (e.g., $2$--$5$), since $T$ is already very large.

Finally, we remark that CNNs, RNNs, and other neural nets can be easily combined to tackle tasks that involve different sources of input data. For example, in image captioning, the images are first processed through a CNN, and then the high-level features are fed into an RNN as inputs. Theses neural nets combined together form a large computational graph, so they can be trained using back-propagation. This generic training method provides much flexibility in various applications.

%\subsection{Modules and skip connections}\label{sec:skip}
\subsection{Modules}\label{sec:skip}

Deep neural nets are essentially composition of many nonlinear functions. A component function may be designed to have specific properties in a given task, and it can be itself resulted from composing a few simpler functions. In LSTM, we have seen that the building block consists of several intermediate variables, including cell states and forget gates that can capture long-term dependency and alleviate numerical issues. 

This leads to the idea of designing \textit{modules} for building more complex neural net models. Desirable modules usually have low computational costs, alleviate numerical issues in training, and lead to good statistical accuracy. Since modules and the resulting neural net models form computational graphs, training follows the same principle briefly described in Section~\ref{sec:super}.

Here, we use the examples of \emph{Inception} and \emph{skip connections} to illustrate the ideas behind modules. 
Figure~\ref{fig:skip}(a) is an example of ``Inception'' modules used in GoogleNet~\citep{szegedy2015going}. As before, all the convolutional layers are followed by the ReLU activation function. The concatenation of information from filters with different sizes give the model great flexibility to capture spatial information. Note that $1 \times 1$ filters is an $1 \times 1 \times d_3$ tensor (where $d_3$ is the number of feature maps), so its convolutional operation does not interact with other spatial coordinates, only serving to aggregate information from different feature maps at the same coordinate. This reduces the number of parameters and speeds up the computation. Similar ideas appear in other work \citep{lin2013network, iandola2016squeezenet}.

%\begin{figure}[H]\label{fig:skip}
%    \centering
%    \begin{subfigure}[b]{0.5\textwidth}
%	\centering
%        \includegraphics[scale = 0.5]{Figure/inception}
%	\caption{The ``Inception'' module from GoogleNet. \texttt{Concat} means combining all features maps into a tensor. }
%    \end{subfigure}
%	~
%    \begin{subfigure}[b]{0.4\textwidth}
%	\centering
%        \includegraphics[scale = 0.5]{Figure/resnet2}
%	\caption{Skip connections are added every two layers in ResNets. }
%    \end{subfigure}
%    %\caption{Pictures of animals}\label{fig:animals}
%\end{figure}

\begin{figure}[htb!]
\centering
\begin{tabular}{cc}
\includegraphics[scale = 0.5]{inception} & \includegraphics[scale = 0.5]{resnet2} \tabularnewline
(a) ``Inception'' module & (b) Skip connections
\end{tabular}
\caption{(a) The ``Inception'' module from GoogleNet. \texttt{Concat} means combining all features maps into a tensor. (b) Skip connections are added every two layers in ResNets. }\label{fig:skip}
\end{figure}



Another module, usually called \textit{skip connections}, is widely used to alleviate numerical issues in very deep neural nets, with additional benefits in optimization efficiency and statistical accuracy. Training very deep neural nets are generally more difficult, but the introduction of skip connections in \emph{residual networks} \citep{he2016deep, he2016identity} has greatly eased the task. 

The high level idea of skip connections is to add an identity map to an existing nonlinear function. Let $\bF(\xx)$ be an arbitrary nonlinear function represented by a (fragment of) neural net, then the idea of skip connections is simply replacing $\bF(\xx)$ with $\xx + \bF(\xx)$. Figure~\ref{fig:skip}(b) shows a well-known structure from residual networks \citep{he2016deep}---for every two layers, an identity map is added:
\begin{equation}\label{eq:mapsto}
\xx \longmapsto \bsigma(\xx + \bF(\xx)) = \bsigma(\xx + \bW' \bsigma(\bW \xx + \bb) + \bb'),
\end{equation}
where $\xx$ can be hidden nodes from any layer and $\bW, \bW', \bb, \bb'$ are corresponding parameters. By repeating (namely composing) this structure throughout all layers, \cite{he2016deep, he2016identity} are able to train neural nets with hundreds of layers easily, which overcomes well-observed training difficulties in deep neural nets. Moreover, deep residual networks also improve statistical accuracy, as the classification error on ImageNet challenge was reduced by $46\%$ from 2014 to 2015. As a side note, skip connections can be used flexibly. %, and they can be combined with other ideas and modules to form complex deep neural nets.
They are not restricted to the form in \eqref{eq:mapsto}, and can be used between any pair of layers $\ell, \ell'$ \citep{Huang17}. 

%\subsection{Note}
%Convolutional neural networks can be traced back to the \emph{Neocognitron} model \citep{fukushima1982neocognitron}, where later it was combined with gradient-based training for classification \citep{lecun1998gradient}. The revolved interest in CNN in this century began with the success of AlexNet, a specific convolutional neural network, in ImageNet Challenge \citep{krizhevsky2012imagenet}. VGGNet \citep{simonyan2014very} greatly simplifies the structure of CNN and made a first step towards deeper CNN (VGGNet has 19 layers, compared with 8 layers in AlexNet). In 2016, He et.~al. \citep{he2016deep} proposed a new convolutional neural network, called \emph{ResNet}, where it adds \emph{skip connections} every two convolutional layers. This ``small'' change of structure enables the training of very deep neural nets (152 layers therein) and achieves state-of-the-art results in several standard image processing tasks. 


\section{Deep unsupervised learning}\label{sec:unsup}
%Previous two sections are devoted to supervised learning, where one is given a labeled dataset $\{(\bm{x}_{i},y_{i})\}_{1\leq i\leq n}$ and wish to learn a mapping from the input $\bm{x}$ to the label $y$.


In supervised learning, given labelled training set $\{(y_i,\bx_i)\}$, we focus on discriminative models, which essentially represents $\P(y\,|\,\xx)$ by a deep neural net $f(\xx; \btheta)$ with parameters $\btheta$. Unsupervised learning, in contrast, aims at extracting \emph{information} from \emph{unlabeled} data $\{\bm{x}_{i}\}$, where the labels $\{y_i\}$ are absent. In regard to this information, it can be a low-dimensional embedding of the data $\{ \xx_i \}$ or a generative model with latent variables to approximate the distribution $\P_{\bX}(\xx)$. To achieve these goals, we introduce two popular unsupervised deep leaning models, namely, autoencoders and generative
adversarial networks~(GANs). The first one can be viewed as a dimension reduction technique, and the second as a density estimation method. DNNs are the key elements for both of these two models. %To this end, here we will need two deep neural nets to represent maps between $\xx$ and its low-dimensional representation $\zz$: encoder/decoder in autoencoders, and discriminator/generator in GANs.


\subsection{Autoencoders}
Recall that in dimension reduction, the goal is to reduce the dimensionality of the data and at the same time preserve its salient features. In particular, in principal component analysis (PCA), the goal is to embed the data $\{\bm{x}_{i}\}_{1\leq i\leq n}$ into a low-dimensional space via a linear function $\ff$ such that maximum variance can be explained. Equivalently, we want to find linear functions $\ff: \R^d \to \R^k$ and $\bgg: \R^k \to \R^d$ ($k \le d$) such that the difference between $\xx_i$ and $\bgg(\ff(\xx_i))$ is minimized. Formally, we let
\[
\ff\left(\bm{x}\right)=\bm{W}_f\bm{x}\triangleq \hh \quad\text{and}\quad \bgg\left(\bm{h}\right)=\bm{W}_g\bm{h}, \quad \text{where}\quad\bm{W}_f\in\mathbb{R}^{k\times d}\text{ and } \bm{W}_g\in\mathbb{R}^{d\times k}.
\]
Here, for simplicity, we assume that the intercept/bias terms for $\ff$ and $\bgg$ are zero. Then, PCA amounts to minimizing the quadratic loss function
\begin{equation}
\text{minimize}_{\bm{W}_f, \bm{W}_g}\qquad \frac{1}{n} \sum_{i=1}^{n}\left\Vert \bm{x}_i-\bm{W}_f\bm{W}_g\bm{x}_i\right\Vert _{2}^{2}.\label{eq:linear-AE}
\end{equation}
It is the same as minimizing $\| \bX - \bW \bX \|_{\mathrm{F}}^2$ subject to $\rank(\bW) \le k$, where $\bX\in \mathbb{R}^{p\times n}$ is the design matrix. The solution is given by the singular value decomposition of $\bX$ \citep[Thm.~2.4.8]{golub2013matrix}, which is exactly what PCA does. It turns out that PCA is a special case of autoencoders, which is often known as the \textit{undercomplete linear autoencoder}.

More broadly, autoencoders are neural network models for (nonlinear) dimension reduction, which generalize PCA. An autoencoder has two key components, namely, the encoder function $\ff(\cdot)$, which maps the input $\bm{x}\in\mathbb{R}^{d}$ to a hidden code/representation $\bm{h}\triangleq\ff(\bm{x})\in\mathbb{R}^{k}$, and the decoder function $\bgg(\cdot)$, which maps the hidden representation $\bm{h}$ to a point $\bgg(\bm{h})\in\mathbb{R}^{d}$. Both functions can be multilayer neural networks as \eqref{eq:fc}. See Figure~\ref{fig:AE} for an illustration of autoencoders. Let $\mathcal{L}(\bm{x}_{1},\bm{x}_{2})$ be a loss function that measures the difference between $\bm{x}_{1}$ and $\bm{x}_{2}$ in $\R^d$. Similar to PCA, an autoencoder is used to find the encoder $\ff$ and decoder $\bgg$ such that $\mathcal{L}(\bm{x},\bgg(\ff(\bm{x})))$
is as small as possible. Mathematically, this amounts to solving the following minimization problem
\begin{equation}
\mbox{minimize}_{\ff,\bgg} \quad\frac{1}{n}\sum_{i=1}^{n}\mathcal{L}\left(\bm{x}_{i},\bgg\left(\bm{h}_{i}\right)\right) \quad \text{with}\quad\bm{h}_{i}=\ff\left(\bm{x}_{i}\right), \quad \text{for all }\, i \in [n].  \label{eq:AE}
\end{equation}
\begin{figure}
\centering\includegraphics[scale=0.3]{AE}
\caption{First an input $\bm{x}$ goes through the decoder $\ff(\cdot)$, and we obtain its hidden representation $\bm{h}= \ff(\bm{x})$. Then, we use the decoder $\bgg(\cdot)$ to get $\bgg(\bm{h})$ as a reconstruction of $\bm{x}$. Finally, the loss is determined from the difference between the original input $\bm{x}$ and its reconstruction $\bgg(\ff(\bm{x}))$.}\label{fig:AE} %\TODO{give more than one layer}
\end{figure}

One needs to make structural assumptions on the functions $\ff$ and $\bg$ in order to find useful representations of the data, which leads to different types of autoencoders. Indeed, if no assumption is made, choosing $\ff$ and $\bgg$ to be identity functions clearly minimizes the above optimization problem. To avoid this trivial solution, one natural way is to require that the encoder $f$ maps the data onto a space with a smaller dimension, i.e., $k < d$. This is the \textit{undercomplete autoencoder} that includes PCA as a special case. There are other structured autoencoders which add desired properties to the model such as sparsity or robustness, mainly through regularization terms. Below we present two other common types of autoencoders.

\begin{itemize}
\item \emph{Sparse autoencoders. } One may believe that the dimension $k$ of the hidden code $\hh_i$ is larger than the input dimension $d$, and that $\hh_i$ admits a sparse representation. As with LASSO \citep{tibshirani1996regression} or SCAD \citep{fan2001variable}, one may add a regularization term to the reconstruction loss $\mathcal{L}$ in \eqref{eq:AE} to encourage sparsity \citep{poultney2007efficient}. A sparse autoencoder solves
\begin{equation*}
\mbox{min}_{\ff,\bgg}  \; \underbrace{ \frac{1}{n}\sum_{i=1}^{n}\mathcal{L}\left(\bm{x}_{i},\bgg\left(\bm{h}_{i}\right)\right) }_{\text{loss}}+ \underbrace{\vphantom{\frac{1}{n}\sum_{i=1}^{n}\mathcal{L}\left(\bm{x}_{i},\bgg\left(\bm{h}_{i}\right)\right)} \lambda\left\Vert \bm{h}_{i}\right\Vert _{1} }_{\text{regularizer}} \quad \text{with} \quad \bm{h}_{i}=\ff\left(\bm{x}_{i}\right), \text{ for all } i \in [n].
\end{equation*}
This is similar to \textit{dictionary learning}, where one aims at finding a sparse representation of input data on an overcomplete basis. Due to the imposed sparsity, the model can potentially learn useful features of the data.

\item \emph{Denoising autoencoders. } One may hope that the model is robust to noise in the data: even if the input data $\bm{x}_i$ are corrupted by small noise $\bxi_i$ or miss some components (the noise level or the missing probability is typically small), an ideal autoencoder should faithfully recover the original data. A denoising autoencoder \citep{vincent2008extracting} achieves this robustness by explicitly building a noisy data $\tilde{\bm{x}}_{i} = \bm{x}_i + \bxi_i$ as the new input, and then solves an optimization problem similar to \eqref{eq:AE} where $\mathcal{L}\left(\bm{x}_{i},\bgg\left(\bm{h}_{i}\right)\right)$ is replaced by $\mathcal{L}\left(\bm{x}_{i},\bgg\left(\ff(\tilde{\bm{x}}_{i})\right)\right)$. A denoising autoencoder encourages the encoder/decoder to be stable in the neighborhood of an input, which is generally a good statistical property. An alternative way could be constraining $f$ and $g$ in the optimization problem, but that would be very difficult to optimize. Instead, sampling by adding small perturbations in the input provides a simple implementation. We shall see similar ideas in Section~\ref{sec:aug}.
    %{\bf Q: where is the regularization: noisy + unnoisy data?}

%\item \emph{Variational autoencoders. } Just as PCA has a probabilistic (or generative) underpinning, namely factor models, autoencoders \eqref{eq:AE} can also be cast in the language of latent variable models. Let the latent variables $\hh$ be $\mathcal{N}(\bzero, \bI_k)$, and suppose the conditional probability $p_{\btheta}(\xx | \hh)$ is associated with a decoder neural network with parameters $\btheta$; for example, $\xx = \bg_{\btheta}(\hh) + \sigma \cdot \mathcal{N}(\bzero, \bI_d)$. To find the MLE for $\btheta$, one difficulty is to compute $p_{\btheta}(\xx)$, which involves complicated integral after expanding this probability in term of $\hh$. The idea of variational autoencoders is to use an encoder neural network $f_{\bvarphi}$ to approximate the posterior $p_{\btheta}(\hh | \xx)$. See \cite{doersch2016tutorial} for details of derivation and examples. Similar to the aforementioned autoencoders, the final optimization formulation of variational autoencoders involves a loss term that represents the reconstruction error, and a Kullback-Leibler divergence that serves as a regularizer.

\end{itemize}

\subsection{Generative adversarial networks}

Given unlabeled data $\{\bm{x}_{i}\}_{1\leq i\leq n}$, density estimation
aims to estimate the underlying probability density function $\mathbb{P}_{\bm{X}}$
from which the data is generated. Both parametric and nonparametric
estimators \citep{silverman2018density} have been proposed and studied under various assumptions
on the underlying distribution. Different from these classical density estimators, where the density function is explicitly defined in relatively low dimension, generative adversarial networks (GANs) \citep{goodfellow2014generative} can be categorized as an \emph{implicit} density estimator in much higher dimension. The reasons are twofold: (1) GANs put more emphasis on sampling from
the distribution $\mathbb{P}_{\bm{X}}$ than estimation; (2) GANs define the density estimation implicitly through a source distribution $\mathbb{P}_{\bm{Z}}$ and a generator function $g(\cdot)$, which is usually a deep neural network. We introduce GANs from the perspective of sampling from $\mathbb{P}_{\bm{X}}$ and later we will generalize the vanilla GANs using its relation to density estimators.

\subsubsection{Sampling view of GANs}
Suppose the data $\{\bm{x}_{i}\}_{1\leq i\leq n}$ at hand are all real images, and we want to generate \emph{new} natural images.
With this goal in mind, GAN models a \emph{zero-sum} game between two players, namely,
the generator $\mathcal{G}$ and the discriminator $\mathcal{D}$. The
generator $\mathcal{G}$ tries to generate fake images akin to the
true images $\{\bm{x}_{i}\}_{1\leq i\leq n}$ while the discriminator
$\mathcal{D}$ aims at differentiating the fake ones from
the true ones. Intuitively, one hopes to learn a generator $\mathcal{G}$ to generate images where the \emph{best} discriminator $\mathcal{D}$ cannot distinguish. Therefore the payoff is higher for the generator~$\mathcal{G}$ if the probability of the discriminator $\mathcal{D}$
getting wrong is higher, and correspondingly the payoff for the discriminator
correlates positively with its ability to tell wrong from truth.

Mathematically, the generator $\mathcal{G}$ consists of two components,
an source distribution $\mathbb{P}_{\bm{Z}}$ (usually a standard multivariate Gaussian distribution with hundreds of dimensions) and a function $\bg(\cdot)$ which maps a sample
$\bm{z}$ from $\mathbb{P}_{\bm{Z}}$ to a point $\bg(\bm{z})$ living
in the same space as $\bm{x}$. For generating images, $\bg(\bm{z})$ would be a 3D tensor. Here $\bg(\bm{z})$ is the fake sample
generated from $\mathcal{G}$. Similarly the discriminator $\mathcal{D}$
is composed of one function which takes an image ${\bm{x}}$ (real or fake)
and return a number $d({\bm{x}})\in[0,1]$, the probability
of ${\bm{x}}$ being a real sample from $\mathbb{P}_{\bm{X}}$ or not.
Oftentimes, both the generating function $\bg(\cdot)$ and the discriminating
function $d(\cdot)$ are realized by deep neural networks, e.g., CNNs introduced in Section~\ref{sec:CNN}. See Figure~\ref{fig:GAN} for an illustration
for GANs. Denote $\btheta_{\mathcal{G}}$ and $\btheta_{\mathcal{D}}$
the parameters in $\bg(\cdot)$ and $d(\cdot)$, respectively. Then
GAN tries to solve the following \emph{min-max }problem:
\begin{figure}
\centering\includegraphics[width=0.9\textwidth]{GAN}
\caption{GANs consist of two components, a generator $\mathcal{G}$ which generates fake samples and a discriminator $\mathcal{D}$ which differentiate the true ones from the fake ones. \label{fig:GAN}}
\end{figure}
\begin{equation}
\min_{\btheta_{\mathcal{G}}}\max_{\btheta_{\mathcal{D}}}
\qquad\mathbb{E}_{\bm{x}\sim\mathbb{P}_{\bm{X}}}\left[\log
\left(d\left(\bm{x}\right)\right)\right]
+\mathbb{E}_{\bm{z}\sim\mathbb{P}_{\bm{Z}}}\left[\log\left(1-d\left(\bg\left(\bm{z}\right)\right)\right)\right].\label{eq:GAN-original}
\end{equation}
Recall that $d(\bm{x})$ models the belief / probability that the
discriminator thinks that $\bm{x}$ is a true sample. Fix the parameters
$\btheta_{\mathcal{G}}$ and hence the generator $\mathcal{G}$ and
consider the inner maximization problem.
We can see that the goal
of the discriminator is to maximize its ability of differentiation.
Similarly, if we fix $\btheta_{\mathcal{D}}$ (and hence the discriminator),
the generator tries to generate more realistic images $\bg(\bm{z})$
to fool the discriminator.

\subsubsection{Density estimation view of GANs}
Let us now take a density-estimation view of GANs. Fixing the source distribution $\mathbb{P}_{\bm{Z}}$, any generator $\mathcal{G}$ induces a distribution $\mathbb{P}_{\mathcal{G}}$ over the space of images. Removing the restrictions on $d(\cdot)$, one can then rewrite (\ref{eq:GAN-original}) as
\begin{equation}\label{eq:GAN-new}
\min_{\mathbb{P}_{\mathcal{G}}}\max_{d(\cdot)}\qquad\mathbb{E}_{\bm{x}\sim\mathbb{P}_{\bm{X}}}\left[\log\left(d\left(\bm{x}\right)\right)\right]+\mathbb{E}_{\bm{x}\sim\mathbb{P}_{\mathcal{G}}}\left[\log\left(1-d\left(\bm{x}\right)\right)\right].
\end{equation}
Observe that the inner maximization problem
is solved by the likelihood ratio,~i.e.
\[
d^{*}\left(\bm{x}\right)=\frac{\mathbb{P}_{\bm{X}}\left(\bm{x}\right)}{\mathbb{P}_{\bm{X}}\left(\bm{x}\right)+\mathbb{P}_{\mathcal{G}}\left(\bm{x}\right)}.
\]
As a result, (\ref{eq:GAN-new}) can be simplified as
\begin{equation}
\min_{\mathbb{P}_{\mathcal{G}}}\qquad\text{JS}\left(\mathbb{P}_{\bm{X}}\;\|\;\mathbb{P}_{\mathcal{G}}\right)\label{eq:JS-min},
\end{equation}
where $\text{JS}(\cdot\|\cdot)$ denotes the Jensen--Shannon divergence
between two distributions
\[
\text{JS}\left(\mathbb{P}_{\bm{X}}\|\mathbb{P}_{\mathcal{G}}\right)=\frac{1}{2}\text{KL}\big(\mathbb{P}_{\bm{X}}\;\|\;\tfrac{\mathbb{P}_{\bm{X}}+\mathbb{P}_{\mathcal{G}}}{2}\big)+\frac{1}{2}\text{KL}\big(\mathbb{P}_{\mathcal{G}}\;\|\;\tfrac{\mathbb{P}_{\bm{X}}+\mathbb{P}_{\mathcal{G}}}{2}\big).
\]
In words, the vanilla GAN (\ref{eq:GAN-original}) seeks a density $\mathbb{P}_{\mathcal{G}}$ that is closest to $\mathbb{P}_{\bm{X}}$ in terms of the Jensen--Shannon divergence. This view allows to generalize GANs to other variants, by changing the distance metric. Examples include f-GAN \citep{nowozin2016f}, Wasserstein GAN (W-GAN) \citep{arjovsky2017wasserstein}, MMD GAN \citep{li2015generative}, etc. We single out the Wasserstein GAN (W-GAN) \citep{arjovsky2017wasserstein} to introduce due to its popularity. As the name suggests, it minimizes
the Wasserstein distance between $\mathbb{P}_{\bm{X}}$ and $\mathbb{P}_{\mathcal{G}}$:
\begin{equation}
\min_{\btheta_{\mathcal{G}}}\quad \text{WS}\left(\mathbb{P}_{\bm{X}}\|\mathbb{P}_{\mathcal{G}}\right)\;\;=\;\;\min_{\btheta_{\mathcal{G}}}\sup_{f:f\text{ 1-Lipschitz}}\mathbb{E}_{\bm{x}\sim\mathbb{P}_{\bm{X}}}
\left[f\left(\bm{x}\right)\right]-\mathbb{E}_{\bm{x}\sim
\mathbb{P}_{\mathcal{G}}}\left[f\left(\bm{x}\right)\right],\label{eq:WS-GAN}
\end{equation}
where $f(\cdot)$ is taken over all Lipschitz functions with coefficient 1.
Comparing W-GAN (\ref{eq:WS-GAN}) with the original formulation of GAN (\ref{eq:GAN-original}), one finds
that the Lipschitz function $f$ in (\ref{eq:WS-GAN}) corresponds
to the discriminator $\mathcal{D}$ in (\ref{eq:GAN-original}) in the sense that they
share similar objectives to differentiate the true distribution
$\mathbb{P}_{\bm{X}}$ from the fake one $\mathbb{P}_{\mathcal{G}}$. In the end, we would like to mention that GANs are more difficult to train than supervised deep learning models such as CNNs~\citep{salimans2016improved}. Apart from the training difficulty, how to evaluate GANs objectively and effectively is an ongoing research.


%%%%%%%%%%%%%%%%
%%%%%%%%%%%%%%%%
\begin{comment}
\subsection{Note}
There are other types of autoencoders. In a contractive autoencoder \citep{rifai2011contractive}, one regularizes the encoder $f(\cdot)$ with a penalty on the gradients of the mapping so as to encourage smoothness. The autoencoders introduced herein can also be stacked together to form deep autoencoders \citep{hinton2006reducing, vincent2010stacked}. Stacked autoencoders was originally used as a pre-training technique to train deep neural networks \citep{hinton2006reducing}. %\TODO{stacked autoencoder is a model, not training technique? mention Boltzmann machine?}
Specifically, we can train the autoencoders layer by layer using only unlabeled data, after which we can discard the decoder component and view the encoders as feed-forward connections in usual neural networks. Connecting the top-layer with a classification layer, we can then fine-tune the parameters (that is, using parameters of the encoder as initial values) using the labels by back-propagation. However, its role in training deep neural nets is diminishing, and recent years have witnessed a trend where deep neural nets are directly trained without resorting to autoencoders as pre-training.

%Many other generative models, such as the restricted Boltzmann machine, predated the popularity of deep neural nets. \TODO{some intro here; due to its historical imporantance.}

GANs, popular for their surprising ability to generate real images, are also notorious for their training difficulty \citep{goodfellow2016nips,radford2015unsupervised,salimans2016improved}. It is still an active research area to develop better network architectures and training algorithms for GANs.
\end{comment}

%%%%%%%%%%%%%%%%
%%%%%%%%%%%%%%%% 
\section{Representation power: approximation theory}\label{sec:approx}
Having seen the building blocks of deep learning models in the previous sections, it is natural to ask: what is the benefits of composing multiple layers of nonlinear functions. In this section, we address this question from a approximation theoretical point of view. Mathematically, letting $\cH$ be the space of functions representable by neural nets (NNs),  how well can a function $f$ (with certain properties) be approximated by functions in $\cH$. We first revisit universal approximation theories, which are mostly developed for shallow neural nets (neural nets with a single hidden layer), and then provide recent results that demonstrate the benefits of depth in neural nets. Other notable works include Kolmogorov-Arnold superposition theorem~\citep{arnold2009functions, sprecher1965structure}, and circuit complexity for neural nets~\citep{parberry1994circuit}.

%Over several decades, efforts from various communities have addressed this question for different types of nets.


\subsection{Universal approximation theory for shallow NNs}
The universal approximation theories study the approximation of $f$ in a space~$\cF$ by a function represented by a one-hidden-layer neural net
\begin{equation}\label{def:HN}
%g(\xx) = \sum_{j=1}^N c_j \sigma_*(\ww_j ^\top \xx - b_j) + c_0,
g(\xx) = \sum_{j=1}^N c_j \sigma_*(\ww_j ^\top \xx - b_j),
\end{equation}
where $\sigma_*: \R \to \R$ is certain activation function and $N$ is the number of hidden units in the neural net. For different space $\cF$ and activation function $\sigma_*$, there are upper bounds and lower bounds on the approximation error $\| f - g \|$. See~\cite{pinkus1999approximation} for a comprehensive overview. Here we present representative results.

First, as $N \to \infty$, any continuous function $f$ can be approximated by some $g$ under mild conditions. Loosely speaking, this is because each component $\sigma_*(\ww_j ^\top \xx - b_j)$ behaves like a basis function %(e.g., Fourier basis, polynomial basis),
and functions in a suitable space $\cF$ admits a basis expansion. Given the above heuristics, the next natural question is: what is the rate of approximation for a finite $N$?

Let us restrict the domain of $\xx$ to a unit ball $B^d$ in $\R^d$. For $p \in [1,\infty)$ and integer $m \ge 1$, consider the $L^p$ space and the Sobolev space with standard norms
\begin{align*}
 \| f \|_p = \Big[ \int_{B^n} | g(\xx) |^p \; d\xx  \Big]^{1/p}, \qquad \| f \|_{m,p} = \Big[ \sum_{0 \le |\kk| \le m} \| D^{\kk} f \|_p^p \Big]^{1/p},
\end{align*}
where $D^{\kk} f$ denotes partial derivatives indexed by $\kk \in \mathbb{Z}_+^d$. Let $\cF \triangleq \cF^m_p$ be the space of functions $f$ in the Sobolev space with $\| f \|_{m,p} \le 1$. Note that functions in~$\cF$ have bounded derivatives up to $m$-th order, and that smoothness of functions is controlled by $m$ (larger $m$ means smoother). Denote by $\cH_N$ the space of functions with the form \eqref{def:HN}. The following general upper bound is due to~\cite{mhaskar1996neural}.
\begin{thm}[Theorem~2.1 in \cite{mhaskar1996neural}]\label{thm:approx1}
Assume $\sigma_*: \R \to \R$ is such that $\sigma_*$ has arbitrary order derivatives in an open interval $I$, and that $\sigma_*$ is not a polynomial on $I$. Then, for any $p \in [1,\infty)$, $d \ge 2$, and integer $m \ge 1$,
\begin{equation*}
\sup_{f \in \cF^m_p} \inf_{g \in \cH_N^{\phantom{a}}} \| f - g \|_p \le C_{d,m,p}\, N^{-m/d},
\end{equation*}
where $C_{d,m,p}$ is independent of $N$, the number of hidden units.
\end{thm}
In the above theorem, the condition on $\sigma_*(\cdot)$ is mainly technical. This upper bound is useful when the dimension $d$ is not large. It clearly implies that the one-hidden-layer neural net is able to approximate any smooth function with enough hidden units. However, it is unclear how to find a good approximator $g$; nor do we have control over the magnitude of the parameters (huge weights are impractical). While increasing the number of hidden units $N$ leads to better approximation, the exponent $-m/d$ suggests the presence of the \emph{curse of dimensionality}. The following (nearly) matching lower bound is %a consequence of~\citep{devore1989optimal}.
stated in~\cite{maiorov2000near}.
\begin{thm}[Theorem~5 in \cite{maiorov2000near}]\label{thm:approx2-2}
Let $p \ge 1$, $m \ge 1$ and $N \ge 2$. If the activation function is the standard sigmoid function $\sigma(t) = (1 + e^{-t})^{-1}$, then
\begin{equation}\label{ineq:approxlower2}
\sup_{f \in \cF^m_p} \inf_{g \in \cH_N^{\phantom{a}}} \| f - g \|_p \ge C'_{d,m,p}\, (N\log N)^{-m/d},
\end{equation}
where $C'_{d,m,p}$ is independent of $N$.
\end{thm}
Results for other activation functions are also obtained by~\cite{maiorov2000near}. Moreover, the term $\log N$ can be removed if we assume an additional continuity condition~\citep{mhaskar1996neural}.

%For any $f$, suppose an approximation method $Q_N: \cF \to \cH_N$ produces $g = Q_N(f)$; ideally, such $g$ attains $\inf_{g \in \cH_N} \| f - g \|_p$. A consequence of~\cite{mhaskar1996neural} implies that any $Q_N$ such that parameters $c_j$, $\ww_j$, and $b_j$ ($j \in [N]$) continuously depends on the input function $f$ satisfies the lower bound
%\begin{equation}\label{ineq:approxlower}
%\sup_{f \in \cF^m_p} \| f - Q_N f \|_p \ge C'_{d,m,p}\, N^{-m/d}.
%\end{equation}

%\begin{thm}\label{thm:approx2}
%Let $Q_N: \cF \to \cH_N$ be any method of approximation where the parameters $c_j$, $\ww_j$, and $b_j$ ($j \in [N]$) are continuously dependent on %the function being approximated. Then,
%\begin{equation}\label{ineq:approxlower}
%\sup_{f \in \cF^m_p} \| f - Q_N f \|_p \ge C'_{d,m,p}\, N^{-m/d},
%\end{equation}
%where $C'_{d,m,p}$ is independent of $N$.
%\end{thm}
%There are additional results obtained in~\citep{maiorov2000near}. In particular, even if we drop the requirement of continuity, a similar lower bound holds for the logistic activation function $\sigma_*(t) = (1+e^{-t})^{-1}$ (with $N$ replaced by $N\log N$ in  (\ref{ineq:approxlower})).

For the natural space $\cF^m_p$ of smooth functions, the exponential dependence on $d$ in the upper and lower bounds may look unappealing. However,~\cite{barron1993universal} showed that for a different function space, there is a good dimension-free approximation by the neural nets. Suppose that a function $f: \mathbb{R}^{d} \mapsto \mathbb{R}$ has a Fourier representation
\begin{equation} \label{eq5.3}
f(\xx) = \int_{\R^{d}} e^{i \langle \bomega, \xx \rangle} \tilde f (\bomega)\; d\bomega,
\end{equation}
where $\tilde f (\bomega) \in \mathbb{C}$. Assume that $f(\bzero) = 0$ and that the following quantity is finite
\begin{equation}\label{def:Cf}
C_f = \int_{\R^{d}} \| \bomega \|_2 | \tilde f (\bomega) | \; d\bomega.
\end{equation}
\cite{barron1993universal} uncovers the following dimension-free approximation guarantee.
\begin{thm}[Proposition~1 in \cite{barron1993universal}]\label{thm:approx3}
Fix a $C>0$ and an arbitrary probability measure $\mu$ on the unit ball $B^d$ in $\R^d$. For every function $f$ with $C_f \le C$ and every $N \ge 1$, there exists some $g \in \cH_N$ such that
\begin{equation*}
\left[ \int_{B^d} ( f(\xx) - g(\xx))^2 \, \mu(d\xx) \right]^{1/2} \le \frac{2C}{\sqrt{N}}.
\end{equation*}
Moreover, the coefficients of $g$ may be restricted to satisfy $\sum_{j=1}^N |c_j| \le 2C$.% and $c_0 = f(\bzero)$.
\end{thm}
The upper bound is now independent of the dimension $d$. %The intuition is similar to Monte Carlo method: $f$ is in the closure of the convex hull of $\{ \sigma_*(\ww_j ^\top \xx - b_j) \}$, so $f$ can be viewed as an expected function. Thus, sampling using $N$ units produces the bound $2C / \sqrt{N}$, where $2C$ is a bound on the variance.
However, $C_f$ may implicitly depend on $d$, as the formula in \eqref{def:Cf} involves an integration over $\R^{d}$ (so for some functions $C_f$ may depend exponentially on $d$). Nevertheless, this theorem does characterize an interesting function space with an improved upper bound. Details of the function space are discussed by~\cite{barron1993universal}. This theorem can be generalized; see~\cite{makovoz1996random} for an example.

To help understand why a dimensionality-free approximation holds, let us appeal to a heuristic argument given by Monte Carlo simulations. It is well-known that Monte Carlo approximation errors are independent of dimensionality in evaluation of high-dimensional integrals.  Let us generate $\{\bomega_j\}_{1\leq j \leq N}$ randomly from a given density $p(\cdot)$ in $\R^d$.  Consider the approximation to \eqref{eq5.3} by
\begin{equation} \label{eq5.4}
g_N(\xx) = \frac{1}{N} \sum_{j=1}^N c_j e^{i \langle \bomega_j, \xx \rangle}, \qquad c_j = \frac{\tilde f (\bomega_j)}{p(\bomega_j)}.
\end{equation}
Then, $g_N(\xx)$ is a one-hidden-layer neural network with $N$ units and the sinusoid activation function.  Note that $\E g_N(\xx) = f(\xx)$, where the expectation is taken with respect to randomness $\{\bomega_j\}$.  Now, by independence, we have
$$
    \E( g_N(\xx) - f(\xx))^2 = \frac{1}{N} \var(c_j e^{i \langle \bomega_j, \xx \rangle})\leq   \frac{1}{N} \E c_j^2,
$$
if $\E c_j^2 < \infty$.  Therefore, the rate is independent of the dimensionality $d$, though the constant can be.




\subsection{Approximation theory for multi-layer NNs}
The approximation theory for multilayer neural nets is less understood compared with neural nets with one hidden layer. Driven by the success of deep learning, there are many recent papers focusing on expressivity of deep neural nets. As studied by~\cite{telgarsky2016benefits, eldan2016power, mhaskar2016learning, poggio2017and, bauer2017deep, schmidt2017nonparametric, lin2017does,rolnick2017power}, deep neural nets excel at representing \textit{composition} of functions. This is perhaps not surprising, since deep neural nets are themselves defined by composing layers of  functions. Nevertheless, it points to a new territory rarely studied in statistics before. Below we present a result based on~\cite{lin2017does,rolnick2017power}.

Suppose that the inputs $\xx$ have a bounded domain $[-1,1]^d$ for simplicity. As before, let $\sigma_*: \R \to \R$ be a generic function, and $\bsigma_* = (\sigma_*, \cdots, \sigma_*)^\top$ be element-wise application of $\sigma_*$. Consider a neural net which is similar to \eqref{eq:fc} but with scaler output: $g(\xx) = \bW_\ell \bsigma_*(\cdots \bsigma_*(\bW_2 \bsigma_*(\bW_1 \xx))\cdots)$. A unit or neuron refers to an element of vectors $\bsigma_*(\bW_k \cdots \bsigma_*(\bW_2 \bsigma_*(\bW_1 \xx)) \cdots)$ for any $k=1,\ldots,\ell-1$. For a multivariate polynomial $p$, define $m_k(p)$ to be the smallest integer such that, for any $\epsilon > 0$, there exists a neural net $g(\xx)$ satisfying $\sup_\xx \left| p(\xx) - g(\xx) \right| < \epsilon$, with $k$ hidden layers (i.e., $\ell = k+1$) and no more than $m_k(p)$ neurons in total. Essentially, $m_k(p)$ is the minimum number of neurons required to approximate $p$ arbitrarily well.

\begin{thm}[Theorem~4.1 in \cite{rolnick2017power}]\label{thm:approx4}
Let $p(\xx)$ be a monomial $x_1^{r_1} x_2^{r_2} \cdots x_d^{r_d}$ with $q = \sum_{j=1}^d r_j$. Suppose that $\sigma_*$ has derivatives of order $2q$ at the origin, and that they are nonzero. Then,\\
(i) $m_1(p) = \prod_{j=1}^d (r_j + 1)$; \\
(ii) $\min_k m_k(p) \le \sum_{j=1}^d \left( 7 \lceil \log_2(r_j) \rceil + 4  \right)$.
\end{thm}

This theorem reveals a sharp distinction between shallow networks (one hidden layer) and deep networks. To represent a monomial function, a shallow network requires exponentially many neurons in terms of the dimension $d$, whereas linearly many neurons suffice for a deep network (with bounded $r_j$). The exponential dependence on $d$, as shown in Theorem~\ref{thm:approx4}(i), is resonant with the curse of dimensionality widely seen in many fields; see~\cite{donoho2000high}. One may ask: how does depth help? Depth circumvents this issue, at least for certain functions, by allowing us to represent function composition efficiently. Indeed, Theorem~\ref{thm:approx4}(ii) offers a nice result with clear intuitions: it is known that the product of two scalar inputs can be represented using $4$ neurons~\citep{lin2017does}, so by composing multiple products, we can express monomials with $O(d)$ neurons.

Recent advances in nonparametric regressions also support the idea that deep neural nets excel at representing composition of functions~\citep{bauer2017deep, schmidt2017nonparametric}. In particular,~\cite{bauer2017deep} considered the nonparametric regression setting where we want to estimate a function $\hat f_n(\xx)$ from i.i.d.~data $\mathcal{D}_n = \{ (y_i, \xx_i) \}_{1\leq i\leq n}$. If the true regression function $f(\xx)$ has certain hierarchical structure with intrinsic dimensionality\footnote{Roughly speaking, the true regression function can be represented by a tree where each node has at most $d^*$ children. See~\cite{bauer2017deep} for the precise definition.} $d^*$, then the error
\begin{equation*}
\E_{\mathcal{D}_n} \E_{\xx} \left| \hat f_n(\xx) - f(\xx) \right|^2
\end{equation*}
has an optimal minimax convergence rate $O(n^{-\frac{2q}{2q+d^*}})$, rather than the usual rate $O(n^{-\frac{2q}{2q+d}})$ that depends on the ambient dimension $d$. Here $q$ is the smoothness parameter. This provides another justification for deep neural nets: if data are truly hierarchical, then the quality of approximators by deep neural nets depends on the intrinsic dimensionality, which avoids the curse of dimensionality.
%
%Some other notations are recently proposed to study the function space of DNNs, including spectral norms, margins, etc.; see~\cite{NIPS2017_7204, }.

We point out that the approximation theory for deep learning is far from complete.
%The existing theory has not yet fully explained the successes of deep learning algorithms.
For example, in Theorem~\ref{thm:approx4}, the condition on $\sigma_*$ excludes the widely used ReLU activation function, there are no constraints on the magnitude of the weights (so they can be unreasonably large)%, and the proof of (ii) uses a special type of nets
.

%Moreover, as we will soon discuss, the \textit{existence} of neural nets as good function approximators does not explain why in practice we can easily \textit{find} them.

\section{Training deep neural nets }\label{sec:opt}
The \textit{existence} of a good function approximator in the NN function class does not explain why in practice we can easily \textit{find} them.
In this section, we introduce standard methods, namely \emph{stochastic gradient descent} (SGD) and its variants, to train deep neural networks (or to find such a good approximator). As with many statistical machine learning tasks, training DNNs follows the \emph{empirical risk minimization} (ERM) paradigm which solves the following optimization problem
\begin{equation}
\text{minimize}_{\btheta\in\mathbb{R}^{p}}\qquad\ell_{n}\left(\btheta\right)\triangleq\frac{1}{n}\sum_{i=1}^{n}\mathcal{L}\left(f\left(\bm{x}_{i};\btheta\right),y_{i}\right).\label{eq:ERM_for_DL}
\end{equation}
Here $\mathcal{L}(f(\bm{x}_{i};\btheta),y_{i})$ measures the discrepancy between the prediction $f(\bm{x}_{i};\btheta)$ of the neural network and the true label $y_{i}$. Correspondingly, denote by $\ell(\btheta) \triangleq \mathbb{E}_{(\bm{x},y)\sim\mathcal{D}}[\mathcal{L}(f(\bm{x};\btheta),\bm{y})]$ the out-of-sample error, where $\mathcal{D}$ is the joint distribution over $(y, \bm{x})$.  Solving ERM~(\ref{eq:ERM_for_DL}) for deep neural nets faces various challenges that roughly fall into the following three categories.
\begin{itemize}
\item \emph{Scalability and nonconvexity.} Both the sample size $n$ and the number of parameters $p$ can be huge for modern deep learning applications, as we have seen in Table~\ref{tab:intro}.
%In such a large-scale setting, many
Many optimization algorithms are not practical due to the computational costs and memory constraints. What is worse, the empirical loss function $\ell_{n}(\bm{\theta})$ in deep learning is often nonconvex. It is \emph{a priori }not clear whether an optimization algorithm can drive the empirical loss (\ref{eq:ERM_for_DL}) small.

\item \emph{Numerical stability.} With a large number of layers in DNNs, the magnitudes of the hidden nodes can be drastically different, which may result in the ``exploding gradients'' or ``vanishing gradients'' issue during the training process. This is because the recursive relations across layers often lead to exponentially increasing$\,$/$\,$decreasing values in both forward passes and backward passes.

\item \emph{Generalization performance.} Our ultimate goal is to find a parameter $\hat {\bm{\theta}}$ such that the out-of-sample error $\ell(\hat \btheta)$ is small. %However, in the over-parametrized regime where the number of parameters $p$ in a deep neural network is much larger than the sample size $n$,
However, in the over-parametrized regime where $p$ is much larger than $n$, the underlying neural network has the potential to fit the training data perfectly while performing poorly on the test data. To avoid this overfitting issue, proper regularization, whether explicit or implicit, is needed in the training process for the neural nets to generalize.
\end{itemize}

In the following three subsections, we discuss practical solutions$\,$/$\,$proposals to address these challenges.

\subsection{Stochastic gradient descent \label{sec:stochastic-opt}}

Stochastic gradient descent (SGD)~\citep{robbins1951stochastic} is by far the most popular optimization algorithm to solve ERM~(\ref{eq:ERM_for_DL})
for large-scale problems. It has the following simple update rule:
\begin{equation}
\btheta^{t+1}=\btheta^{t}-\eta_{t}G(\bm{\theta}^{t})\qquad\text{with}\qquad G\left(\bm{\theta}^{t}\right)=\nabla\mathcal{L}\left(f\left(\bm{x}_{i_{t}};\btheta^{t}\right),y_{i_{t}}\right)\label{eq:SGD_DL}
\end{equation}
for $t=0,1,2,\ldots$, where $\eta_{t}>0$ is the step size (or learning rate), $\btheta^{0}\in\mathbb{R}^{p}$ is an initial point and $i_{t}$ is chosen randomly from $\{1,2,\cdots, n\}$. It is easy to verify that $G(\bm{\theta}^{t})$ is an unbiased estimate of $\nabla\ell_{n}(\bm{\theta}^{t})$.
%Depending on the rule of selecting $i_t$ in each step, there are two variants of SGD:
%\begin{itemize}
%	\item \emph{One-pass SGD}. Each $i_{t}$ is drawn \emph{without} replacement from $\{1,2,\cdots,n\}$ and the maximum number of steps is $n$, the number of samples available. In this setting, the stochastic gradient $G(\bm{\theta}^{t})$ is an unbiased estimate of $\nabla\ell(\bm{\theta}^{t})$, the gradient of the \emph{population} loss and one-pass SGD is regarded as a method to minimize $\ell(\btheta)$.
%	\item \emph{Multi-pass SGD}. Each $i_{t}$ is drawn \emph{with} replacement from $\{1,2,\cdots,n\}$ independently and there is no essential upper limit on the number of steps. Under this rule, the stochastic gradient $G(\bm{\theta}^{t})$ is an unbiased estimate of $\nabla\ell_{n}(\bm{\theta}^{t})$, the gradient of the \emph{empirical} loss and multi-pass SGD can be viewed as an algorithm to minimize $\ell_{n}(\btheta)$.
%\end{itemize}
%This division is merely for theoretical purpose and practitioners often strike a balance between these two approaches: we often run SGD for a number of \emph{epochs}, where in each epoch we follow the rule for one-pass SGD.
The advantage of SGD is clear: compared with gradient descent, which goes over the entire dataset in every update, SGD uses a single example in each update and hence is considerably more efficient in terms of both computation and memory (especially in the first few iterations).
%In addition, the one-pass SGD allows to update the parameters online, i.e., when the training
%data arrive on the fly. These desirable features make SGD scalable to huge datasets, as is often the case in deep learning.

Apart from practical benefits of SGD, how well does SGD perform theoretically in terms of minimizing $\ell_{n}(\btheta)$? We begin with the convex case, i.e., the case where the loss function is convex w.r.t.~$\bm{\theta}$. It is well understood in literature that with proper choices of the step sizes $\{\eta_{t}\}$, SGD is guaranteed to achieve both \emph{consistency} and \emph{asymptotic normality}.
\begin{itemize}
\item {\emph{Consistency}.} If $\ell(\btheta)$ is a strongly convex function\footnote{For results on consistency and asymptotic normality, we consider the case where in each step of SGD, the stochastic gradient is computed using a fresh sample $(y, \bm{x})$ from $\mathcal{D}$. This allows to view SGD as an optimization algorithm to minimize the population loss $\ell(\btheta)$.}, then under some mild conditions\footnote{One example of such condition can be constraining the second moment of the gradients: $\E\left[\|\nabla\mathcal{L}\left(\bm{x}_{i},y_{i};\btheta^{t}\right)\|_{2}^{2}\right]\le C_{1}+C_{2}\|\btheta^{t}-\btheta^{*}\|_{2}^{2}$ for some $C_{1},C_{2}>0$. See~\cite{bottou1998online} for details.}, learning rates that satisfy
\begin{equation}\label{cond:lr}
\sum_{t=0}^{\infty}\eta_{t}=+\infty\qquad\text{and}\qquad\sum_{t=0}^{\infty}\eta_{t}^{2}<+\infty
\end{equation}
guarantee almost sure convergence to the unique minimizer $\btheta^* \triangleq \argmin_{\btheta} \ell(\btheta)$, i.e., $\btheta^{t}\xrightarrow{\mbox{a.s.}}\btheta^{*}$ as $t\to\infty$~\citep{robbins1951stochastic,kiefer1952stochastic,bottou1998online,kushner2003stochastic}. The requirements in \eqref{cond:lr} can be viewed from the perspective of bias-variance tradeoff: the first condition ensures that the iterates can reach the minimizer (controlled bias), and the second ensures that stochasticity does not prevent convergence (controlled variance).

\item {\emph{Asymptotic normality}.} It is proved by~\cite{polyak1979adaptive} that for robust linear regression with fixed dimension $p$, under the choice $\eta_{t}=t^{-1}$, $\sqrt{t}\,(\btheta^{t}-\btheta^{*})$ is asymptotically normal under some regularity conditions (but $\btheta^{t}$ is not asymptotically efficient in general). Moreover, by averaging the iterates of SGD,~\cite{polyak1992acceleration} proved that even with a \emph{larger} step size $\eta_{t}\propto t^{-\alpha},\alpha\in(1/2,1)$, the averaged iterate $\bar{\btheta}^{t}=t^{-1}\sum_{s=1}^{t}\btheta^{s}$ is asymptotic efficient for robust linear regression. These strong results show that SGD with averaging performs as well as the MLE asymptotically, in addition to its computational efficiency. %\cm{To Yiqiao, mention that this holds for fixed $p$ and diverging $n$.}
\end{itemize}

These classical results, however, fail to explain the effectiveness of SGD when dealing with nonconvex loss functions in deep learning. Admittedly, finding global minima of nonconvex functions is computationally infeasible in the worst case. Nevertheless, recent work~\citep{allen2018convergence,du2018gradient} bypasses the worst case scenario by focusing on losses incurred by over-parametrized deep learning models. In particular, they show that (stochastic) gradient descent converges linearly towards the \emph{global }minimizer of $\ell_{n}(\btheta)$ as long as the neural network is sufficiently \emph{over-parametrized}. This phenomenon is formalized below.

\begin{thm}[Theorem~2 in \citealp{allen2018convergence}]Let $\{(y_i, \bm{x}_{i})\}_{1 \leq i \leq n}$
be a training set satisfying $\min_{i,j:i\neq j}\|\bm{x}_{i}-\bm{x}_{j}\|_{2}\geq\delta>0$. Consider fitting the data using a feed-forward neural network~(\ref{model:1}) with
ReLU activations. Denote by $L$ (resp.~$W$) the depth (resp.~width) of the network. Suppose that the neural network is sufficiently over-parametrized, i.e.,
\begin{equation}
W\gg\mathsf{poly}\left(n,L,\frac{1}{\delta}\right), \label{eq:overparametrize}
\end{equation}
where $\mathsf{poly}$ means a polynomial function. Then with high probability, running SGD~(\ref{eq:SGD_DL}) with \emph{certain
random initialization} and properly chosen step sizes yields $\ell_{n}(\bm{\theta}^{t})\leq\varepsilon$
in $t\asymp\log\frac{1}{\varepsilon}$ iterations. \end{thm}

Two
notable features are worth mentioning:~(1) first, the network under
consideration is sufficiently over-parametrized (cf.~(\ref{eq:overparametrize})) in which the number of parameters is \emph{much} larger than the number of samples,
and (2) one needs to initialize the weight matrices to be in near-isometry such that the magnitudes of the hidden nodes do not blow up or vanish. In a nutshell, \emph{over-parametrization}
and \emph{random initialization} together ensure that the loss function
(\ref{eq:ERM_for_DL}) has a benign landscape\footnote{In~\cite{allen2018convergence},
the loss function $\ell_{n}(\btheta)$ satisfies the PL condition.} around the initial
point, which in turn implies fast convergence of SGD iterates.


%Next, we present two theoretical results of SGD in the nonconvex scenario: the first one deals with \emph{general} nonconvex functions and the second focuses on the loss functions of neural networks.
%\begin{itemize}
%\item \emph{SGD with general nonconvex loss functions.} We abuse the notation and denote by $\ell (\bm{\theta})$ a general nonconvex loss function (not necessarily the expectation of (\ref{eq:ERM_for_DL})). When dealing with nonconvex $\ell(\bm{\theta})$, computational hardness results (e.g., NP-hardness) prevent us from finding the local optimum in general, not to mention the global optimum. Therefore a modest and realistic goal would be to find \emph{a first-order stationary point} $\bm{\theta}$ of $\ell(\cdot)$, i.e., the point $\btheta$ obeying $\nabla\ell (\bm{\theta})=\bm{0}$. The following result from stochastic optimization shows that one can indeed find such (approximate) first-order stationary points using SGD.
%
%\begin{thm}[Corollary~2.2 in \citealp{ghadimi2013stochastic}]Suppose
%that $\ell (\bm{\theta})$ is 1-smooth in the sense that $\|\nabla\ell(\bm{\theta}_{1})-\nabla\ell(\bm{\theta}_{2})\|_{2}\leq\|\bm{\theta}_{1}-\bm{\theta}_{2}\|_{2}$
%for all $\bm{\theta}_{1}$ and $\bm{\theta}_{2}$. Consider the stochastic gradient descent method, where in each iteration~$t$ the stochastic
%gradient $G(\bm{\theta}^{t})$ is an unbiased estimate of $\nabla\ell(\bm{\theta}^{t})$ and the variance\footnote{Here the expectation and the variance are conditioning on all the randomness up to the $t$-th iteration. } of the stochastic gradient is upper bounded by $\sigma^{2}$, i.e., $\mathbb{E}[\|G(\bm{\theta}^{t})-\nabla\ell(\bm{\theta}^{t})\|_{2}^{2}]\leq\sigma^{2}$. Then running SGD for $T$ steps with properly chosen constant step sizes achieves
%\[
%\mathbb{E}\big[\|\nabla\ell(\hat{\bm{\theta}})\|_{2}^{2}\big]\lesssim\frac{1}{T}+\frac{\sigma}{\sqrt{T}}.
%\]
%Here $\hat{\bm{\theta}}$ is chosen from the SGD iterates $\{\bm{\theta}^{t}\}_{0\leq t\leq T}$
%uniformly at random.
%\end{thm}
%
%
%
%In addition to first-order stationary points, a line of work~\citep{ge2015escaping, fang2019sharp, jin2019stochastic} focuses on finding second-order
%stationary points using SGD (or its variants); see the reference therein
%for the precise definition of second-order stationary points. However, a first-order (or second-order) stationary point does not necessarily translate to a point $\btheta$ that has small loss $\ell(\btheta)$.
%\emph{SGD for over-parametrized neural nets.} Concentrating on the nonconvex loss functions of deep neural networks, several recent papers~\citep{allen2018convergence,du2018gradient} %\end{itemize}

There are certainly other challenges for vanilla SGD to train deep neural nets: (1) training algorithms are often implemented in GPUs, and therefore it
is important to tailor the algorithm to the infrastructure, (2) the
vanilla SGD might converge very slowly for deep neural networks,
albeit good theoretical guarantees for well-behaved problems, and (3) the learning
rates $\{\eta_{t}\}$ can be difficult to tune in practice. To address
the aforementioned challenges, three important variants of SGD, namely
\emph{mini-batch SGD, momentum-based SGD}, and \emph{SGD with adaptive
learning rates }are introduced.

\subsubsection{Mini-batch SGD}

Modern computational infrastructures (e.g., GPUs)
can evaluate the gradient on a number (say 64) of examples as efficiently
as evaluating that on a single example. To utilize this advantage, mini-batch SGD
with batch size $K\geq1$ forms the stochastic gradient through $K$
random samples:
\begin{equation}
\bm{\theta}^{t+1}=\bm{\theta}^{t}-\eta_{t}G(\bm{\theta}^{t})\qquad\text{with}\qquad G(\bm{\theta}^{t})=\frac{1}{K}\sum_{k=1}^{K}\nabla\mathcal{L}\big(f\big(\bm{x}_{i_{t}^{k}};\btheta^{t}\big),y_{i_{t}^{k}}\big),\label{eq:mini-SGD}
\end{equation}
where for each $1\leq k\leq K$, $i_{t}^{k}$ is sampled uniformly
from $\{1,2,\cdots,n\}$. Mini-batch SGD, which is an ``interpolation''
between gradient descent and stochastic gradient descent, achieves
the best of both worlds: (1) using $1\ll K\ll n$ samples to estimate
the gradient, one effectively reduces the variance and hence accelerates
the convergence, and (2) by taking the batch size $K$ appropriately (say
64 or 128), the stochastic gradient $G(\bm{\theta}^{t})$ can be efficiently
computed using the matrix computation toolboxes on GPUs.

\subsubsection{Momentum-based SGD}
While mini-batch SGD forms the foundation
of training neural networks, it can sometimes be slow to converge
due to its oscillation behavior~\citep{sutskever2013importance}. Optimization community has long investigated
how to accelerate the convergence of gradient descent,
which results in a beautiful technique called \emph{momentum methods}
\citep{polyak1964some,nesterov1983method}. Similar to gradient descent with moment, \emph{momentum-based SGD}, instead of moving the iterate
$\bm{\theta}^{t}$ in the direction of the current stochastic gradient $G(\btheta^t)$, smooth the past (stochastic) gradients $\{G(\btheta^t)\}$ to stabilize the update directions. Mathematically,
let $\bm{v}^{t}\in\mathbb{R}^{p}$ be the direction of update in the
$t$th iteration, i.e.,
\[
\bm{\theta}^{t+1}=\bm{\theta}^{t}-\eta_{t}\bm{v}^{t}.
\]
Here $\bm{v}^{0}=G(\bm{\theta}^{0})$ and for $t=1,2,\cdots$
\begin{equation} \label{eq6.6}
\bm{v}^{t}=\rho\bm{v}^{t-1}+G(\bm{\theta}^{t})
\end{equation}
with $0<\rho<1$. A typical choice of $\rho$ is 0.9. Notice
that $\rho=0$ recovers the mini-batch SGD (\ref{eq:mini-SGD}),
where no past information of gradients is used. A simple unrolling
of $\bm{v}^{t}$ reveals that $\bm{v}^{t}$ is actually an exponential
averaging of the past gradients, i.e., $\bm{v}^{t}=\sum_{j=0}^{t}\rho^{t-j}G(\bm{\theta}^{j}).$
Compared with vanilla mini-batch SGD, the inclusion of the momentum
``smoothes'' the oscillation direction and accumulates the persistent
descent direction. We want to emphasize that theoretical justifications of momentum in the \emph{stochastic} setting is not fully understood~\citep{kidambi2018insufficiency, jain2017accelerating}.

%Note that \eqref{eq6.6} is indeed the same as exponential smoothing in time-domain in statistics:
%$$
%  \bm{v}^{t}=\rho\bm{v}^{t-1}+ (1 - \rho) G(\bm{\theta}^{t}),
%$$
%which yields the solution $\bm{v}^{t}= (1-\rho) \sum_{j=0}^{t}\rho^{t-j}G(\bm{\theta}^{j})$.  The factor $(1-\rho)$ can be absorbed into the learn rate or step size parameter.

\subsubsection{SGD with adaptive learning rates}

In optimization, \emph{preconditioning} is often used to accelerate first-order optimization algorithms. In principle, one can apply this to SGD, which yields the following update rule:
\begin{equation}
\bm{\theta}^{t+1}=\bm{\theta}^{t}-\eta_{t}\bm{P}_{t}^{-1}G(\bm{\theta}^{t})\label{eq:precondition-SGD}
\end{equation}
with $\bm{P}_t \in \mathbb{R}^{p\times p}$ being a preconditioner at the $t$-th step. Newton's method can be viewed as one type of preconditioning where $\bm{P}_{t} = \nabla^2 \ell(\btheta^t)$. The advantages of preconditioning are two-fold: first, a good preconditioner reduces the condition number by changing the local geometry to be more homogeneous, which is amenable to fast convergence; second, a good preconditioner frees practitioners from laboring tuning of the step sizes, as is the case with Newton's method. AdaGrad, an adaptive gradient method proposed by~\cite{duchi2011adaptive}, builds a preconditioner $\bm{P}_{t}$ based on information of the past gradients:
\begin{equation}
\bm{P}_{t}= \Big \{ \mathsf{diag}\Big(\sum_{j=0}^{t}G\left(\bm{\theta}^{t}\right)G
\left(\bm{\theta}^{t}\right)^{\top}\Big) \Big \}^{1/2} \label{eq:adagrad-precondition}.
\end{equation}
Since we only require the diagonal part, this preconditioner (and its inverse) can be efficiently computed in practice. In addition, investigating~(\ref{eq:precondition-SGD}) and~(\ref{eq:adagrad-precondition}), one can see that AdaGrad adapts to the importance
of each coordinate of the parameters by setting smaller learning rates
for frequent features, whereas larger learning rates for those infrequent
ones. In practice, one adds a small quantity $\delta > 0$ (say $10^{-8}$) to the diagonal entries to avoid singularity (numerical underflow). A notable drawback of AdaGrad is that the effective learning rate vanishes quickly along the learning process. This
is because the historical sum of the gradients can only
increase with time. RMSProp~\citep{hinton2012neural} is a popular
remedy for this problem which incorporates the idea of exponential averaging:
\begin{equation}
\bm{P}_{t}= \Big \{ \mathsf{diag}\Big(\rho \bm{P}_{t-1} +
(1-\rho)G\left(\bm{\theta}^{t}\right)G\left(\bm{\theta}^{t}\right)^{\top}
\Big) \Big \}^{1/2} \label{eq:rmsprop-precondition}.
\end{equation}
Again, the decaying parameter $\rho$ is usually set to be $0.9$.
Later, Adam~\citep{kingma2014adam, reddi2018convergence} combines
the momentum method and adaptive learning rate and becomes the default training algorithms in many deep learning applications.
%Note that for all the methods
%mentioned above, we use the same step size$\,$/$\,$learning rate $\eta_{t}$
%for all the coordinates in $\btheta$, albeit $\eta_{t}$ can vary
%with time $t$. Nothing prevents us from applying different step sizes
%for each coordinate of the parameters $\Theta$. Moreover, one would
%prefer an adaptive learning rate which can automatically adapt to
%the importance of each dimension.

%This is precisely the goal of AdaGrad
%, which uses to automatically determine the learning rate for each coordinate.
%Formally, one records a historical sum $\bm{r}^{t}\in\mathbb{R}^{p}$
%via the following recursive formula
%\[
%\bm{r}^{t+1}=\bm{r}^{t}+\bm{v}^{t}\otimes\bm{v}^{t},\qquad\text{for }t=0,1,2,\cdots,
%\]
%with $\bm{r}^{0}=\bm{0}$ and $\otimes$ denoting the element-wise
%product. Then the update rule of AdaGrad is given by
%\begin{equation}
%\bm{\theta}^{t+1}=\bm{\theta}^{t}-\frac{\epsilon}{\sqrt{\bm{r}^{t+1}}+\delta}\otimes\bm{v}^{t},\label{eq:AdaGrad}
%\end{equation}
%where $\sqrt{\cdot}$ denotes the element-wise square root and $\delta>0$
%is set to be some small constant, say $10^{-8}$, to prevent numerical
%overflow. In addition $\epsilon>0$ is some global learning rate,
%which is usually a fixed small constant, say, 0.01.
%\subsection{Batch normalization}\label{sec:batch-norm}
\subsection{Easing numerical instability}\label{sec:batch-norm}

%We have already seen some partial solutions: (1) the use of ReLU function helps overcome vanishing gradients because its derivative remains $1$ even for a large input (recall Eqn.~\ref{eq:grad})---this is in contrast with the sigmoid function which tends to ``kill'' gradients; (2) skip connections stabilize the gradients by adding an identity map (recall Eqn.~\ref{eq:skipgrad}).

For very deep neural networks or RNNs with long dependencies, training difficulties often arise when the values of nodes have different magnitudes or when the gradients ``vanish'' or ``explode'' during back-propagation. Here we discuss three partial solutions to alleviate this problem.

\subsubsection{ReLU activation function}

One useful characteristic of the ReLU function is that its derivative is either $0$ or $1$, and the derivative remains $1$ even for a large input. This is in sharp contrast with the standard sigmoid function $(1 + e^{-t})^{-1}$ which results in a very small derivative when inputs have large magnitude. The consequence of small derivatives across many layers is that gradients tend to be ``killed'', which means that gradients become approximately zero in deep nets. %Indeed, during back-propagation, gradient calculation is multiplicative due to the recursive relation~\eqref{eq:grad}, and thus the standard sigmoid function is susceptible to producing vanishing gradients.

The popularity of the ReLU activation function and its variants (e.g., leaky ReLU) is largely attributable to the above reason. It has been well observed that the ReLU activation function has superior training performance over the sigmoid function~\citep{krizhevsky2012imagenet, maas2013rectifier}.

\subsubsection{Skip connections}

We have introduced skip connections in Section~\ref{sec:skip}. Why are skip connections helpful for reducing numerical instability? This structure does not introduce a larger function space, since the identity map can be also represented with ReLU activations: $\xx = \bsigma(\xx) - \bsigma(-\xx)$.

One explanation is that skip connections bring ease to the training$\,$/$\,$optimization process. Suppose that we have a general nonlinear function $\bF(\xx_\ell; \btheta_{\ell})$. With a skip connection, we represent the map as $\xx_{\ell+1} = \xx_\ell + \bF(\xx_\ell; \btheta_{\ell})$ instead. Now the gradient $\partial \xx_{\ell+1} / \partial \xx_{\ell}$ becomes
\begin{equation}\label{eq:skipgrad}
\frac{\partial \xx_{\ell+1}}{\partial \xx_{\ell}} = \bI + \frac{\partial \bF(\xx_\ell; \btheta_{\ell})}{\partial \xx_\ell} \qquad \text{instead of} \qquad \frac{\partial \bF(\xx_\ell; \btheta_{\ell})}{\partial \xx_\ell},
\end{equation}
where $\bI$ is an identity matrix. By the chain rule, gradient update requires computing products of many components, e.g., $\frac{\partial \xx_L}{\partial \xx_1} = \prod_{\ell=1}^{L-1} \frac{\partial \xx_{\ell+1}}{\partial \xx_\ell}$, so it is desirable to keep the spectra (singular values) of each component $\frac{\partial \xx_{\ell+1}}{\partial \xx_\ell}$ close to $1$. In neural nets, with skip connections, this is easily achieved if the parameters have small values; otherwise, this may not be achievable even with careful initialization and tuning. Notably, training neural nets with hundreds of layers is possible with the help of skip connections.

%With skip connections, it is plausible that improved training/optimization finds ``better'' local optima. To what extent or in what sense these the local optima are better? While a thorough understanding remains elusive, we believe that an answer requires statistical analyses of optimization algorithms, since skip connections do not enlarge the function space.

\subsubsection{Batch normalization}

%Here, we discuss another widely used approach: \textit{batch normalization}~\citep{ioffe2015batch}.
%Another approach to alleviate numerical instability is \textit{batch normalization}~\citep{ioffe2015batch}. The key to batch normalization is \emph{standardization}.
Recall that in regression analysis, one often standardizes the design matrix so that the features have zero mean and unit variance. Batch normalization extends this standardization procedure from the input layer to all the hidden layers. Mathematically, fix a mini-batch of input data $\{(\bm{x}_{i},y_{i})\}_{i\in \mathcal{B}}$, where $\mathcal{B} \subset [n]$. Let $\bm{h}_{i}^{(\ell)}$ be the feature of the $i$-th example in the $\ell$-th layer ($\ell=0$ corresponds to the input $\bm{x}_{i}$). The batch normalization layer computes the normalized version of $\bm{h}_{i}^{(\ell)}$ via the following steps:
\begin{align*}
\bm{\mu} & \triangleq \frac{1}{\left|\mathcal{B}\right|}\sum_{i\in \mathcal{B}}\bm{h}_{i}^{(\ell)},\qquad \bm{\sigma}^{2}  \triangleq\frac{1}{\left|\mathcal{B}\right|}\sum_{i\in \mathcal{B}}\big(\bm{h}_{i}^{(\ell)}-\bm{\mu}\big)^{2} \qquad\text{and}\qquad \bm{h}_{i,\text{norm}}^{(l)}  \triangleq\frac{\bm{h}_{i}^{(\ell)}-\bm{\mu}}{\bm{\sigma}}.
\end{align*}
Here all the operations are element-wise. In words, batch normalization computes the z-score for each feature over the mini-batch $\mathcal{B}$ and use that as inputs to subsequent layers. To make it more versatile, a typical batch normalization layer has two additional learnable parameters $\bm{\gamma}^{(\ell)}$ and $\bm{\beta}^{(\ell)}$ such that
\[
\bm{h}_{i,\text{new}}^{(l)}=\bm{\gamma}^{(l)}\odot\bm{h}_{i,\text{norm}}^{(l)}+\bm{\beta}^{(l)}.
\]
Again $\odot$ denotes the element-wise multiplication. As can be seen, $\bm{\gamma}^{(\ell)}$ and $\bm{\beta}^{(\ell)}$ set the new feature $\bm{h}_{i \text{new}}^{(l)}$ to have mean $\bm{\beta}^{(\ell)}$ and standard deviation $\bm{\gamma}^{(\ell)}$. The introduction of batch normalization makes the training of neural networks much easier and smoother. More importantly, it allows the neural nets to perform well over a large family of hyper-parameters including the number of layers, the number of hidden units, etc. At test time, the batch normalization layer needs more care. For brevity we omit the details and refer to~\cite{ioffe2015batch}.

%\cm{The current version does not explain the intuition of
%batch normalization, nor discuss benefits of batch normalization (improve
%gradient flow, initialization, etc.) We can discuss this.}

%One analogous problem in statistics is when covariates/features of
%input data have varying means and scales. In ,
%this can often lead to multicollinearity, which means some variables
%are heavily correlated. From the perspective of numerical stability,
%this issue manifests in a large condition number of $\bX^{\top}\bX$
%where $\bX$ is the design matrix. To improve prediction and inference,
%a common practice for this issue is to standardize the data, that
%is, shifting and rescaling each covariate/feature to have zero mean
%and unit variance. Likewise, in numerical analysis and optimization,
%preconditioning is usually applied to make a matrix or loss function
%better conditioned.

%In statistics, a common practice for prepossessing is to standardize data, that is, making each covariate/feature of the input data to have zero mean and unit variance. This can Similar preprocessing on the input data has been done in optimization community to make the loss function well conditioned.




\subsection{Regularization techniques} \label{sec:regularization}
So far we have focused on training techniques to drive the empirical loss (\ref{eq:ERM_for_DL}) small efficiently. Here we proceed to discuss common practice to improve the generalization power of trained neural nets.


\subsubsection{Weight decay} \label{sec:weight}
One natural regularization idea is to add an $\ell_{2}$
penalty to the loss function. This regularization technique is known
as the weight decay in deep learning. We have seen one example in
\eqref{eq:regloss}. For general deep neural nets, the loss to optimize
is $\ell_n^{\lambda}(\btheta)=\ell_{n}(\btheta)+r_{\lambda}(\btheta)$ where %\cm{To Yiqiao, I think we do not need to spend too much on weight decay.}
\[
r_{\lambda}(\btheta)=\lambda\sum_{\ell=1}^{L}\sum_{j,j'}\big[W_{j,j'}^{(\ell)}\big]^{2}.
\]
Note that the bias (intercept) terms are not penalized. If $\ell_{n}(\btheta)$
is a least square loss, then regularization with weight decay gives
precisely ridge regression. The penalty $r_{\lambda}(\btheta)$ is a smooth
function and thus it can be also implemented efficiently with back-propagation.

\subsubsection{Dropout} \label{sec:dropout}

Dropout, introduced by~\cite{hinton2012improving}, prevents overfitting by randomly dropping out subsets of features
during training. Take the $l$-th layer of the feed-forward neural
network as an example. Instead of propagating all the features in $\bm{h}^{(\ell)}$ for later
computations, dropout randomly omits some of its entries by
\[
\bm{h}_{\text{drop}}^{(\ell)}=\bm{h}^{(\ell)}\odot\mathsf{mask}^{\ell},
\]
where $\odot$ denotes element-wise multiplication as before, and $\mathsf{mask}^{\ell}$ is a vector of Bernoulli variables with success probability $p$. It is sometimes useful to rescale the features $\bm{h}_{\text{inv drop}}^{(\ell)}=\bm{h}_{\text{drop}}^{(\ell)}/p$, which is called \textit{inverted dropout}. During training, $\mathsf{mask}^{\ell}$ are i.i.d. vectors across mini-batches and layers. However, when testing on fresh samples, dropout is disabled
and the original features $\hh^{(\ell)}$ are used to compute the
output label $y$. It has been nicely shown by~\cite{wager2013dropout}
that for generalized linear models, dropout serves as adaptive
regularization. In the simplest case of linear regression, it is equivalent
to $\ell_{2}$ regularization. Another possible way to understand
the regularization effect of dropout is through the lens of bagging~\citep{deeplearningbook}.
Since different mini-batches has different masks, dropout can be viewed
as training a large ensemble of classifiers at the same time, with
a further constraint that the parameters are shared. %This viewpoint has not been theoretically justified.
Theoretical justification remains elusive.
%
%\TODO{I think we can mention random forests and add a figure---this
%looks interesting.}

%In the end, we show a snippet of the code to illustrate how to build a neural net with dropout.
%\begin{python}
%MLP = Sequential([
%  Flatten(),
%  Dense(512, activation=tf.nn.relu),
%  Dropout(0.25),
%  Dense(10, activation=tf.nn.softmax)
%])
%\end{python}

\subsubsection{Data augmentation}\label{sec:aug}
Data augmentation is a technique of enlarging the dataset when we have knowledge about invariance structure of data. It implicitly increases the sample size and usually regularizes the model effectively. For example, in image classification, we have strong prior knowledge about what invariance properties a good classifier should possess. The label of an image should not be affected by translation, rotation, flipping, and even crops of the image. Hence one can augment the dataset by randomly translating, rotating and cropping the images in the original dataset.

Formally, during training we want to minimize the loss $\ell_{n}(\btheta)=\sum_{i}\cL(f(\xx_{i};\btheta),y_{i})$
w.r.t.~parameters $\btheta$, and we know a priori that certain transformation
$T\in\mathcal{T}$ where $T:\R^{d}\to\R^{d}$ (e.g., affine transformation)
should not change the category$\,$/$\,$label of a training sample. In principle,
if computation costs were not a consideration, we could convert this
knowledge to a constraint $f_{\btheta}(T\xx_{i})=f_{\btheta}(\xx_{i}),\forall\,T\in\mathcal{T}$
in the minimization formulation. Instead of solving a constrained
optimization problem, data augmentation enlarges the training dataset
by sampling $T\in\mathcal{T}$ and generating new data $\{(T\xx_{i},y_{i})\}$.
In this sense, data augmentation induces invariance properties through sampling, which results in a much bigger dataset than the original one. 



%\subsection{Note}

%Stochastic gradient descent, as a stochastic approximation algorithm
%was first proposed in the seminal work by Robbins and Monro published
%in the \emph{The Annals of Mathematical Statistics}~\citep{robbins1951stochastic}.
%Since then, it has found numerous applications in optimization, signal
%processing, machine learning,~etc.

%The momentum method we introduced in this section is more akin to
%the \emph{Heave-ball} method proposed by Polyak~\citep{polyak1964some}.
%The goal then was to find optimal methods to optimize a smooth convex
%function. Later, Nesterov came up with an ingenious momentum-based
%method, known as Nesterov's accelerated methods~\citep{nesterov1983method},
%to achieve optimal convergence rate for smooth functions under the
%oracle model; see~\citep{nesterov2013introductory} for more details.

%The original AdaGrad~\citep{duchi2011adaptive} uses (\ref{eq:adagrad-precondition}) without $\mathsf{diag}(\cdot)$ as a preconditioner. However, this is computationally challenging when the dimension
%$p$ of the parameter is large.
%
%For brevity, we do not discuss the weight initialization methods,
%i.e., the choice of the initial point $\btheta^{0}$. Unlike convex
%optimization, where initialization of (stochastic) gradient descent
%has minor impact on the convergence property, proper weight initialization
%plays an important role in training deep neural networks~\citep{sutskever2013importance}.
%See~\citep{glorot2010understanding,he2015delving} for some popular
%choices.






\section{Generalization power}\label{sec:generalization}

Section~\ref{sec:opt} has focused on the in-sample$\,$/$\,$ training
error obtained via SGD, but this alone does not guarantee good performance with respect to the
out-of-sample$\,$/$\,$test error. The gap between the in-sample
error and the out-of-sample error, namely the \emph{generalization
gap}, has been the focus of statistical learning theory since its
birth; see \cite{shalev2014understanding} for an excellent introduction
to this topic.

%Understanding the generalization power of deep neural nets is a major open problem. This problem is further complicated by the fact that the capacity of (unrestricted) deep neural networks is large enough for them to memorize$\,$/$\,$overfit the training data~\citep{zhang2016understanding}. In this section, we survey recent attempts towards understanding the generalization power of deep learning. From a high level point of view, these approaches can be divided into two categories, namely \emph{algorithm-independent controls }and \emph{algorithm-dependent control}s. More specifically, algorithm-independent controls focus solely on bounding the \emph{complexity }of the function class represented by certain deep neural networks. In contrast, algorithm-dependent controls take into account the algorithm (e.g., SGD) used to train the neural network.

While understanding the generalization power of deep neural nets is difficult~\cite{zhang2016understanding,recht2018cifar}, we sample recent endeavors in this section. From a high level point of
view, these approaches can be divided into two categories, namely \emph{algorithm-independent controls }and \emph{algorithm-dependent
control}s. More specifically, algorithm-independent controls focus
solely on bounding the \emph{complexity }of the function class represented
by certain deep neural networks. In contrast, algorithm-dependent
controls take into account the algorithm (e.g., SGD) used
to train the neural network.


\subsection{Algorithm-independent controls: uniform convergence}

The key to algorithm-independent controls is the notion of \emph{complexity
}of the function class parametrized by certain neural networks. Informally,
as long as the complexity is not too large, the generalization gap
of \emph{any} function in the function class is well-controlled. However,
the standard complexity measure (e.g., VC dimension \citep{vapnik1971uniform})
is at least proportional to the number of weights
in a neural network \citep{anthony2009neural, shalev2014understanding}, which fails to
explain the practical success of deep learning. The caveat here is
that the function class under consideration is \emph{all} the functions realized
by certain neural networks, with \emph{no} restrictions on the size
of the weights at all. On the other hand, for the class of linear
functions with bounded norm, i.e., $\{\bm{x}\mapsto\bm{w}^{\top}\bm{x}\,|\,\|\bm{w}\|_{2}\leq M\}$,
it is well understood that the complexity of
this function class (measured in terms of the empirical Rademacher complexity) with respect to a random sample $\{\bm{x}_{i}\}_{1 \leq i \leq n}$
is upper bounded by $\max_{i}\|\bm{x}_{i}\|_{2}M/\sqrt{n}$, which
is independent of the number of parameters in $\bm{w}$. This motivates
researchers to investigate the complexity of \emph{norm-controlled}
deep neural networks\footnote{Such attempts have been made in the seminal work~\cite{bartlett1998sample}.} \citep{neyshabur2015norm,NIPS2017_7204,golowich2017size,li2018tighter}.
Setting the stage, we introduce a few necessary notations and facts.
The key object under study is the function class parametrized by the
following fully-connected neural network with depth~$L$:
\begin{equation}
\mathcal{F}_{L}\triangleq\left\{ \bm{x}\mapsto\bm{W}_{L}\bsigma\left(\bm{W}_{L-1}\bsigma\left(\cdots\bm{W}_{2}\bsigma\left(\bm{W}_{1}\bm{x}\right)\right)\right)\,\big|\,\left(\bm{W}_{1},\cdots,\bm{W}_{L}\right)\in\mathcal{W}\right\}.\label{eq:function-class-ffn}
\end{equation}
Here $(\bm{W}_{1},\bm{W}_{2},\cdots,\bm{W}_{L})\in\mathcal{W}$
represents a certain constraint on the parameters. For instance, one
can restrict the Frobenius norm of each parameter $\bm{W}_{l}$ through
the constraint $\|\bm{W}_{l}\|_{\mathrm{F}}\leq M_{\mathrm{F}}(l)$,
where $M_{\mathrm{F}}(l)$ is some positive quantity. With regard
to the complexity measure, it is standard to use \emph{Rademacher
complexity} to control the capacity of the function class of interest.

\begin{definition}[Empirical Rademacher complexity] The empirical
Rademacher complexity of a function class~$\mathcal{F}$ w.r.t.~a dataset $S\triangleq\{\bm{x}_{i}\}_{1 \leq i \leq n}$  is
defined as
\begin{equation}
\mathcal{R}_{S}\left(\mathcal{F}\right)=\mathbb{E}_{\bm{\varepsilon}}\Big[\sup_{f\in\mathcal{F}}\frac{1}{n}\sum_{i=1}^{n}\varepsilon_{i}f\left(\bm{x}_{i}\right)\Big],\label{eq:empirical-rademacher-complexity}
\end{equation}
where $\bm{\varepsilon}\triangleq(\varepsilon_{1},\varepsilon_{2},\cdots,\varepsilon_{n})$
is composed of i.i.d.~Rademacher random variables, i.e., $\mathbb{P}(\varepsilon_{i}=1)=\mathbb{P}(\varepsilon_{i}=-1)=1/2$.
\end{definition}

In words, Rademacher complexity measures
the ability of the function class to fit the random noise represented
by $\bm{\varepsilon}$. Intuitively, a function class with a larger Rademacher
complexity is more prone to overfitting. We now formalize the connection between the empirical Rademacher complexity
and the out-of-sample error; see Chapter 24 in~\cite{shalev2014understanding}.

\begin{thm}Assume that for all $f\in\mathcal{F}$
and all $(y,\bm{x})$ we have $|\mathcal{L}(f(\bm{x}),y)|\leq1$.
In addition, assume that for any fixed $y$, the univariate function $\mathcal{L}(\cdot,y)$ 
is Lipschitz with constant 1. Then with probability at least $1-\delta$ over the sample $S\triangleq\{(y_{i},\bm{x}_{i})\}_{1\leq i\leq n}\overset{\mathrm{i.i.d.}}{\sim}\mathcal{D}$, one has for all $f\in\mathcal{F}$
\[
\underbrace{\vphantom{\frac{1}{n}\sum_{i=1}^{n}}\mathbb{E}_{(y,\bm{x})\sim \mathcal{D}}\left[\mathcal{L}\left(f(\bm{x}), y\right)\right]}_{\mathrm{out}\text{-}\mathrm{of}\text{-}\mathrm{sample\;error}}\leq\underbrace{\frac{1}{n}\sum_{i=1}^{n}\mathcal{L}\left(f(\bm{x}_{i}),y_{i}\right)}_{\mathrm{in}\text{-}\mathrm{sample\;error}}+2\mathcal{R}_{S}\left(\mathcal{F}\right)+4\sqrt{\frac{\log\left(4/\delta\right)}{n}}.
\]
\end{thm}
%The intuition can indeed
%be made precise; see \citep[Chapter 26]{shalev2014understanding} for
%a formal statement about the relationship between Rademacher complexity
%and the generalization error.
In English, the generalization gap of any function $f$ that lies in $\mathcal{F}$ is well-controlled as long as the Rademacher complexity of $\mathcal{F}$ is not too large. With this connection in place, we single out the
following complexity bound.

\begin{thm}[Theorem 1 in~\citep{golowich2017size}]Consider the
function class $\mathcal{F}_{L}$ in~(\ref{eq:function-class-ffn}),
where each parameter $\bm{W}_{l}$ has Frobenius norm at most $M_{\mathrm{F}}(l)$.
Further suppose that the element-wise activation function $\sigma(\cdot)$
is $1$-Lipschitz and positive-homogeneous (i.e., $\sigma(c\cdot x)=c\sigma(x)$
for all $c\geq0$). Then the empirical Rademacher complexity~(\ref{eq:empirical-rademacher-complexity})
w.r.t. $S\triangleq\{\bm{x}_{i}\}_{1 \leq i \leq n}$ satisfies
\begin{equation}
\mathcal{R}_{S}\left(\mathcal{F}_{L}\right)\leq\max_{i}\|\bm{x}_{i}\|_{2}\cdot\frac{4\sqrt{L}\prod_{l=1}^{L}M_{\mathrm{F}}(l)}{\sqrt{n}}.\label{eq:rademacher-nn}
\end{equation}
\end{thm}The upper bound of the empirical Rademacher complexity
(\ref{eq:rademacher-nn}) is in a similar vein to that of linear functions
with bounded norm, i.e., $\max_{i}\|\bm{x}_{i}\|_{2}M/\sqrt{n}$,
where $\sqrt{L}\prod_{l=1}^{L}M_{\mathrm{F}}(l)$ plays the role of
$M$ in the latter case. Moreover, ignoring the term $\sqrt{L}$,
the upper bound (\ref{eq:rademacher-nn}) does not depend on the size
of the network in an explicit way  if $M_F(l)$ sharply concentrates around $1$. This reveals that the capacity
of the neural network is well-controlled, regardless of the number
of parameters, as long as the Frobenius norm of the parameters is
bounded. Extensions to other norm constraints, e.g., spectral norm
constraints, path norm constraints have been considered by \cite{neyshabur2015norm,NIPS2017_7204,li2018tighter,klusowski2016risk, E19}.
This line of work improves upon traditional capacity analysis of neural
networks in the over-parametrized setting, because the upper bounds derived are often size-independent.
Having said this, two important remarks are in order: (1) the upper bounds (e.g., $\prod_{l=1}^{L}M_{\mathrm{F}}(l)$)
involve implicit dependence on the size of the weight matrix and
the depth of the neural network, which is hard to characterize; (2) the upper bound on the Rademacher complexity offers a uniform bound
over all functions in the function class, which is a pure statistical
result. However, it stays silent about how and why standard training
algorithms like SGD can obtain a function whose parameters have small
norms.

\subsection{Algorithm-dependent controls}

In this subsection, we bring computational thinking into statistics and investigate the role of algorithms in the generalization power of deep learning. The consideration of algorithms is quite natural and well motivated: (1) local/global minima reached by different algorithms can exhibit totally different generalization behaviors due to extreme nonconvexity, which marks a huge difference from traditional models, (2) the \emph{effective }capacity of neural nets is possibly not large, since a particular algorithm does not explore the entire parameter space.

These demonstrate the fact that on top of the complexity of the function class,
the inherent property of the algorithm we use plays an important role in the generalization ability of deep learning. In what follows, we survey three different ways to obtain upper bounds on the generalization errors by exploiting properties of the algorithms.
\subsubsection{Mean field view of neural nets} As we have emphasized, modern deep learning models are highly over-parametrized. %, i.e., the number of parameters is much larger than the number of inputs$\,$/$\,$input dimensions.
A line of work~\citep{mei2018mean,sirignano2018mean,rotskoff2018neural,chizat2018global,mei2019mean,javanmard2019analysis}
%investigates the asymptotic regime where the number of weights is infinite and tries to approximate the dynamics of SGD using solutions to certain partial different equations.
approximates the ensemble of weights by an asymptotic limit as the number of hidden units tends to infinity, so that the dynamics of SGD can be studied via certain partial different equations.

More specifically, let $\hat f(\xx; \btheta) = N^{-1} \sum_{i=1}^N \sigma(\btheta_i^\top \xx)$ be a function given by a one-hidden-layer neural net with $N$ hidden units, where $\sigma(\cdot)$ is the ReLU activation function and parameters $\btheta \triangleq [\btheta_1,\ldots,\btheta_N]^\top \in \R^{N \times d}$ are suitably randomly initialized. Consider the regression setting where we want to minimize the population risk $R_N(\btheta) =  \E[(y - \hat f(\xx; \btheta))^2]$ over parameters $\btheta$. A key observation is that this population risk depends on the parameters $\btheta$ only through its empirical distribution, i.e., $\hat \rho^{(N)} = N^{-1} \sum_{i=1}^N \delta_{\btheta_i}$ where $\delta_{\btheta_i}$ is a point mass at $\btheta_i$. This motivates us to view express $R_N(\btheta)$ equivalently as $R(\hat \rho^{(N)})$, where $R(\cdot)$ is a functional that maps distributions to real numbers. Running SGD for $R_N(\cdot)$---in a suitable scaling limit---results in a gradient flow on the space of distributions endowed with the Wasserstein metric that minimizes $R(\cdot)$. It turns out that the  empirical distribution $\hat \rho^{(N)}_k$ of the parameters after $k$ steps of SGD is well approximated by the gradient follow, as long as the the neural net is over-parametrized (i.e., $N\gg d$) and the number of steps is not too large. In particular, \cite{mei2018mean} have shown that under certain regularity conditions,
\[
\sup_{k\in[0,T/\varepsilon]\cap\mathbb{N}}\left| R(\hat \rho^{(N)})-R\left(\rho_{k\varepsilon}\right)\right|\lesssim e^{T}\sqrt{\frac{1}{N}\vee\varepsilon}\cdot\sqrt{d+\log\frac{N}{\varepsilon}},
\]
where $\varepsilon >0$ is an proxy for the step size of SGD and $\rho_{k\varepsilon}$ is the distribution of the gradient flow at time $k\varepsilon$.  In words, the out-of-sample error under $\btheta^{k}$ generated by SGD is well-approximated by that of $\rho_{k\varepsilon}$.
Viewing the optimization problem from the distributional aspect greatly simplifies the problem conceptually, as the complicated optimization problem is now passed into its limit version---for this reason, this analytical approach is called the mean field perspective. In particular, \cite{mei2018mean} further demonstrated that in some simple settings, the out-of-sample error $R(\rho_{k\varepsilon})$ of the distributional limit can be fully characterized. Nevertheless, how well does $R(\rho_{k\varepsilon})$ perform and how fast it converges remain largely open for general problems.

%Traditional wisdom may warn us against over-parametrized models, due to the potential risk of inflated generalization errors. However, at least for one-hidden-layer neural nets, it has been shown that SGD can steer away from this danger .

%The key insight is that, if we view $R_N$ as a function with respect to the parameter distribution (not parameters), then $R_N$ is a convex function; moreover,   \citep{mei2018mean} has established rigorous results, which show that increasing $N$ after reaching some threshold $N_0$ essentially does not inflate the excess risk $R_N(\btheta^k) - \inf_{\btheta} R_N(\btheta)$, where $\btheta^k$ is the parameters after $k$ SGD iterations.


\subsubsection{Stability} A second way to understand the generalization ability of deep learning is through the \emph{stability} of SGD. An algorithm is considered
stable if a slight change of the input does not alter the output much. It has long been observed that a stable algorithm has a small generalization gap; examples include $k$ nearest neighbors~\citep{rogers1978finite, devroye1979distribution}, bagging~\citep{breiman1996bagging, breiman1996heuristics}, etc. The precise connection between stability and generalization gap is stated by~\cite{bousquet2002stability, shalev2010learnability}. In what follows, we formalize the idea of \emph{stability} and its connection with the generalization
gap. Let $\mathcal{A}$ denote an algorithm (possibly randomized) which takes a sample $S\triangleq\{(y_{i},\bm{x}_{i})\}_{1 \leq i \leq n}$
of size $n$ and returns an estimated parameter $\hat{\bm{\theta}}\triangleq\mathcal{A}(S)$.
Following \cite{hardt2015train}, we have the following definition
for \emph{stability}.

\begin{definition}An algorithm (possibly randomized) $\mathcal{A}$
is $\varepsilon$-uniformly stable with respect to the loss function
$\mathcal{L}(\cdot,\cdot)$ if for all datasets $S,S'$ of size $n$ which differ
in at most one example, one has 
\[
\sup_{\bm{x},y}\mathbb{E}_{\mathcal{A}}\left[\mathcal{L}\left(f(\bm{x};\mathcal{A}\left(S\right)),y\right)-\mathcal{L}\left(f(\bm{x}; \mathcal{A}\left(S'\right)),y\right)\right]\leq\varepsilon.
\]
Here the expectation is taken w.r.t.~the randomness in the algorithm
$\mathcal{A}$ and $\varepsilon$ might depend on $n$. The loss function $\mathcal{L}(\cdot,\cdot)$ takes an example (say $(\bm{x},y)$) and the estimated
parameter (say $\mathcal{A}(S)$) as inputs and outputs a real value. \end{definition}

Surprisingly, an $\varepsilon$-uniformly stable algorithm incurs
small generalization gap \emph{in expectation}, which is stated in the following
lemma.
\begin{lem}[Theorem 2.2 in \citealp{hardt2015train}]\label{lemma:stability-generalization}Let
$\mathcal{A}$ be $\varepsilon$-uniformly stable. Then the expected
generalization gap is no larger than $\varepsilon$, i.e.,
\begin{equation}
\left|\mathbb{E}_{\mathcal{A},S}\left[\frac{1}{n}\sum_{i=1}^{n}\mathcal{L}(f(\bm{x}_{i};\mathcal{A}\left(S\right)),y_{i})-\mathbb{E}_{(\bm{x},y)\sim\mathcal{D}}\left[\mathcal{L}\left(f(\bm{x};\mathcal{A}\left(S\right)),y\right)\right]\right]\right|\leq\varepsilon.\label{eq:stability-generalization-gap}
\end{equation}
\end{lem}

%As a side note, the message conveyed in Lemma~\ref{lemma:stability-generalization} is very similar to that of Stein's unbiased risk estimator~\citep{stein1981estimation}. In the latter case, the gap between the population risk and the empirical risk of a mean estimator is exactly quantified by the sensitivity of the estimator to the training data.
With Lemma~\ref{lemma:stability-generalization} in hand, it suffices to prove stability bound on specific
algorithms. It turns out that SGD introduced
in Section~\ref{sec:opt} is uniformly stable when solving smooth
nonconvex functions.

\begin{thm}[Theorem 3.12 in~\cite{hardt2015train}]\label{thm:sgd-stability}Assume
that for any fixed $(y, \bm{x})$, the loss function $\mathcal{L}(f(x;\btheta), y)$, viewed as a function of $\btheta$,
is $L$-Lipschitz and $\beta$-smooth. Consider running SGD on the
empirical loss function with decaying step size $\alpha_{t}\leq c/t$,
where $c$ is some small absolute constant. Then SGD is uniformly
stable with
\[
\varepsilon\lesssim\frac{T^{1-\frac{1}{\beta c+1}}}{n},
\]
where we have ignored the dependency on $\beta,c$ and $L$. \end{thm}Theorem~\ref{thm:sgd-stability}
reveals that SGD operating on nonconvex loss functions is indeed uniformly
stable as long as the number of steps $T$ is not large compared with
$n$. This together with Lemma~\ref{lemma:stability-generalization}
demonstrates the generalization ability of SGD in expectation. %In addition, benefits of various optimization tricks including weight decay and  dropout can be understood through the lens of stability; see~\cite[Section 4]{hardt2015train} for details.
Nevertheless, two important limitations are worth mentioning.  First, Lemma~\ref{lemma:stability-generalization} provides an upper bound on the out-of-sample error \emph{in expectation}, but ideally, instead of an on-average guarantee under $\mathbb{E}_{\mathcal{A},S}$, we would like to have a high probability guarantee as in the convex case~\citep{feldman2019high}.
%. However, it does not reveal any information on its tail behavior. Constructing high probability bounds over the randomness of the data and the algorithm seems necessary to prove the benefits of stability. Such an effort has been made in the case with convex loss functions and deterministic algorithms~\citep{feldman2019high}.
Second, controlling the generalization gap alone is not enough to achieve a small out-of-sample error, since it is unclear whether SGD can achieve a small training error within $T$ steps.
%Second, controlling the generalization gap alone as in Lemma~\ref{lemma:stability-generalization} is not enough to achieve a small out-of-sample error; it remains unclear whether SGD can achieve a small training error in $T$ steps where the stability upper bound (cf.~Theorem~\ref{thm:sgd-stability}) does not blow up.


\subsubsection{Implicit regularization} %As we alluded to earlier, modern deep neural networks have more parameters (say $10^{9}$) than the number of samples (say $10^{6}$). Conventional wisdom informs us that one should apply some regularization techniques (e.g., weight decay) so that the model will not overfit the data. In practice, however, neural networks without explicit regularization generalize well. This phenomenon motivates researchers to look at the regularization effects introduced by training algorithms (e.g., stochastic gradient descent) in this over-parametrized regime, where there might exits multiple, if not infinite global minima of the empirical loss~(\ref{eq:ERM_for_DL}).  The hope is that different algorithmsbias us towards different solutions, with possibly different generalization powers.

In the presence of over-parametrization (number of parameters larger than the sample size), conventional wisdom informs us that we should apply some regularization techniques (e.g., $\ell_1\,/\, \ell_2$ regularization) so that the model will not overfit the data. However, in practice, neural networks without explicit regularization generalize well. This phenomenon motivates researchers to look at the regularization effects introduced by training algorithms (e.g., SGD) in this over-parametrized regime. While there might exits multiple, if not infinite global minima of the empirical loss~(\ref{eq:ERM_for_DL}), it is possible that practical algorithms tend to converge to solutions with better generalization powers.

Take the underdetermined linear system $\bm{X}\bm{\theta}=\bm{y}$
as a starting point. Here $\bm{X}\in\mathbb{R}^{n\times p}$ and $\bm{\theta}\in\mathbb{R}^{p}$
with $p$ much larger than $n$. Running gradient descent on the loss
$\frac{1}{2}\|\bm{X}\bm{\theta}-\bm{y}\|_{2}^{2}$ from the origin
(i.e., $\bm{\theta}^{0}=\bm{0}$) results in the solution with the minimum Euclidean
norm, that is GD converges to
\begin{align*}
\min_{\bm{\theta}\in\mathbb{R}^{p}} & \quad\|\bm{\theta}\|_{2}\qquad\text{subject to}\quad\bm{X}\bm{\theta}=\bm{y}.
\end{align*}
In words, without any $\ell_{2}$ regularization in the loss function,
gradient descent automatically finds the solution with the least $\ell_{2}$ norm.
This phenomenon, often called as \emph{implicit regularization}, not
only has been empirically observed in training neural networks, but
also has been theoretically understood in some simplified cases, e.g., logistic regression with separable data.
In logistic regression, given a training
set $\{(y_{i},\bm{x}_{i})\}_{1\leq i \leq n}$ with $\bm{x}_{i}\in\mathbb{R}^{p}$
and $y_{i}\in\{1,-1\}$, one aims to fit a logistic regression model
by solving the following program:
\begin{equation}
\min_{\bm{\theta}\in\mathbb{R}^{p}}\qquad\frac{1}{n}\sum_{i=1}^{n}\ell\big(y_{i}\bm{x}_{i}^\top \bm{\theta}^{t}\big).\label{eq:loss-logistic}
\end{equation}
Here, $\ell(u)\triangleq\log(1+e^{-u})$ denotes the logistic loss. Further
assume that the data is separable, i.e., there exists $\bm{\theta}^{*}\in\mathbb{R}^{p}$
such that $y_{i}\bm{\theta}^{*\top}\bm{x}_{i}>0$ for all $i$. Under this condition,
%the loss function (\ref{eq:loss-logistic}) has a minimal value of
%zero, however, this cannot be attained by any finite $\bm{\theta}$.
the loss function (\ref{eq:loss-logistic}) can be arbitrarily close to zero for certain $\btheta$ with $\| \btheta \|_2 \to \infty$.
What happens when we minimize (\ref{eq:loss-logistic}) using gradient
descent? \cite{soudry2018implicit} uncovers a striking phenomenon.

\begin{thm}[Theorem 3 in \citealp{soudry2018implicit}]Consider
the logistic regression (\ref{eq:loss-logistic}) with separable data.
If we run GD
\[
\bm{\theta}^{t+1}=\bm{\theta}^{t}-\eta\frac{1}{n}\sum_{i=1}^{n}y_{i}\bm{x}_{i}\ell'\big(y_{i}\bm{x}_{i}^\top \bm{\theta}^{t}\big)
\]
from any initialization $\bm{\theta}^{0}$ with appropriate step size
$\eta>0$, %then $\bm{\theta}^{t}$ converges in direction to the $\ell_{2}$
%max margin vector. That is
then normalized $\bm{\theta}^{t}$ converges to a solution with the maximum $\ell_2$ margin. That is,
\begin{equation}
\lim_{t\to\infty}\frac{\bm{\theta}^{t}}{\|\bm{\theta}^{t}\|_{2}}=\hat{\bm{\theta}},\label{eq:converge-in-direction}
\end{equation}
where $\hat{\bm{\theta}}$ is the solution to the hard margin support
vector machine: % (SVM):
\begin{equation}
\hat{\bm{\theta}}\triangleq\arg\min_{\bm{\theta}\in\mathbb{R}^{p}}\|\bm{\theta}\|_{2},\qquad\text{subject to}\quad y_{i}\bm{x}_{i}^\top \bm{\theta} \geq1\quad\text{for all }1\leq i\leq n.\label{eq:SVM}
\end{equation}
\end{thm}

The above theorem reveals that gradient descent, when solving logistic
regression with separable data, implicitly regularizes the iterates towards
the $\ell_{2}$ max margin vector (cf.~(\ref{eq:converge-in-direction})),
without any explicit regularization as in (\ref{eq:SVM}). Similar
results have been obtained by \cite{ji2018risk}. In addition, \cite{gunasekar2018characterizing}
%demonstrates other implicit biases brought by optimization algorithms
%other than gradient descent (e.g., mirror descent, coordinate descent).
%For instance, \cite{gunasekar2018characterizing} shows that for separable
%data with exponential loss (i.e., $l(u)\triangleq e^{-u}$), coordinate
%descent with a suitable choice of step size converges to the $\ell_{1}$
%max margin vector, that is
%\[
%\lim_{t\to\infty}\frac{\bm{\theta}^{t}}{\|\bm{\theta}^{t}\|_{2}}=\arg\min_{\bm{\theta}\in\mathbb{R}^{p}}\|\bm{\theta}\|_{1},\quad\text{subject to}\quad y_{i}\bm{\theta}^{\top}\bm{x}_{i}\geq1\quad\text{for all }1\leq i\leq n,
%\]
%as long as the right hand side is uniquely defined.
studied algorithms other than gradient descent and showed that coordinate descent produces a solution with the maximum $\ell_1$ margin.

Moving beyond logistic regression, which can be viewed as a one-layer neural net, the theoretical understanding of implicit regularization in deeper neural networks is still limited; see~\cite{gunasekar2018implicit} for an illustration in deep linear convolutional neural networks.

%The stability bound we present here is \emph{data-independent}
%in the sense that it does not take the distribution of the data into
%account. Investigation of data-dependent stability bound of SGD has
%later been done in \citep{kuzborskij2017data}.


%\cm{Shall we
%mention the connection to adaboost and Rosset et al?}
%\item \emph{Gradient descent on multi-layer linear (convolutional) neural
%networks. }\cm{The assumptions therein are way to strong and hard to
%interpret.}

%
%\subsection{Notes}
%Another line of work views neural nets as an \emph{inverse problem} where a true neural network is assumed. In such case, algorithms with provable guarantees has been suggested; examples include gradient descent with tensor initialization~\citep{zhong2017recovery}, gradient descent with random initialization~\citep{chen2018gradient}



%\subsection{Notes}
%In addition to the approaches we present here, compression
%bounds \citep{arora2018stronger} and PAC-Bayes bounds \citep{neyshabur2017pac}
%are also popular in controlling the generalization error of deep neural
%nets. 
%\input{humanaid.tex}
\section{Discussion}\label{sec:discuss}

Due to space limitations, we have omitted several important deep learning models; notable examples include deep reinforcement learning~\citep{mnih2015human}, deep probabilistic graphical models~\citep{salakhutdinov2009deep}, variational autoencoders~\citep{kingma2013auto}, transfer learning~\citep{yosinski2014transferable}, etc. Apart from the modeling aspect, interesting theories on generative adversarial networks~\citep{arora2017generalization, bai2018approximability}, recurrent neural networks~\citep{AL2018-RNNgen}, connections with kernel methods~\citep{jacot2018neural,arora2019fine} are also emerging. We have also omitted the inverse-problem view of deep learning where the data are assumed to be generated from a certain neural net and the goal is to recover the weights in the NN with as few examples as possible. Various algorithms (e.g., GD with spectral initialization) have been shown to recover the weights successfully in some simplified settings~\citep{zhong2017recovery, soltanolkotabi2017learning, goel2018learning, mondelli2018connection, chen2018gradient,fu2018local}.

In the end, we identify a few important directions for future research.
\begin{itemize}
\item{\emph{New characterization of data distributions.} The success of deep learning relies on its power of efficiently representing complex functions relevant to real data. Comparatively, classical methods often have optimal guarantee if a problem has a certain known structure, such as smoothness, sparsity, and low-rankness~\citep{stone1982optimal, donoho1994ideal, candes2009power,chen2019noisy}, but they are insufficient for complex data such as images. How to characterize the high-dimensional real data that can free us from known barriers, such as the curse of dimensionality is an interesting open question? % We have discussed hierarchical models \citep{bauer2017deep, schmidt2017nonparametric} in the nonparametric statistics literature in Section~\ref{sec:approx}, which suggests a way of overcoming the curse of dimensionality, but this also leaves many problems open.
}

%These questions are connected to nonparametric statistics, where, as shown in Section~\ref{sec:approx}, hierarchical models can be well expressed by deep neural nets and the accuracy depends on the intrinsic dimensionality. This suggests a way of overcoming the curse of dimensionality, but also leaves many problems open. We believe that a thorough understanding, if ever possible, requires deciphering properties of deep models.
%}
%\item{\emph{Statistical properties of optimization algorithms.} We have seen that particular algorithms may have regularization effects even without explicit regularizers. It is interesting to study (1) how common training practices change the statistical properties of neural networks, and conversely (2) how desirable statistical properties push for new training algorithms and techniques.

%One specific statistical property, which most current deep neural networks lack, is \textit{stability}. It is well known that deep neural nets are susceptible to small adversarial perturbations \citep{szegedy2013intriguing}. Another desirable property is \textit{interpretability}, which is not found for most neural nets. We believe that robust and interpretable training procedures have great practical relevance.
%}
\item \emph{Understanding various computational algorithms for deep learning.} As we have emphasized throughout this survey, computational algorithms (e.g., variants of SGD) play a vital role in the success of deep learning. They allow fast training of deep neural nets and probably contribute towards the good generalization behavior of deep learning in practice. Understanding these computational algorithms and devising better ones are crucial components in understanding deep learning. 


\item{\emph{Robustness.} It has been well documented that DNNs are sensitive to small adversarial perturbations that are indistinguishable to humans~\citep{szegedy2013intriguing}. This raises serious safety issues once if deploy deep learning models in applications such as self-driving cars, healthcare, etc. It is therefore crucial  to refine current training practice to enhance robustness in a principled way~\citep{singh2018hierarchical}.  %We believe that for these problems, future success of deep learning depends on fusing domain knowledge into the training pipeline.
%We believe that addressing these issues well are crucial for a much wider application of deep learning
}


\item{\emph{Low SNRs.} Arguably, for image data and audio data where the signal-to-noise ratio (SNR) is high, deep learning has achieved great success. In many other statistical problems, the SNR may be very low. For example, in financial applications, the firm characteristic and covariates may only explain a small part of the financial returns; in healthcare systems, the uncertainty of an illness may not be predicted well from a patient's medical history. How to adapt deep learning models to excel at such tasks is an interesting direction to pursue?

}


\end{itemize}




%\RMK{Something to (briefly) mention: choice of activation functions, saturation, initialization}

\section*{Acknowledgements}
J.~Fan is supported in part by the NSF grants DMS-1712591 and DMS-1662139, the NIH grant R01-GM072611 and the ONR grant N00014-19-1-2120. We thank Ruying Bao, Yuxin Chen, Chenxi Liu, Weijie Su, Qingcan Wang and Pengkun Yang for helpful comments and discussions.


%\bibliographystyle{plain}

%\bibliographystyle{ims}

\bibliographystyle{plain}
\bibliography{bibDeepLearning}


\end{document}
